% abtex2-modelo-artigo.tex, v-1.9.2 laurocesar
% Copyright 2012-2014 by abnTeX2 group at http://abntex2.googlecode.com/ 
%

% ------------------------------------------------------------------------
% ------------------------------------------------------------------------
% abnTeX2: Modelo de Artigo Acadêmico em conformidade com
% ABNT NBR 6022:2003: Informação e documentação - Artigo em publicação 
% periódica científica impressa - Apresentação
% ------------------------------------------------------------------------
% ------------------------------------------------------------------------

\documentclass[
	% -- opções da classe memoir --
	article,			% indica que é um artigo acadêmico
	11pt,				% tamanho da fonte
	oneside,			% para impressão apenas no verso. Oposto a twoside
	a4paper,			% tamanho do papel. 
	% -- opções da classe abntex2 --
	%chapter=TITLE,		% títulos de capítulos convertidos em letras maiúsculas
	%section=TITLE,		% títulos de seções convertidos em letras maiúsculas
	%subsection=TITLE,	% títulos de subseções convertidos em letras maiúsculas
	%subsubsection=TITLE % títulos de subsubseções convertidos em letras maiúsculas
	% -- opções do pacote babel --
	english,			% idioma adicional para hifenização
	brazil,				% o último idioma é o principal do documento
	sumario=tradicional
	]{abntex2}


% ---
% PACOTES
% ---

% ---
% Pacotes fundamentais 
% ---
\usepackage{listings}
\usepackage{lmodern}			% Usa a fonte Latin Modern
\usepackage[T1]{fontenc}		% Selecao de codigos de fonte.
\usepackage[utf8]{inputenc}		% Codificacao do documento (conversão automática dos acentos)
\usepackage{indentfirst}		% Indenta o primeiro parágrafo de cada seção.
\usepackage{nomencl} 			% Lista de simbolos
\usepackage{color}				% Controle das cores
\usepackage{graphicx}			% Inclusão de gráficos
\usepackage{microtype} 			% para melhorias de justificação
% ---
		
% ---
% Pacotes adicionais, usados apenas no âmbito do Modelo Canônico do abnteX2
% ---
\usepackage{lipsum}				% para geração de dummy text
% ---
		
% ---
% Pacotes de citações
% ---
\usepackage[brazilian,hyperpageref]{backref}	 % Paginas com as citações na bibl
\usepackage[alf]{abntex2cite}	% Citações padrão ABNT
% ---

% ---
% Configurações do pacote backref
% Usado sem a opção hyperpageref de backref
\renewcommand{\backrefpagesname}{Citado na(s) página(s):~}
% Texto padrão antes do número das páginas
\renewcommand{\backref}{}
% Define os textos da citação
\renewcommand*{\backrefalt}[4]{
	\ifcase #1 %
		Nenhuma citação no texto.%
	\or
		Citado na página #2.%
	\else
		Citado #1 vezes nas páginas #2.%
	\fi}%
% ---

% ---
% Informações de dados para CAPA e FOLHA DE ROSTO
% ---
\titulo{Relatório do Trabalho Final INE5404 \\ Grid de RPG Interativo}
\data{Universidade Federal de Santa Catarina,\\Brasil\\28 de Novembro de 2017}
\autor{Gustavo Kundlatsch \and Lucas Godoy \and Teo Gallarza}
% ---

% ---
% Configurações de aparência do PDF final

% alterando o aspecto da cor azul
\definecolor{blue}{RGB}{41,5,195}

% informações do PDF
\makeatletter
\hypersetup{
     	%pagebackref=true,
		pdftitle={\@title}, 
		pdfauthor={\@author},
    	pdfsubject={Modelo de artigo científico com abnTeX2},
	    pdfcreator={LaTeX with abnTeX2},
		pdfkeywords={abnt}{latex}{abntex}{abntex2}{atigo científico}, 
		colorlinks=true,       		% false: boxed links; true: colored links
    	linkcolor=blue,          	% color of internal links
    	citecolor=blue,        		% color of links to bibliography
    	filecolor=magenta,      		% color of file links
		urlcolor=blue,
		bookmarksdepth=4
}
\makeatother
% --- 

% ---
% compila o indice
% ---
\makeindex
% ---

% ---
% Altera as margens padrões
% ---
\setlrmarginsandblock{3cm}{3cm}{*}
\setulmarginsandblock{3cm}{3cm}{*}
\checkandfixthelayout
% ---

% --- 
% Espaçamentos entre linhas e parágrafos 
% --- 

% O tamanho do parágrafo é dado por:
\setlength{\parindent}{1.3cm}

% Controle do espaçamento entre um parágrafo e outro:
\setlength{\parskip}{0.2cm}  % tente também \onelineskip

% Espaçamento simples
\SingleSpacing

% ----
% Início do documento
% ----
\begin{document}

% Retira espaço extra obsoleto entre as frases.
\frenchspacing 

% ----------------------------------------------------------
% ELEMENTOS PRÉ-TEXTUAIS
% ----------------------------------------------------------

%---
%
% Se desejar escrever o artigo em duas colunas, descomente a linha abaixo
% e a linha com o texto ``FIM DE ARTIGO EM DUAS COLUNAS''.
% \twocolumn[    		% INICIO DE ARTIGO EM DUAS COLUNAS
%
%---
% página de titulo
\maketitle




% ]  				% FIM DE ARTIGO EM DUAS COLUNAS
% ---

% ----------------------------------------------------------
% ELEMENTOS TEXTUAIS
% ----------------------------------------------------------
\textual

% ----------------------------------------------------------
% Introdução
% ----------------------------------------------------------

\section{Análise do Problema}

Role Playing Games (RPG) são jogos constituídos por um mestre e seus jogadores, onde o primeiro narra uma história e o segundo atua, como em um teatro de mesa. Para isso, utiliza-se um sistema de jogo, que dita as regras que precisam ser seguidas durante a partida. Nosso programa tem como objetivo fornecer uma ferramenta simples e poderosa que agrega diversos módulos em um lugar só, para resolver a dificuldade de encontrar boas ferramentas gratuitas que funcionem offline e tenham todas as características que nosso projeto engloba.

\section{Uso do Programa}

O programa é dividido em seis módulos:
 \begin{itemize}
   \item Banco de Fichas, onde o usuário pode criar fichas, editar o código html delas e salvar em um txt, para que ter acesso mais tarde.
   
	\item Rolagem de Dados, em que o mestre pode escolher a quantidade e tipo de dados que vai fazer uma rolagem e o programa retorna números aleatórios correspondentes;
    
	\item Bloco de Notas, que é apenas uma ferramenta de interface gráfica para o usuário inserir qualquer texto, que não será salvo pelo programa;
    
    \item Biblioteca, que também é a versão com persistência do bloco de notas (você pode escrever qualquer coisa, atribuir um título e salvar para usar mais tarde).
    
    \item Mesa de Som, onde o usuário escolhe um dos botões para reproduzir seu som correspondente. Ao clicar no último botão, é possível escolher o próprio arquivo .wav para ser reproduzido;
 
    \item o Grid, que é um plano quadriculado onde podem ser adicionados \textit{tokens} (imagens png ou jpg fornecidas pelo usuário) que representam tanto os jogadores da partida quanto monstros e NPCS.
 \end{itemize}
 	
    Além dos módulos, ainda é possível alterar a cor de fundo, para realizar ambientações de acordo com a situação do jogo.
    
 	Nosso programa foi desenvolvido para ser utilizado em duas telas, sendo uma o computador do mestre, que pode ser acessado a qualquer momento de uma sessão de jogo, e a tela do grid, que pode ser um projetor, uma televisão ou qualquer outra tela secundária que todos os outros jogadores podem vizualizar completamente.
    
    \subsection{Exemplo de uso}
    
    Vamos propor uma situação de jogo onde o mestre está narrando uma história para outros três jogadores. Em dado momento, o grupo de jogadores (aventureiros) entram em combate com uma gangue de goblins. Nesse instante o mestre insere no Grid os \textit{tokens} dos jogadores (um guerreiro, um arqueiro e um mago) e de três goblins. Então abre a quatro guias de fichas: uma para cada personagem e uma para a ficha dos monstros. O mestre também abre o módulo de rolagem de dados e de sons, para ambientar. Para concluir o arranjo do software, ele troca a cor do fundo para verda, para combinar com os inimigos.
    
    O mestre então usa a Mesa de Som, apertando o ícone de monstro, que retorna um grunhido feral. Em seguida, ele vê o Ataque corpo-a-corpo dos goblins, e utiliza a rolagem de dados para realizar o teste para tentar acertar o guerreiro. Utilizando um dado de vinte faces (1d20), consegue um acerto, e calcula o dano. Em seguida, na ficha do guerreiro, o mestre pode clicar para editar, se abrirá uma tela com o html da ficha e ele pode alterar o valor de vida.
    
    Esse exemplo simples mostra o quão prático é montar as ferramentas de acordo com a demanda do momento da sessão de RPG. A ideia é que o software seja bastante intuitivo e agradável, atraindo novos usuários.
    
\section{Arquitetura e Projeto}

	O projeto é dividido em dias grandes partes: o Grid e a Interface do Mestre. O grid é a parte visível para os usuários, e como já foi explicado, não precisa de persistência de dados, os ícones entrados são temporários, e a aparência de painel quadriculado vem de uma matriz de botões com borda cuja aparência é transparente. Assim, ficam apenas as linhas, tendo um grid clicável. Para adicionar um novo \textit{token} deve-se clicar em algum quadrado com o botão direito do mouse, abrindo um JPopupMenu, que contém a opção de Adicionar Ícone. Ao clicar nessa opção, o usuário deve escolher uma imagem no formato JPG ou PNG.
    
    A parte da Interface do Mestre é um JDesktopPane, que chama cada um dos outros panes (cada um uma classe própria, com exceção do bloco de notas) que é chamado de acordo com sua necessidade. Cada um dos módulos possui sua própria classe de controle (similar a um modelo MVC), sendo que a Biblioteca e o Banco de Fichas utilizam as classes FileWriter e GenericFileReader para realizar as operações de persistência, e o banco de fichas foi modelado a partir da interface Sheet, que é implementada por cada tipo de ficha (NPCSheetManager, MonsterSheetManager e CharacterSheetManager).
    
\section{Comportamento e Resultados}

	A integração entre os módulos da aplicação já foi descrita no item anterior, e esse design permite que o programa funcione de maneira bastante independente. Os principais instrumentos utilizados que não vimos em aula foi a classe FileChooser e as classes de som do javax, ambas essenciais para que nosso projeto atingisse seu objetivo. Abaixo, estão explicados esses trechos de código:
	
    \begin{lstlisting}
try 
   {
   AudioInputStream audioInputStream = AudioSystem.getAudioInputStream
     (new File(soundName).getAbsoluteFile( ));
   Clip clip = AudioSystem.getClip( );
   clip.open(audioInputStream);
   clip.start( );
   }
   catch(UnsupportedAudioFileException e)
   {
   JOptionPane.showMessageDialog(null,
       "Formato invalido, escolha um arquivo .wav");
   }
   catch(Exception ex)
   {
     System.out.println("Error with playing sound.");
     ex.printStackTrace( );
   }
	\end{lstlisting}
	
    O código acima faz a reprodução do som. Ele está dentro do construtor da classe Soundboard, que recebe uma string como parametro (soundName), e será utilizado pela classe SoundBoardPanel nos \textit{listeners} de cada um dos botões desse módulo.

\hfill

    \begin{lstlisting}
class PopupActionListener implements ActionListener{
    public void actionPerformed(ActionEvent e){
	    if(e.getSource()==items[0]){
		    JFileChooser chooser = new JFileChooser();
		    chooser.setFileSelectionMode(JFileChooser.FILES_ONLY);
		} 
	}
}
    \end{lstlisting}
    
    Esse segundo código é o \textit{listener} da opção "Adicionar Ícone" do popup menu que aparece quando algum dos botões do grid é clicado. A classe JFileChooser é utilizada para pegar um arquivo, que posteriormente será ícone do button correspondente.
    
    
    


% ---
\section*{Considerações finais}
\addcontentsline{toc}{section}{Considerações finais}

Nosso programa entrega todas as funcionalidades que planejamos, exceto a possibilidade de desenhar sobre o grid quadriculado, para que o mestre faça marcações e desenhos para simular o mapa. Apesar disso, o funcionamento do restante do programa está bastante claro, intuitivo e reutilizável. Entretanto, o código foi mal escalonado, pois em etapas finais do código já havia muitas linhas nas classes principais, o que em certos momentos tornava a implementação de outras funcionalidades um tanto quanto confuso. Mesmo assim, a decisão de fazer o código em módulos independentes permitiu que trabalhássemos em partes pequenas, que no final se juntaram e constituíram o todo. Nenhum padrão de projeto foi utilizado, mas as classes DMInterface e Grid poderiam ser singletons, por exemplo.

A parte mais longa e trabalhosa foi a decisão e implementação da interface gráfica, mas no final atingimos um resultado bom, com painéis bastante simples e elegantes, sem nenhuma dificuldade para o usuário final.


% ]  				% FIM DE ARTIGO EM DUAS COLUNAS
% ---
\section*{Bibliografia}
Foram utilizadas todos os slides fornecidos pela professora, principalmente o referenta a aula do dia 17/10 (GUI 4).
% -------------------
\end{document}