\section{Trabalhos relacionados}

Existem diversas abordagens para otimizar as percepções recebidas por um agente, isto é, garantir que todas as informações coletadas pelos sensores sejam utilizadas da melhor maneira possível. Diversos artigos apresentam processos de detectar percepções inválidas e tratá-las, de maneira similar ao modelo que será proposto neste trabalho.

Por exemplo, no artigo \emph{Scalable perception for BDI-agents embodied in virtual environments} \cite{van2011scalable} é definido um \emph{framework} onde os objetos do ambiente são definidos por classes e características e se organizam de maneira hierárquica, e a partir dessa organização o \emph{framework} é capaz de decidir que tipo de percepções o agente deseja perceber de acordo com seus interesses, alterando dinamicamente ao longo do tempo.
 
Outro trabalho similar é o \emph{PMK -- a knowledge processing framework for autonomous robotics perception and manipulation} \cite{Diab_2019}, onde o objetivo dos autores foi criar um mapa de todas as percepções possíveis no ambiente, de maneira que todas as percepções recebidas pelo agente estivessem em seu contexto, evitando anomalias. Todavia, isso não é viável em cenários abertos ou de alta complexidade, pois o mapeamento necessário é extenso demais. O modelo de programação lógica K-CoPMan (\textit{Knowledge enabled Cognitive Perception for Manipulation} ou Percepção Cognitiva Ativada pelo Conheci-mento para Manipulação) apresenta uma maneira do agente criar sua própria base de conhecimentos a partir da percepção passiva, eliminando as anomalias conforme elas são adicionadas ao contexto \cite{pangercic2010}.

Esses exemplos são de artigos que buscam otimizar as percepções recebidas pelo agente de maneira simbólica, mas esse problema pode ser abordado de maneira conexionista também. No artigo \emph{Understanding human intention by connecting perception and action learning in artificial agents} \cite{kim2017understanding} os autores propõem um modelo, chamado OA-SMTRNN (\textit{Object Augmented Supervised Multiple Timescale
Recurrent Neural Network}), para entender a intenção do usuário e responder ativamente da maneira mais adequada, através do uso de redes neurais. Para implementar o reconhecimento de intenção, são focados dois processos cognitivos, a percepção da disponibilidade de objetos e a previsão da ação humana. Com o uso de redes neurais, a ideia é extrair semântica de percepções que seriam inicialmente inválidas para o agente.

O objetivo do modelo proposto neste trabalho é detectar anomalias de maneira simbólica, classificando as percepções recebidas com o uso do contexto do agente como referencial, para então utilizar um processo de planejamento automatizado para gerar novos planos para o agente a partir das percepções inválidas.