\section{Introdução}

Dentro da Inteligência Artificial (IA), agentes inteligentes são entidades capazes de raciocinar a respeito do ambiente em que estão inseridos e tomar decisões baseadas na situação em que se encontram \cite{russell2016artificial}. Dessa maneira, podemos descrever um agente pelo seus processos de percepção, raciocínio e atuação. O agente ocupa um ambiente, do qual recebe informações e no qual atua. O ambiente é o mundo em que o agente está inserido, podendo ser uma construção virtual como uma simulação ou uma parte do mundo real, no caso de um agente físico. Existem diversos tipos de ambientes que podem ser classificados de acordo com o seu fechamento (que determina se agentes de fora do ambiente podem afetar o sistema), dinamismo (a maneira como o ambiente evolui), determinismo (a consistência dos efeitos no ambiente) e cardinalidade (o número de objetos a serem afetados e percebidos) \cite{moya2007towards}.

Uma das maneiras de um agente atualizar seu conhecimento a respeito do ambiente é a percepção, o processo de utilizar sensores para detectar o ambiente e transformar os dados coletados em informações úteis \cite{weyns2004towards}.  O raciocínio, por sua vez, é o processamento das percepções baseado nos objetivos do agente, que resulta em um conjunto ações a serem tomadas através dos atuadores. O processo do raciocínio é comandado pela arquitetura cognitiva do agente, um modelo computacional inspirado na estrutura da mente humana \cite{DYACHENKO2018130}. As arquiteturas cognitivas podem ser divididas em três categorias: simbólicas, emergentes e híbridas \cite{yeCognitivearchitectures}. Arquiteturas simbólicas descrevem o ambiente através de símbolos armazenados em memória em uma base de conhecimentos, e utilizam lógica simbólica para realizar o ciclo de percepção, raciocínio e ação. Arquiteturas emergentes se baseiam na estrutura biológica do cérebro e normalmente utilizam redes neurais em uma estrutura hierárquica para lidar com situações de incerteza. Por fim, arquiteturas híbridas combinam o comportamento emergente e o processamento simbólico para resolver problemas de diversos domínios. 

Todavia, sensores podem apresentar problemas para o processo de percepção por razões como campo de visão, distância do objeto observado, resolução dos sensores e leituras não confiáveis \cite{chrisman1991intelligent}. Tratar deste problema normalmente é responsabilidade da arquitetura cognitiva do agente, pois a arquitetura precisa ser capaz de fazer a ponte entre o ambiente e o conhecimento do agente \cite{langley2009cognitive}.

O objetivo deste trabalho é apresentar um modelo genérico (independente da arquitetura do agente) que pode ser acoplado entre o processo de percepção e raciocínio, capaz de detectar e tratar percepções inválidas para transformá-las em informações úteis através de um processo de criação de novos planos. Esse modelo pressupõe um ambiente aberto (onde agentes externos podem influenciar o ambiente), dinâmico (mudanças no ambiente são causadas por eventos aleatórios) e não determinístico (ações do agente causam resultados diferentes no ambiente, mesmo em situações aparentemente idênticas, pois os resultados variam dependendo da percepção do agente daquele evento).  