\subsection{Formalização}

\label{section:formalizacao}

O bloco básico do modelo de revisão de percepções proposto, chamado de HAIL, é composto por um módulo para alucinação e ilusão $M_{ai}$ e uma função de refinamento $\theta$, conforme descrito na Definição \ref{def:modeloHAIL}. O módulo de ilusão e alucinação é uma quádrupla, apresentada na definição \ref{def:illuHallu}. A função de refinamento é uma função abstrata, cuja entrada é obtida através dos sensores do agentes e a saída é a entrada do módulo de alucinação e ilusão, conforme já foi descrito anteriormente na Seção \ref{refinamento}.
 
 \vspace{0.2cm}
 
 \begin{definition}{}
    O modelo de revisão de percepções HAIL é uma dupla $HAIL = \langle M_{ai}, \theta \rangle$, onde:
    
    \begin{itemize}
        \item $M_{ai}$ é o módulo de ilusão e alucinação; e
        \item $\theta$ é a função de refinamento $\theta(p) = \rho$, onde $p$ é um conjunto de percepções e $\rho$ é um subconjunto próprio de $p$.
    \end{itemize}{}
    \label{def:modeloHAIL}
\end{definition}

Após ter passado pela função $\theta$, as percepções $\rho$ irão passar pelo Algoritmo \ref{algorithm:decisor}, e serão encaminhadas de acordo com sua classificação. O bloco de ilusão e alucinação é descrito por uma quádrupla, com conjuntos de decisores, blocos e uma função de transição.

\vspace{0.2cm}

\begin{definition}
\label{def:illuHallu}
    O bloco de ilusão e alucinação é uma quádrupla $M_{ai} = \langle D, Ab, Ap, \Delta \rangle$, onde:
    
    \begin{itemize}
        \item $D$ é o conjunto de decisores $D = \{d_{a}, d_{h}, d_{i}\}$, onde:
             \begin{itemize}
                \item $d_{a}$ é o decisor de anomalias, definido pela função:
                \[ d_{a} = \left\{ \begin{array}{ll}
                0 & \mbox{se $\rho(x)$ está em $c$\footnotemark};\\
                1 & \mbox{se $\rho(x)$ não está em $c$}.\end{array} \right. \]
             
                \item $d_{h}$ é o decisor de alucinação, definido pela função:
                \[ d_{h} = \left\{ \begin{array}{ll}
                0 & \mbox{se nem $\rho$ nem $(x)$ está em $c$};\\
                1 & \mbox{se $\rho$ ou $(x)$ está em $c$}.\end{array} \right. \]
                
                \item $d_{i}$ é o decisor de ilusão, definido pela função:
                \[ d_{i} = \left\{ \begin{array}{ll}
                0 & \mbox{se $\rho$ está em $c$};\\
                1 & \mbox{se $(x)$ está em $c$}.\end{array} \right. \]
            \end{itemize}
        
        \footnotetext{ $c$ é o contexto do agente, de acordo com a definição \ref{definition::context}.}
        
        \item $Ab$ é o conjunto de blocos avaliadores $Ab = \{Ab_{h}, Ab_{i1}, Ab_{i2}\}$, onde $Ab_{h}$ é o bloco avaliador de alucinações, $Ab_{i1}$ é o bloco avaliador de ilusões classe 1 e $Ab_{i2}$ é o bloco avaliador de ilusões classe 2.
        
        \item $Ap$ é o conjunto de blocos de planejamento automatizado $Ap = \{Ap_{h}, Ap_{i}\}$, onde $Ap_{h}$ é o bloco de planejamento automatizado de alucinações e $Ap_{i}$ é o bloco de planejamento automatizado de ilusões.
        
        \item $\Delta$ é a função de transição definido pela tabela abaixo, onde $out$ é um estado final, que leva a percepção para fora do modelo de revisão de percepções, ou seja, pode tanto significar o encaminhamento de uma percepção válida para o raciocínio do agente quanto o fim da execução de um ciclo de revisão.
        
            \begin{table}[htb]
                \caption{Função de transição $\Delta$ do módulo de ilusão e alucinação.}
                \centering
                \begin{tabular}{c c c c} 
                    \toprule
                    \textbf{Estado} & \textbf{0} & \textbf{1} \\
                    \midrule
                    $d_{a}$     & $out$     & $d_{h}$       \\
                    $d_{h}$     & $Ab_{h}$  & $d_{i}$       \\
                    $d_{i}$     & $Ab_{i1}$ & $Ab_{i2}$     \\
                    $Ab_{h}$    & $out$     & $Ap_{h}$      \\
                    $Ab_{i1}$   & $out$     & $Ap_{i}$      \\
                    $Ab_{i2}$   & $out$     & $Ap_{i}$      \\
                    \bottomrule
                \end{tabular}
                \label{transition-table}
                
            \end{table}
    \end{itemize}{}
\end{definition}{}

O módulo de ilusão e alucinação é o artefato principal do modelo. Ele é subdividido em três partes principais: os decisores, os blocos avaliadores e os blocos de planejamento automatizado. Seu comportamento é descrito por uma função de transição que define o estado atual do módulo.

\vspace{0.2cm}

\begin{definition}
    Um bloco avaliador é uma tripla $Ab_{x} = \langle L, Pf, Cf) \rangle$, $x \in \{h, i1, i2\}$, onde:

    \begin{itemize}
        \item $L$ é uma lista ordenada pelo número de vezes que uma mesma anomalia é dada como entrada;
        \item $Pf$ é a função de processamento, definida abaixo:
            
             \[ Pf = \left\{ \begin{array}{ll}
                        1 & \mbox{se $T_{m}(A) \leq T_{m}(V) * (|A| - |A_{pr}|)$;}\\
                        0 & \mbox{caso contrário}.\end{array} \right. \]
        
            
            Onde:
            
            \begin{itemize}
                \item $T_{m}$ é a função que retorna a média do tempo gasto para processar as percepções de um conjunto;
                \item $A$ é o conjunto de anomalias, $A(x)$ é um elemento específico $x$ e $|A|$ o número de anomalias do conjunto;
                \item $A_{pr}$ é o conjunto de anomalias que já foram validadas para serem processadas pela função de processamento neste ciclo de raciocínio ($A_{pr}$ é instanciada vazia a cada ciclo de raciocínio), e $|A_{pr}|$ o número de anomalias desse conjunto.
                \item $V$ é o conjunto de percepções válidas.
            \end{itemize}{}
        
        \item $Cf$ é a função de limpeza definida abaixo com auxílio da função equação de limpeza $Cf$, sendo $\alpha$ um coeficiente de limpeza variável que precisa ser definido pela instância implementada do modelo (por padrão, toma-se $\alpha = 1$):
        
        \[ Cf = \left\{ \begin{array}{ll}
                        1 & \mbox{se  $Ce = Verdadeiro$;}\\
                        0 & \mbox{caso contrário}.\end{array} \right. \]
            
            \[ Ce = \sum_{i=1}^{|L|} W_{n}(L_{i}) > \alpha \sum_{j=1}^{|L|} W_{1}(L_{j}) \]
            
            Onde:
            
            \begin{itemize}
                \item $L$ é a lista ordenada do bloco, sendo $|L|$ seu número de anomalias e $L_{i}$ a anomalia $i$ da lista.
                \item $W$ é a função peso da anomalia $L_{i}$ definida como $W(L_{i}) = |L_{i}|$, sendo $|L_{i}|$ o peso da anomalia especificada (número de entradas recebidas dessa mesma anomalia na lista). A função $W$ é utilizada para especificar as seguintes funções:
                \\
                
                    (i) $ W_{1}(L_{i}) = \left\{ \begin{array}{ll}
                        1 & \mbox{se $W(L_{i}) = 1$;}\\
                        0 & \mbox{caso contrário}.\end{array} \right. $
                \\
                
                    (ii) $ W_{n}(L_{i}) = \left\{ \begin{array}{ll}
                        W(L{i}) & \mbox{se $W(L_{i}) > 1$;}\\
                        0 & \mbox{caso contrário}.\end{array} \right. $
            \end{itemize}{}
    \end{itemize}
\end{definition}{}

\vspace{0.2cm}

O bloco de planejamento automatizado recebe como entrada uma anomalia de um dos blocos avaliadores quando a função de processamento retorna verdadeiro, e pode ser definido da seguinte maneira:

\vspace{0.2cm}

\begin{definition}
    Um bloco de planejamento automatizado é uma instância do modelo conceitual de planejamento automatizado (Definição \ref{definition::autoplanning}).
\end{definition}

