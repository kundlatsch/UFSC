\section{Conclusão}

Agentes são entidades autônomas que, inseridas em um ambiente, são capazes de tomar decisões de maneira autônoma. Para se comunicar com o ambiente no qual estão inseridos, os agentes realizam o processo de percepção, utilizando sensores para receber diversos tipos de informações do mundo ao seu redor. Todavia, as percepções recebidas podem ser incorretas por diversos motivos.

Neste trabalho, apresentamos um modelo genérico capaz de detectar percepções inválidas, classificá-las de acordo com suas características e criar novos planos a partir delas. Esse modelo foi implementado e experimentos foram feitos para mostrar que seu funcionamento era de acordo com o esperado.

Para os trabalhos futuros, é necessário criar versões alternativas de cada um de seus componentes para validar se seu comportamento é o melhor possível. Além disso, é necessário testar o modelo em um ambiente real, acoplado a uma arquitetura específica, para testar seu desempenho em diferentes situações.