% ----------------------------------------------------------
\chapter{Código do analisador}
% ----------------------------------------------------------

O simulador foi implementado utilizando a versão 3.7 do Python, e o gerencia-mento de pacotes foi feito com o Poetry. As dependências estão disponíveis no arquivo \texttt{pyproject.toml}.

\begin{section}{\_\_main\_\_.py}
    \begin{minted}
    [
    frame=lines,
    framesep=2mm,
    baselinestretch=1.2,
    breaklines=True,
    fontsize=\footnotesize,
    linenos
    ]
    {python}

from factorial2k import *
from data import RESULTS_PLANS_CREATED
import seaborn as sb
import pandas as pd
import matplotlib.pyplot as plt
from statistics import mean


def normalize(values):
    vmin = min(values)
    vmax = max(values)
    return [(v - vmin) / (vmax - vmin) for v in values]


def main():
    FACTORS_2KN = 4
    results = RESULTS_PLANS_CREATED
    EFFECTS_2KN = effects_table_method(FACTORS_2KN, results)

    sb.set(style="darkgrid")

    # 1 - Independent Errors
    x = normalize(get_residuals(results, EFFECTS_2KN, FACTORS_2KN))
    y = normalize(nonlinear_regression(EFFECTS_2KN, FACTORS_2KN)) #y_hat
    plt.plot(x, y, 'bo')
    plt.xlabel('Residuals', fontsize=10)
    plt.ylabel('Predicted Value', fontsize=10)
    plt.show()

    # 2 - Normally Distributed Errors
    r = []
    for re in results:
        r.append(re[0])
    x = normalize(r)
    y = normalize(nonlinear_regression(EFFECTS_2KN, FACTORS_2KN)) #y_hat
    plt.plot([0,1], 'r')
    plt.plot(x, y, 'bo')
    plt.xlabel('Simulation Value', fontsize=10)
    plt.ylabel('Predicted Value', fontsize=10)
    plt.show()

    # 3 - Constant Standard Deviation of Errors
    x = [i for i in range(2 ** FACTORS_2KN)]
    y = normalize(nonlinear_regression(EFFECTS_2KN, FACTORS_2KN))  # y_hat
    m = mean(y)
    y2 = [m for i in range(2 ** FACTORS_2KN)]
    plt.plot(x, y, "bo")
    plt.plot(x, y2, "r--")
    plt.ylabel("Predicted Value", fontsize=10)
    plt.show()


if __name__ == "__main__":
    main()

    \end{minted}
\end{section}
\newpage

\begin{section}{factorial2k.py}
    \begin{minted}
    [
    frame=lines,
    framesep=2mm,
    baselinestretch=1.2,
    breaklines=True,
    fontsize=\footnotesize,
    linenos
    ]
    {python}

import pyDOE2 as pd
import numpy as np
from itertools import combinations
import string


def create_sign_table(factors):
    # This combination string is used to create the factorial_string,
    # because pyDOE2 input is the comlumns that we want to create in
    # the signal table. Here we are using it to generate the 2k^N table,
    # so we need the combination of N comlumns.

    combination_string = ""
    for _, letter in zip(range(0, factors), string.ascii_lowercase):
        combination_string += letter

    _combinations = [
        "".join(l)
        for i in range(len(combination_string))
        for l in combinations(combination_string, i + 1)
    ]

    factors_string = " ".join(_combinations)
    factorial_columns = pd.fracfact(factors_string)

    image_column = np.ones((2 ** factors, 1))
    sign_table = np.hstack((image_column, factorial_columns))
    return sign_table


def effects_table_method(factors, results):
    results_column = np.array(results)
    sign_table = create_sign_table(factors)

    tpf = 2 ** factors
    coefficients = [
        float(np.dot(sign_table[:, i], results_column) / tpf) for i in range(tpf)
    ]

    return coefficients


def variation_allocation(effects):
    # The first elements represents q0, and is not important here
    effects.pop(0)

    allocation = [4 * effect * effect for effect in effects]
    total_variation = sum(allocation)

    percentages = [(100 * a) / total_variation for a in allocation]

    return percentages


def get_residuals(results, effects, factors):
    predicted_results = nonlinear_regression(effects, factors)
    return [float(predicted_results[i] - results[i]) for i in range(len(results))]


def nonlinear_regression(coefficients, factors):
    sign_table = create_sign_table(factors)
    return [regression_eq(coefficients, r) for r in sign_table]


def regression_eq(coefficients, independent_values):
    map(lambda x: np.array(x), (independent_values, coefficients))
    return sum(np.multiply(coefficients, independent_values))

    \end{minted}
\end{section}
\newpage

\begin{section}{data.py}
    \begin{minted}
    [
    frame=lines,
    framesep=2mm,
    baselinestretch=1.2,
    breaklines=True,
    fontsize=\footnotesize,
    linenos
    ]
    {python}

# Model
# RESULTS = [[1], [9], [5], [13], [3], [11], [7], [15], [2], [10], [6], [14], [4], [12], [8], [16]]
# This are the corresponding indexes in the results table.
# TODO: Get results automatically from the simulation files

# Vtime
RESULTS_VTIME = [
    [4874.60],
    [3226.35],
    [152096.65],
    [48257.85],
    [20800.40],
    [228349.49],
    [168025.60],
    [273475.84],
    [4999.93],
    [5000.00],
    [153436.30],
    [140958.00],
    [20813.00],
    [304313.00],
    [168032.00],
    [312032.00],
]

# Perceptions Processed
RESULTS_PECEPTIONS_PROCESSED = [
    [4749.20],
    [1452.70],
    [4749.10],
    [1452.63],
    [4749.20],
    [1454.77],
    [4749.20],
    [1453.88],
    [4751.26],
    [2606.74],
    [4816.31],
    [4745.69],
    [4750.14],
    [1295.42],
    [4750.03],
    [1311.50],
]

# Plans Created
RESULTS_PLANS_CREATED = [
    [250.80],
    [3547.3],
    [250.9],
    [3547.37],
    [250.8],
    [3545.23],
    [250.8],
    [3546.12],
    [502],
    [9304],
    [2936.6],
    [4066.4],
    [251],
    [4751],
    [251],
    [4751],
]


# Test
RESULTS_2K2 = [[15], [45], [25], [75]]

    \end{minted}
\end{section}
\newpage

\begin{section}{tests/test\_factorial2k.py}
    \begin{minted}
    [
    frame=lines,
    framesep=2mm,
    baselinestretch=1.2,
    breaklines=True,
    fontsize=\footnotesize,
    linenos
    ]
    {python}

from pytest import approx
from src.factorial2k import (
    effects_table_method,
    variation_allocation,
    nonlinear_regression,
)


# 2K^2 FACTORIAL DESIGN TESTS #
FACTORS_2K2 = 2
RESULTS_2K2 = [[15], [45], [25], [75]]
EFFECTS_2K2 = effects_table_method(FACTORS_2K2, RESULTS_2K2)
ALLOCATION_2K2 = variation_allocation(EFFECTS_2K2.copy())


def test_effects2k2():
    assert EFFECTS_2K2 == [40.0, 20.0, 10.0, 5.0]


def test_variation2k2():
    assert ALLOCATION_2K2 == approx([76, 19, 5], rel=1e-1, abs=1)


# 2K^n FACTORIAL DESIGN TESTS #
FACTORS_2KN = 3
RESULTS_2KN = [[14], [22], [10], [34], [46], [58], [50], [86]]
EFFECTS_2KN = effects_table_method(FACTORS_2KN, RESULTS_2KN)
ALLOCATION_2KN = variation_allocation(EFFECTS_2KN.copy())


def test_effects2kN():
    assert EFFECTS_2KN == [40.0, 10.0, 5.0, 20.0, 5.0, 2.0, 3.0, 1.0]


def test_variation2kN():
    assert ALLOCATION_2KN == approx([18, 4, 71, 4, 1, 2, 0], rel=1e-1, abs=1)


# VERIFYING THE ASSUMPTIONS


def test_nonlinear_regression():
    assert nonlinear_regression(EFFECTS_2K2, FACTORS_2K2) == [15.0, 45.0, 25.0, 75.0]

    \end{minted}
\end{section}
\newpage

\begin{section}{pyproject.toml}
    \begin{minted}
    [
    frame=lines,
    framesep=2mm,
    baselinestretch=1.2,
    breaklines=True,
    fontsize=\footnotesize,
    linenos
    ]
    {python}

[tool.poetry]
name = "factorial-design"
version = "0.1.0"
description = ""
authors = ["kundlatsch <gustavo.kundlatsch@gmail.com>"]

[tool.poetry.dependencies]
python = "^3.7"
pyDOE2 = "^1.3.0"
pytest = "^5.4.1"
seaborn = "^0.10.1"
pandas = "^1.0.3"

[tool.poetry.dev-dependencies]

[build-system]
requires = ["poetry>=0.12"]
build-backend = "poetry.masonry.api"

    \end{minted}
\end{section}
\newpage