%================================================================%
%======  Modelo de Monografia ( UFOP - DECOM) ===================%
% Proposta de texto em conformidade com normas da ABNT ----------%
% implementadas pelo projeto abntex2, que pode ser acessado pela %
% página  http://abntex2.googlecode.com/  -----------------------%
%================================================================%
\documentclass[12pt, % tamanho da fonte
   %openright,	     % capítulos começam em página ímpar
	oneside,		  % twoside para impressão em frente e verso.  
	a4paper,			% tamanho do papel. 
	english,			% Idioma adicional para hifenização
    brazil,				% Idioma principal 
    sumario=tradicional % Comente para o sumário ser conforme a opção padrão recomendada pela ABNT NBR 6027:2012.
	]{abntex2}

%------------------------------------------------------------
%------------    Estrutura do texto   -----------------------      

% Pacotes Básicos:
\usepackage[T1]{fontenc}		% Seleção de códigos de fonte.
\usepackage[utf8]{inputenc}		% Codificação do documento (conversão automática dos acentos)
\usepackage{lastpage}		    % Usado pela Ficha catalográfica
\usepackage{indentfirst}		
\usepackage{color}				% Controle das cores
\usepackage{graphicx}			% Inclusão de gráficos
\usepackage{tabularx}
\usepackage{microtype} 		  	% Melhorias da justificação
\usepackage{pdfpages}           %inserir páginas em PDF
% Pacotes Extras:
\usepackage{amsmath,amsthm}     % Símbolos Matemáticos
\usepackage[portuguese, ruled, linesnumbered,commentsnumbered, algo2e, vlined, lined, boxed, algochapter]{algorithm2e} % Algoritmos 
\usepackage{hyperref}      % Criação de links.

% Escolha da formatação das referências Bibliográficas: 
\usepackage[alf,abnt-etal-list=0,abnt-etal-cite=3]{abntex2cite}	% Citações padrão ABNT  (AUTOR, ANO)
\usepackage{etoolbox}
%\usepackage[num]{abntex2cite}  % Citações numéricas (1)
%\citebrackets[] % Usar este comando para a citação numérica aparecer com [].

% Numeração das Figuras e Tabelas
\counterwithin{figure}{chapter}
\counterwithin{table}{chapter}

% Defininção de Cores:
\definecolor{blue}{RGB}{25,25,112}
\makeatletter % informações do PDF
\hypersetup{
    %pagebackref= false,
	pdftitle={\@title}, 
	pdfauthor={\@author},
    %pdfsubject={\imprimirpreambulo},
	pdfcreator={LaTeX with abnTeX2},
	pdfkeywords={abnt}{latex}{abntex}{abntex2}{trabalho acadêmico}, 
	colorlinks=true,    % false: boxed links; true: colored links
    linkcolor=blue,     % color of internal links
    citecolor=blue,     % color of links to bibliography
    filecolor=magenta, 	% color of file links
	urlcolor=blue,
	bookmarksdepth=4
}
\makeatother

% -------------------------------------------- 
% Espaçamentos entre linhas e parágrafos 
\setlength{\parindent}{1.3cm} % O tamanho do parágrafo

% Controle do espaçamento entre um parágrafo e outro:
\setlength{\parskip}{0.2cm}  % tente também \onelineskip

% Definição de ambientes matemáticos em português 
\newtheorem{teorema}{Teorema}[chapter]
\newtheorem{axioma}{Axioma}[chapter]
\newtheorem{corolario}{Corolário}[chapter]
\newtheorem{lema}{Lema}[chapter]
\newtheorem{proposicao}{Proposição}[chapter]
\newtheorem{definicao}{Definição}[chapter]
\newtheorem{exemplo}{Exemplo}[chapter]
\newtheorem{observacao}{Observação}[chapter]

% Novos Comandos
\usepackage{tgtermes}
\renewcommand{\ABNTEXchapterfont}{\rmfamily\bfseries}

% Variáveis adicionais
\providecommand{\imprimirautorcite}{}
\newcommand{\autorcite}[1]{\renewcommand{\imprimirautorcite}{#1}} 
\providecommand{\imprimirsubtitulo}{}
\newcommand{\subtitulo}[1]{\renewcommand{\imprimirsubtitulo}{#1}} 
\providecommand{\imprimirsigla}{}
\newcommand{\sigla}[1]{\renewcommand{\imprimirsigla}{#1}}
\providecommand{\imprimiruf}{}
\newcommand{\uf}[1]{\renewcommand{\imprimiruf}{#1}}
\providecommand{\imprimircurso}{}
\newcommand{\curso}[1]{\renewcommand{\imprimircurso}{#1}}
\providecommand{\imprimirinstituto}{}
\newcommand{\instituto}[1]{\renewcommand{\imprimirinstituto}{#1}}
\providecommand{\imprimirdepartamento}{}
\newcommand{\departamento}[1]{\renewcommand{\imprimirdepartamento}{#1}}
\providecommand{\imprimirano}{}
\newcommand{\ano}[1]{\renewcommand{\imprimirano}{#1}}
\providecommand{\imprimirgrau}{}
\newcommand{\grau}[1]{\renewcommand{\imprimirgrau}{#1}}
\providecommand{\imprimirexaminadorum}{}
\newcommand{\examinadorum}[1]{
    \renewcommand{\imprimirexaminadorum}{#1}}
\providecommand{\imprimirexaminadordois}{}
\newcommand{\examinadordois}[1]{
    \renewcommand{\imprimirexaminadordois}{#1}}
\providecommand{\imprimirexaminadortres}{}
\newcommand{\examinadortres}[1]{
    \renewcommand{\imprimirexaminadortres}{#1}}
\providecommand{\imprimirexaminadorquatro}{}
\newcommand{\examinadorquatro}[1]{
    \renewcommand{\imprimirexaminadorquatro}{#1}}
\providecommand{\imprimirttorientador}{}
\newcommand{\ttorientador}[1]{
    \renewcommand{\imprimirttorientador}{#1}} 
\providecommand{\imprimirttcoorientador}{}
\newcommand{\ttcoorientador}[1]{
    \renewcommand{\imprimirttcoorientador}{#1}}
\providecommand{\imprimirttexaminadorum}{}
\newcommand{\ttexaminadorum}[1]{
    \renewcommand{\imprimirttexaminadorum}{#1}}
\providecommand{\imprimirttexaminadordois}{}
\newcommand{\ttexaminadordois}[1]{\renewcommand{
        \imprimirttexaminadordois}{#1}}
\providecommand{\imprimirttexaminadortres}{}
\newcommand{\ttexaminadortres}[1]{
    \renewcommand{\imprimirttexaminadortres}{#1}}
\providecommand{\imprimirttexaminadorquatro}{}
\newcommand{\ttexaminadorquatro}[1]{
    \renewcommand{\imprimirttexaminadorquatro}{#1}}
		
%----------------------------------------------------
\renewcommand{\imprimircapa}{ % Capa 
\begin{capa}
        \begin{center}
                \begin{DoubleSpace}
                \MakeUppercase{\imprimirinstituicao } \\
                 \MakeUppercase{\imprimirinstituto } \\
                \MakeUppercase{\imprimirdepartamento} \\
                \end{DoubleSpace}
                \vspace{5cm}
				\MakeUppercase{\imprimirautor}  \\
                \imprimirorientadorRotulo ~\imprimirorientador \\
                \imprimircoorientadorRotulo ~\imprimircoorientador \\
                        				
				\vspace{5cm}
             \textbf{{\large\MakeUppercase{\imprimirtitulo}}} \\
			 \textbf{{\large \MakeUppercase{\imprimirsubtitulo}}} \\
				\vfill
        {\large{\imprimirlocal, ~\imprimiruf \\ \imprimirano  }}
        \end{center}
\end{capa}   
} % Capa



%----------------------------------------------------
\renewcommand{\imprimirfolhaderosto}{% folha de rosto
       \begin{center}
                \MakeUppercase{\imprimirinstituicao } \\
                 \MakeUppercase{\imprimirinstituto } \\
                \MakeUppercase{\imprimirdepartamento} \\
                
                \vspace{4cm}
				\MakeUppercase{\imprimirautor}  \\
				\vspace{2cm}
			    \begin{DoubleSpace}
                \MakeUppercase{\textbf{\imprimirtitulo} } \\
                \MakeUppercase{\textbf{\imprimirsubtitulo}} \\
                \end{DoubleSpace} 
      \end{center}
    \vfill 
    \begin{flushright} 
    \parbox{0.6\linewidth}{
    Proposta de TCC submetida para a aprovação na matéria de Introdução ao TCC, requisito parcial para obtenção do grau de Bacharel em Ciências da Computação pela Universidade Federal de Santa Catarina.\\ \\
% 		\imprimirtipotrabalho~ apresentada ao Curso de \imprimircurso~ da \imprimirinstituicao~ como parte dos
% 		requisitos necessários para a obtenção do grau de \imprimirgrau. \\ \\
		\textbf{\imprimirorientadorRotulo}~\imprimirorientador \\
		
		\textbf{\imprimircoorientadorRotulo}~\imprimircoorientador}
   \end{flushright} 
		\vfill
   \begin{center}
   {\large{\imprimirlocal, ~ \imprimiruf \\
   \imprimirano} }
   \end{center} }  % folha de rosto

%----------------------------------------------------


 % Estrutura do documento e pacotes usados. Outros pacotes (packages) devem ser  adicionados ao arquivo structure.tex. 

% -- Informações para Capa e Folha de Rosto: ---------------
\titulo{Revisão de Percepções} 
\subtitulo{}
\autor{Gustavo Emanuel Kundlatsch} \autorcite{Aluno, Gustavo Emanuel Kundlatsch}
\local{Florianópolis} \uf{Santa Catarina}
\data{21 de agosto de 2020} \ano{2020}
\orientador{Me. Thiago Ângelo Gelaim}  % Nome do orientador. Caso seja uma orientadora use o comando ∖orientador[Orientadora:]{Nome}
\ttorientador{Universidade Federal de Santa Catarina} % Instituição do orientador
\coorientador{Prof. Dr. Elder Rizzon Santos}   % Nome do coorientador
\ttcoorientador{Universidade Federal de Santa Catarina} % Instituição do Coorientador
\instituicao{Universidade Federal de Santa Catarina} \sigla{UFSC}
\instituto{Centro Tecnológico}
\departamento{Departamento de Informática e Estatística}
\curso{Ciência da Computação}	
\tipotrabalho{Monografia} % Monografia (Monografia II)
\grau{Bacharel em Ciência da Computação}

%------Nomes dos membros da banca.  
\examinadorum{Prof. Dr. Membro da Banca 1}
\ttexaminadorum{Universidade Federal de ... - UFXX}
\examinadordois{Prof. Dr. Membro da Banca  2}
\ttexaminadordois{Universidade Federal de ... - UFXX}
%\examinadortres{Prof. Dr. Membro da Banca  3} 
\ttexaminadortres{Universidade Federal de ... - UFXX}
%\examinadorquatro{Prof. Dr. Membro da Banca  4}
\ttexaminadorquatro{Universidade Federal de ... - UFXX}



% MEUS PACOTES

\usepackage{booktabs}
\usepackage{graphicx}
\usepackage{enumitem}
\usepackage{csquotes}
\usepackage[linesnumbered]{algorithm2e}
\usepackage{amsthm}
\usepackage{amssymb}
\usepackage{verbatim}
\usepackage{amsmath}
\usepackage{caption}
\usepackage{algorithm}
\usepackage{tocbibind}

 \usepackage[table,xcdraw]{xcolor}
 

\newcolumntype{Y}{>{\centering\arraybackslash}X}

\theoremstyle{definition}
\newtheorem{definition}{Definição}
\newtheorem{example}{Exemplo}[section]

% SIGON
\usepackage{listings}
\renewcommand{\lstlistingname}{Código}
\usepackage{xcolor}
\lstset{%
    language=Python,
    basicstyle=\small\ttfamily,%
    numbers=left, numberstyle=\tiny, stepnumber=1, numbersep=5pt,%
    aboveskip=3mm,
    belowskip=3mm,
    showstringspaces=false,
    columns=flexible,
        morekeywords={plan, action, sensor, actuator, sense, member},
    basicstyle={\small},
    numberstyle=\tiny\color{gray},
    keywordstyle={[2]\color{blue}},
    keywordstyle={[3]\color{red}},
    keywordstyle={[4]\color{black}},
    otherkeywords={String,async,await,Task,var},
    keywords=[2]{communication, beliefs, desires, intentions, planner},
    keywords=[3]{:},
    keywords=[4]{sensor, actuator},
    commentstyle=\color{ForestGreen},
    stringstyle=\color{red},
    captionpos=b, 
    breaklines=true,
    breakatwhitespace=true,
    tabsize=3
}%

% ------------------------------------------------------
\makeindex   

\begin{document} % Início do documento

\frenchspacing  % Retira espaço obsoleto entre as frases.

% ----------------------------------------------------------
% -- Elementos Pré-Textuais: -------------------------------
\pagenumbering{roman}

\imprimircapa  % Capa
\imprimirfolhaderosto % Folha de rosto
\begin{table}[H]
\centering
\begin{tabular}{ll}
\multicolumn{2}{c}{\cellcolor[HTML]{C0C0C0}\textbf{FOLHA DE APROVAÇÃO DE PROPOSTA DE TCC}}                \\ \hline
\multicolumn{1}{|l|}{\textbf{Acadêmico}}            & \multicolumn{1}{l|}{Gustavo Emanuel Kundlatsch}              \\ \hline
\multicolumn{1}{|l|}{\textbf{Título do trabalho}} &
  \multicolumn{1}{l|}{\begin{tabular}[c]{@{}l@{}}Revisão de Percepções\end{tabular}} \\ \hline
\multicolumn{1}{|l|}{\textbf{Curso}}                & \multicolumn{1}{l|}{Ciência da Computação/INE/UFSC} \\ \hline
\multicolumn{1}{|l|}{\textbf{Área de Concentração}} & \multicolumn{1}{l|}{Inteligência Artificial}                    \\ \hline
\end{tabular}%
\end{table}
\noindent
\textbf{Instruções para preenchimento pelo \underline{ORIENTADOR DO TRABALHO}:}

\noindent - Para  cada  critério avaliado,  assinale  um  X  na  coluna  SIM  apenas  se  considerado  aprovado.

\noindent Caso contrário, indique as alterações necessárias na coluna Observação.
\begin{table}[H]
\resizebox{\textwidth}{!}{%
\begin{tabular}{|l|
>{\columncolor[HTML]{C0C0C0}}l |
>{\columncolor[HTML]{C0C0C0}}l |
>{\columncolor[HTML]{C0C0C0}}l |
>{\columncolor[HTML]{C0C0C0}}l |l|}
\hline
\multicolumn{1}{|c|}{\cellcolor[HTML]{C0C0C0}} &
  \multicolumn{4}{c|}{\cellcolor[HTML]{C0C0C0}\textbf{Aprovado}} &
  \cellcolor[HTML]{C0C0C0} \\ \cline{2-5}
\multicolumn{1}{|c|}{\multirow{\cellcolor[HTML]{C0C0C0}\textbf{Critérios}}} &
  \multicolumn{1}{c|}{\cellcolor[HTML]{C0C0C0}\textbf{Sim}} &
  \multicolumn{1}{c|}{\cellcolor[HTML]{C0C0C0}\textbf{Parcial}} &
  \multicolumn{1}{c|}{\cellcolor[HTML]{C0C0C0}\textbf{Não}} &
  \multicolumn{1}{c|}{\cellcolor[HTML]{C0C0C0}\textbf{\begin{tabular}[c]{@{}c@{}}Não \\ se aplica\end{tabular}}} &
  \multirow{\cellcolor[HTML]{C0C0C0}\textbf{Observação}} \\ \hline
\begin{tabular}[c]{@{}l@{}}1. O trabalho é adequado para um TCC no \\ CCO/SIN (relevância / abrangência)?\end{tabular} &
   &
   &
   &
   &
   \\ \hline
2. O titulo do trabalho é adequado? &
   &
   &
   &
   &
   \\ \hline
3. O tema de pesquisa está claramente descrito? &
   &
   &
   &
   &
   \\ \hline
\begin{tabular}[c]{@{}l@{}}4. O problema/hipóteses de pesquisa do\\ trabalho está claramente identificado?\end{tabular} &
   &
   &
   &
   &
   \\ \hline
5. A relevância da pesquisa é justificada? &
   &
   &
   &
   &
   \\ \hline
\begin{tabular}[c]{@{}l@{}}6. Os objetivos descrevem completa e\\ claramente o que se pretende alcançar neste trabalho?\end{tabular} &
   &
   &
   &
   &
   \\ \hline
\begin{tabular}[c]{@{}l@{}}7. É definido o método a ser adotado no\\ trabalho? O método condiz com os objetivos e \\ é adequado para um TCC?\end{tabular} &
   &
   &
   &
   &
   \\ \hline
\begin{tabular}[c]{@{}l@{}}8. Foi definido um cronograma coerente com \\ o método definido (indicando todas as\\ atividades) e com as datas das entregas\\ (p.ex. Projeto I, II, Defesa)?\end{tabular} &
   &
   &
   &
   &
   \\ \hline
\begin{tabular}[c]{@{}l@{}}9. Foram identificados custos relativos \\ à execução deste trabalho (se houver)?\\ Haverá financiamento para estes custos?\end{tabular} &
   &
   &
   &
   &
   \\ \hline
\begin{tabular}[c]{@{}l@{}}10. Foram identificados todos os envolvidos\\ neste trabalho?\end{tabular} &
   &
   &
   &
   &
   \\ \hline
\begin{tabular}[c]{@{}l@{}}11. As formas de comunicação foram\\ definidas (ex: horários para orientação)?\end{tabular} &
   &
   &
   &
   &
   \\ \hline
\begin{tabular}[c]{@{}l@{}}12. Riscos potenciais que podem causar\\ desvios do plano foram identificados?\end{tabular} &
   &
   &
   &
   &
   \\ \hline
\begin{tabular}[c]{@{}l@{}}13.  Caso o TCC envolva a produção de um \\ software ou outro tipo de produto e seja \\ desenvolvido também como uma atividade \\ realizada numa empresa ou laboratório, \\ consta da proposta uma declaração (Anexo 3)\\ de ciência e concordância com a entrega do\\ código fonte e/ou documentação produzidos?\end{tabular} &
   &
   &
   &
   &
   \\ \hline
\end{tabular}%
}
\end{table}

\begin{table}[H]
\centering
\begin{tabular}{|l|lll|}
\hline
\textbf{Avaliação}          & \textbf{[ ] Aprovado}      & \textbf{}                & \textbf{[ ] Não Aprovado} \\ \hline
 & \textit{Nome} & \textit{Data} & \textit{Assinatura}                                                                                     \\ \hline
\textbf{Professor Responsável} & \multicolumn{1}{l|}{Elder Rizzon Santos} & \multicolumn{1}{l|}{ } &  \\ \hline
\textbf{Orientador externo} & \multicolumn{1}{l|}{Thiago Ângelo Gelaim} & \multicolumn{1}{l|}{} &                      \\ \hline
\end{tabular}
\end{table}
\pagenumbering{arabic} 
\setcounter{page}{1}
% % ---------------------------------------------------------------
% ----------------  Ficha Catalográfica  -------------------------
% ---------------------------------------------------------------
% Modelo de ficha catalográfica. Você deverá substituir esta
% folha na versão final da monografia por um pdf fornecido pela 
% biblioteca. Salve o modelo oficial como ficha_catalografica.pdf
% e use o comando abaixo para inseri-lo na versão final do texto.

%\begin{fichacatalografica}
%    \includepdf{ficha_catalografica.pdf}
%\end{fichacatalografica}



%% Modelo de Como fazer a Ficha Catalográfica:

\begin{fichacatalografica}
	\sffamily
	\vspace*{\fill}					% Posição vertical
	\begin{center}					% Minipage Centralizado
	\fbox{\begin{minipage}[c][8cm]{14cm}		% Largura
	\small
	\imprimirautorcite.
	%Sobrenome, Nome do autor
	
	\hspace{0.5cm}  \\
	\imprimirtitulo  / \imprimirautor. --, \imprimirano-
	
	\hspace{0.5cm} \pageref{LastPage} p. 1 :il. (colors; grafs; tabs).\\
	
	\hspace{0.5cm} \imprimirorientadorRotulo~\imprimirorientador\\
	
	\hspace{0.5cm}
	\parbox[t]{\textwidth}{\imprimirtipotrabalho~--~\imprimirinstituicao, ~ \\
	\imprimirinstituto, ~\imprimirdepartamento,~\imprimirano.}\\
	
	\hspace{0.5cm} % Palavras-chave do trabalho
		1. Palavra-chave 1.
		2. Palavra-chave 2.
		2. Palavra-chave 3.
		3. Palavra-chave 4.
		4. Palavra-chave 5.
		I. \imprimirorientador.
		II. \imprimirinstituicao.
		III. \imprimirtitulo  			
	\end{minipage} }
	\end{center}
\end{fichacatalografica}



% \begin{errata}

\noindent AUTOR, \textbf{\imprimirtitulo} \imprimirsubtitulo. nº de páginas. \imprimirtipotrabalho - 
\imprimirdepartamento, \imprimirinstituicao, \imprimirlocal, \imprimirano.

\begin{table}[!ht]
\centering
\begin{tabular}{|c|c|c|c|} \hline
\textbf{Página} & \textbf{Linha} & \textbf{Onde se lê} & \textbf{Leia-se} \\ \hline
16 & 10 & &  \\ \hline
\end{tabular}
\end{table}


Este elemento pré-textual é opcional e deve ser inserido no texto, geralmente após o trabalho já ter sido impresso, após a folha de rosto. Deve conter a referência do trabalho e o texto da errata com a indicação da página a linha \cite{NBR14724:2011}.

\end{errata}
% % ---------------------------------------------------------------
% ----------------  Folha de aprovação  -------------------------
% ---------------------------------------------------------------
% Modelo de Folha de aprovação. Você deverá substituir esta folha na versão final da monografia por um pdf fornecido pelo colegiado do seu curso. Salve o modelo oficial como 
% folhadeaprovacao_final.pdf e use o comando abaixo para inseri-lo na versão final do texto. 
% A versão abaixo foi feita seguindo as normas ABNT NBR 14724:2011 em vigor.

%\begin{fichacatalografica}
%    \includepdf{folhadeaprovacao_final.pdf}
%\end{fichacatalografica} Esta folha será 


\begin{folhadeaprovacao}


\begin{center}
     {\large \imprimirautor}\\
       	\vspace{2cm}	
    \begin{DoubleSpace}    
    {\large \textbf{\MakeUppercase{\imprimirtitulo}}} \\
    {\large \textbf{\MakeUppercase{\imprimirsubtitulo}}}
    \end{DoubleSpace}
		\vspace{1cm}
        
\end{center}		


\begin{flushright} 
    \parbox{0.6\linewidth}{
		\imprimirtipotrabalho~ apresentada ao Curso de \imprimircurso~ da \imprimirinstituicao~ como parte dos
		requisitos necessários para a obtenção do grau em \imprimirgrau. \\}
   \end{flushright} 

\noindent Aprovada em \imprimirlocal,~ \imprimirdata. 
\begin{center}
\vfill
       \rule{10cm}{.1pt} \\
       {\imprimirorientador} \\ {\imprimirttorientador} \\
			 Orientador 
       \vfill
			 \ifdefvoid{\imprimircoorientador}{}{
       \rule{10cm}{.1pt} \\
       \imprimircoorientador \\ \imprimirttcoorientador \\ Coorientador }
			 \vfill
       \rule{10cm}{.1pt} \\
       {\imprimirexaminadorum} \\ {\imprimirttexaminadorum} \\ Examinador
        \vfill
        \ifdefvoid{\imprimirexaminadordois}{}{
        \rule{10cm}{.1pt} \\
        \imprimirexaminadordois \\ \imprimirttexaminadordois \\ Examinador}
				\vfill
        \ifdefvoid{\imprimirexaminadortres}{}{
        \rule{10cm}{.1pt} \\
        \imprimirexaminadortres \\ \imprimirttexaminadortres \\ Examinador}
				\vfill
        \ifdefvoid{\imprimirexaminadorquatro}{}{
        \rule{10cm}{.1pt} \\
        \imprimirexaminadorquatro \\ \imprimirttexaminadorquatro \\ Examinador}
\end{center}
  
\end{folhadeaprovacao}
% --- 
% \begin{dedicatoria}
   \vspace*{\fill}
   \centering
   \noindent
   \textit{ Espaço para prestar uma homenagem ou dedicar o trabalho a alguém.} 
	 \vspace*{\fill}
\end{dedicatoria}
% \begin{agradecimentos}

Espaço para agradecer às pessoas e/ou instituições que contribuíram de forma relevante à elaboração do trabalho.

\end{agradecimentos}

% \begin{epigrafe}
    \vspace*{\fill}
    
	\begin{flushright}
   Alguma citação importante para a construção do trabalho. A fonte deve ser indicada e também deve constar na lista de referências bibliográficas.
	\end{flushright}
    
\end{epigrafe}
%--------------------------------------------------------------------------
%--------------------- Resumo em Português --------------------------------
%--------------------------------------------------------------------------

\setlength{\absparsep}{18pt} % ajusta o espaçamento dos parágrafos do resumo
\begin{resumo}
Percepções são a forma mais simples de uma entidade se comunicar com o ambiente. Cada pessoa possui uma maneira diferente de perceber e interpretar o mundo. Entretanto, sabe-se que na percepção humana existem ilusões e alucinações, sendo que a primeira são percepções de objetos presentes no mundo mas com características inadequadas ou características corretas em objetos inadequados, e a segunda são percepções falsas de coisas reais. Dito isso, como podemos saber se nossas percepções são reais ou se são apenas fruto de nossa imaginação? E a questão derivada disso é: e computadores? Agentes possuem diversos sensores para reconhecerem o mundo a sua volta, e esses sensores podem falhar. Nesse trabalho, apresentamos um modelo genérico de revisão de percepções, capaz de tratar de percepções inválidas recebidas pelo agente, e criar novos planos para se adaptar ao ambiente.

 \vspace{\onelineskip}
 \noindent
 \textbf{Palavras-chave}: Agentes. Percepção. Ilusão. Alucinação.

\end{resumo}

%--------------------------------------------------------------------------
%--------------------- Resumo em Inglês --------------------------------
%--------------------------------------------------------------------------
\iffalse
\begin{resumo}[Abstract]
 \begin{otherlanguage*}{english}
   This is the english abstract.


   \vspace{\onelineskip}
   \noindent 
   \textbf{Keywords}: Keywords1, Keywords2, Keywords3.
 \end{otherlanguage*}
\end{resumo}
\fi % (Abstract no mesmo arquivo)

% As listas abaixo são opcionais. Caso o trabalho não possua alguma(s) dela(s) basta comentar os seus respectivos comandos.

% Lista de Figuras. 
\pdfbookmark[0]{\listfigurename}{lof}
\listoffigures*   
\cleardoublepage
% lista de Tabelas
\pdfbookmark[0]{\listtablename}{lot}
\listoftables*  
\cleardoublepage
% Lista de Algoritmos
% \pdfbookmark[0]{\listalgorithmcfname}{lof}
% \listofalgorithmes
%\cleardoublepage

% Lista de Siglas e Símbolos. Estas listas são criadas manualmente e seus arquivos estão na pasta PreTextuais.
% ---------------------------------------------------
% ------ Lista de abreviaturas e siglas -------------
% ---------------------------------------------------
\begin{siglas}
  \item[ BDI ] \emph{Belief-Desire-Intention}
  \item[ IA ] Inteligência Artificial
  \item[ NPC ] Número de Percepções recebidas por Ciclo 
  \item[ PPI ] Porcentagem de Percepções Inválidas
  \item[ SMC ] Sistema multicontexto
  \item[ TMA ] Tempo Médio gasto pelo Autoplanejamento
  \item[ TMC ] Tempo Médo gasto em um Ciclo de raciocínio
\end{siglas}
% ---------------------------------------------------
% ----------- Lista de símbolos ---------------------
% ---------------------------------------------------

\begin{simbolos}
  \item[$ \Delta $] Função de transição do modelo de revisão de percepções
  \item[$ \gamma $] Função de percepção do agente
  \item[$ \theta $] Função de refinamento
  \item[$ \rho $] Conjunto de percepções refinadas
  \item[$ A $] Conjunto de anomalias
  \item[$ Ab $] Conjunto de blocos avaliadores
  \item[$ Ag$ ] Agente
  \item[$ Ap $] Conjunto de blocos de planejamento automatizado
  \item[$ A_{pr} $] Conjunto de anomalias processadas no ciclo de raciocínio
  \item[$ c $] Contexto do agente
  \item[$ Cf $] Função de limpeza
  \item[$ D $] Conjunto de decisores
  \item[$ K $] Conjunto de conhecimentos do agente
  \item[$ L $] Lista ordenada
  \item[$ M_{ih} $] Módulo de ilusão e alucinação
  \item[$ P $] Conjunto de planos do agente
  \item[$ p $] Conjunto de percepções iniciais
  \item[$ Pf $] Função de processamento
  \item[$ P(L_i) $] Função peso da lista ponderada
  \item[$ T_{m}(x)$] Função tempo médio de x
  \item[$ V $] Conjunto de percepções válidas
\end{simbolos}



% Sumário:
\pdfbookmark[0]{\contentsname}{toc}
\tableofcontents*
\cleardoublepage

%% ------------- Capítulos ----------------------%%

% \setcounter{page}{1}
\textual 
\section{Introdução}
\subsection{Motivação}
	Internet das Coisas (ou IoT, sigla em inglês para Internet of Things) é não só um tópico que nos últimos anos vem ganho destaque na mídia como tem também ganho os holofotes dos especialistas em tecnologia. IoT é um conceito ligado ao crescimento da interconexão digital de objetos cotidianos com a internet, como eletrodomésticos, microcontroladores e variedades ainda mais simples, como lâmpadas e tomadas. Em resumo, é a conexão dos objetos, mais do que das pessoas, a internet. A interconexão dos sistemas permite um ambiente diário totalmente inteligente e eficiente. Portanto, um dos maiores impactos da IoT é, a partir da interpretação de dados recebidos pelos sensores dos dispositivos, a capacidade dos objetos de se comunicar com os usuários, promovendo melhorias na qualidade de vida, maior produtividade e agilidade nos processos. Além do impacto imediato na mudança do estilo de vida dos usuários, essa nova onda de produtividade permitirá que as pessoas foquem naquilo que não pode ser automatizado.
	Essa tecnologia tem ganhado espaço no vida das pessoas \cite{fan2018blockchain}. No entanto, essas tecnologias também representam graves riscos de privacidade e preocupações com a segurança dos dados \cite{dwivedi2019decentralized}. Suponha uma loja em que todos os sistemas estejam conectados, desde os as lâmpadas e a rede elétrica até os dados de compra e vendas do lugar. Sendo assim, se um hacker encontrar uma vulnerabilidade em algum desses sistemas e conseguir invadir a tomada, por exemplo, o mesmo poderá usá-la de porta de entrada para acessar outros sistemas dentro da loja. Portanto, garantir a segurança desse tipo de rede é algo essencial.


\subsection{Justificativas}
	A crescente no uso de dispositivos conectados a rede faz com que os fabricantes de IoT se concentraram na funcionalidade e nos recursos dos mesmos e que a segurança se tonasse algo secundário \cite{falco2019neuromesh}. Como dispositivos desse tipo geralmente possuem um processamento reduzido para um baixo consumo de energia, assim como possuem uma memória pequena para manter o custo barato, a tarefa de torná-los seguros é bastante árdua, e é um desafio ainda em aberto para a academia. Nos últimos anos, abordagens que utilizam blockchain tem se tornado uma boa alternativa para resolver esse problema, e diversas pesquisas tem surgido na área, mas apesar de fornecerem segurança e privacidade descentralizadas, envolvem energia significativa, atraso e sobrecarga computacional que é não é adequado para a maioria dos dispositivos IoT com recursos limitados \cite{dorri2017blockchain}. Então, apesar de blockchain ser uma alternativa bastante interessante para ser tomada, é preciso reduzir o custo computacional e elétrico envolvido para ter um custo benefício que valha a pena. A urgência da necessidade de modelos seguros de IoT pode ser sentida quando analisarmos os grandes players que estão no mercado de dispositivos para casa, como Google, Amazon e Apple, que já possuem milhares de casas com aparelhos inteligentes, como lâmpadas, assistentes pessoais, janelas automatizadas, robôs para limpeza do chão e diversos outros dispositivos que precisam ser seguros, caso contrário ataques de hackers podem deixar em perigo toda a infraestrutura das casas das pessoas que os utilizam. Proteger a rede IoT é um dos objetivos importantes de projetar novas arquiteturas distribuídas \cite{sharma2018software}.

\subsection{Objetivos}
\subsubsection{Gerais}
	O aumento acentuado dos aplicativos da Internet das Coisas (IoT), requer soluções robustas para o problema da violabilidade de seus dados. Por tanto, é preciso utilizar métodos computacionais que permitem tal robustês, como blockchain, para garantir a segurança não só da vida online de seus proprietários como até a segurança física que pode ser comprometida dependendo do tipo de aparelhos instalados em sua casa \cite{dorri2017towards}. Portanto o principal objetivo é garantir a segurança dos dispositivos IoT utilizando blockchain.

\subsubsection{Específicos}
\begin{itemize}
\item Compreender Internet das Coisas e blockchain;
\item Obter uma compreensão abrangente sobre como blockchain pode ser usado para garantir a segurança em IoT;
\item Fornecer uma visão geral sobre segurança em IoT em geral;
\item Identificar abordagens existentes, seus principais casos de uso, identificação dos principais problemas, e encaminhar possíveis soluções.
\end{itemize}

\subsection{Organização do Artigo}
	A seção 2 apresenta os conceitos básicos de internet das coisas, segurança e blockchain. Na seção 3 são apresentados a revisão bibliográfica sistemática, com a tabela de buscas, e os trabalhos correlatos, exemplificando a situação do estado da arte no uso de blockchain como solução para segurança em redes de aparelhos de internet das coisas. Na seção 4 são expostos os aspectos relevantes que remetem tanto ao problema quanto a solução, contendo uma discussão sobre as blockchains que já operam hoje em dia. Na seção 5 discutimos os problemas existentes no uso de blockchain para soluções de segurança em IoT, explorando aspectos de redes blockchain e sua aplicação no mercado. Na seção 6 apresentamos possíveis soluções, utilizando como base um caso de estudo de uma casa inteligente que utiliza redes blockchain para manter a segurança do sistema. Na seção 7, abstraímos o exemplo da seção 6 para um modelo teórico, que pode ser utilizado para modelar outras redes de internet das coisas, e mostramos resultados práticos e indicativos que colaboram para validar o modelo proposto. Por fim na seção 8 temos a conclusão e sugestões de trabalhos futuros a serem desenvolvidos dentro do tema do artigo.
	
\chapter{Planejamento}

\section{Escopo}

O trabalho consiste na análise do estado da arte na área de percepção em agentes inteligentes, com ênfase em soluções para a correção de percepções incompletas ou errôneas, a proposta de um modelo formal para o tratamento destas percepções inválidas, a implementação do modelo proposto livre de um domínio específico, a realização de simulações para testar a implementação e a análise dos dados obtidos.

Esse trabalho de conclusão de curso é uma lapidação do trabalho que foi desenvolvido pelo autor como bolsista PIBIC nos ciclos de 2018-2019 e 2019-2020.

\section{Método de Pesquisa}

A pesquisa será realizada através de uma revisão do estado da arte, a proposta de um modelo e a análise de tal modelo utilizando o \textit{factorial design} \cite{jain1990art}, mais especificamente o $2^k$ fatorial. Esse tipo de design consiste em variar $k$ fatores em 2 níveis diferentes, -1 e 1, que são extremos opostos. Por exemplo, em uma pesquisa ligada a um processador, um fator pode ser o número de núcleos, e seus níveis serem 1 núcleo e 8 núcleos. Portanto, o fator é uma variável livre, que é utilizada para analisar a variação de uma variável dependente qualquer. Para analisar os dados gerados, eles serão processados, apresentados em tabelas e dispostos em gráficos.

A pesquisa da parte teórica será feita através de livros e artigos das áreas abordadas (inteligência artificial, agentes, percepção e planejamento automatizado), enquanto a parte prática utilizará a linguagem de programação Python, tanto na implementação e simulação do modelo quanto no processamento dos dados.

\section{Custo}

Os custos não foram estimados, pois são constituídos apenas pelas horas trabalhadas dos professores envolvidos, uma vez que aquisições adicionais não são necessárias. O autor não recebe bolsa de pesquisa, portanto o projeto também não possui orçamento.

\section{Cronograma}

O gráfico de Gantt do cronograma é apresentado na tabela 2.1. Devido a pandemia do novo corona vírus, o cronograma pode ser alterado. As atividades iniciais são revisões, pois o texto já foi iniciado, conforme descrito no escopo. As atividades de desenvolvimento são verificações pois a implementação já foi realizada pelo autor, mas testes podem detectar erros que ainda não foram identificados.

% Please add the following required packages to your document preamble:
% \usepackage[table,xcdraw]{xcolor}
% If you use beamer only pass "xcolor=table" option, i.e. \documentclass[xcolor=table]{beamer}
\begin{table}[H]
\resizebox{\textwidth}{!}{\begin{tabular}{ccccccccccc}
\rowcolor[HTML]{CBCEFB} 
\cellcolor[HTML]{C0C0C0}\textbf{Atividade}                        & dez. & jan. & fev. & mar. & abr. & mai. & jun. & jul. & ago. & set. \\
Revisão da pesquisa do estado da arte                             & X    & X    &      &      &      &      &      &      &      &      \\
Revisão do modelo proposto                                        &      & X    & X    &      &      &      &      &      &      &      \\
Entrega parcial para TCC 1                                        &      &      & X    &      &      &      &      &      &      &      \\
Verificar corretude da implementação do modelo                    &      &      &      & X    & X    & X    &      &      &      &      \\
Verificar dados obtidos e executar nova simulação caso necessário &      &      &      & X    & X    & X    &      &      &      &      \\
Terminar rascunho do TCC                                          &      &      &      &      &      & X    & X    &      &      &      \\
Entregar rascunho do TCC                                          &      &      &      &      &      &      &      & X    &      &      \\
Preparação para a defesa pública                                  &      &      &      &      &      &      &      & X    &      &      \\
Defesa pública                                                    &      &      &      &      &      &      &      &      & X    &      \\
Ajustes no relatório final                                        &      &      &      &      &      &      &      &      & X    & X   
\end{tabular}}
\caption{Gráfico de Grantt.}
\end{table}

\section{Recursos Humanos}

Os recursos humanos do projeto e seus papéis estão descritos na tabela 2.2.

\begin{table}[H]
\centering
\resizebox{0.7\textwidth}{!}{\begin{tabular}{cc}
\rowcolor[HTML]{CBCEFB} 
\textbf{Nome}              & \textbf{Papel}              \\
Gustavo Emanuel Kundlatsch & Autor                       \\
Thiago Ângelo Gelaim       & Orientador                  \\
Elder Rizzon Santos        & Co-orientador e Responsável \\
Renato Cislaghi            & Professor das disciplinas de TCC\\
A definir                  & Membro da Banca             \\
A definir                  & Membro da Banca            
\end{tabular}}
\caption{Recursos humanos.}
\end{table}

\section{Comunicação}

Devido a pandemia, o projeto precisa ser desenvolvido de maneira completamente remota até o retorno das atividades presenciais da UFSC. Dessa forma, as reuniões de orientação precisam ser feitas por alguma ferramenta online (optamos pela ferramenta \textit{hangouts}). O fluxo de comunicação está descrito na tabela 2.3


\begin{table}[H]
\centering
\resizebox{\textwidth}{!}{\begin{tabular}{|c|c|c|c|c|}
\hline
\rowcolor[HTML]{CBCEFB} 
\textbf{O que}       & \textbf{Por quem} & \textbf{Para quem}                                                                                      & \textbf{Como}  & \textbf{Frequência} \\ \hline
Proposta de projeto  & Autor             & \begin{tabular}[c]{@{}c@{}}Orientador, Co-orientador e \\ Professor das disciplinas de TCC\end{tabular} & Sistema de TCC & Singular            \\ \hline
Andamento do projeto & Autor             & Orientador e Co-orientador                                                                              & Hangouts       & Quando necessário   \\ \hline
Relatório de TCC I   & Autor             & \begin{tabular}[c]{@{}c@{}}Orientador, Co-orientador e \\ Professor das disciplinas de TCC\end{tabular} & Sistema de TCC & Singular            \\ \hline
Relatório de TCC II  & Autor             & \begin{tabular}[c]{@{}c@{}}Orientador, Co-orientador e \\ Professor das disciplinas de TCC\end{tabular} & Sistema de TCC & Singular      \\ \hline
\end{tabular}}
\caption{Fluxo de comunicação.}
\label{tab:my-table}
\end{table}

\section{Riscos}

Os riscos mais prováveis e perigosos levantados foram apresentados na tabela 2.4. 
\begin{table}[H]
\centering
\resizebox{\textwidth}{!}{\begin{tabular}{|l|l|l|l|l|}
\hline
\rowcolor[HTML]{CBCEFB} 
\textbf{Risco}               & \textbf{Probabilidade} & \textbf{Impacto} & \textbf{Estratégia de Resposta}         & \textbf{Ações de Prevenção}    \\ \hline
Perda de Dados               & Baixa                  & Baixo            & Realizar simulações novamente           & Criar backups dos dados        \\ \hline
Resultados Insatisfatórios   & Baixa                  & Alto             & Adaptar modelo para suprir necessidades & Análise do estado da arte      \\ \hline
Dificuldade de Implementação & Médio                  & Médio            & Estudar a linguagem através de cursos   & Implementar provas de conceito \\ \hline
\end{tabular}}
\caption{Riscos ativos.}
\label{tab:my-table}
\end{table}
%\chapter{Fundamentação teórica}

\label{conceitos-fundamentais}

%\input{Content/trabalhos.tex}
Neste capítulo é apresentada a fundamentação teórica utilizada para o desenvolvimento do trabalho. Ele apresenta os conceitos básicos de inteligência artificial e um estudo aprofundado sobre agentes e percepções. Além disso, este capítulo apresenta definições que serão utilizadas na formalização do modelo no Capítulo \ref{chapter:model}.

\section{Inteligência artificial}

A Inteligência Artificial é um campo de estudo que se difere, pois enquanto outros campos do conhecimento se limitam a buscar entender \textit{como} o pensamento humano funciona, a IA se propõe a construir entidades pensantes \cite{russel2013artificial}. Apesar de ser uma área de pesquisa que surgiu na década de 50 \cite{moor2006dartmouth}, não existe uma única definição de Inteligência Artificial nem consenso dentro da comunidade acadêmica.

Uma visão prática do que é IA é sua abordagem como a automatização de diversas tarefas humanas: ``atividades que associamos ao pensamento humano, atividades como a tomada de decisões, a resolução de problemas, o aprendizado'' \cite{bellman1978introduction}. Essa definição é derivada da pergunta ``computadores podem pensar?''. Existem diversos exemplos de atividades do pensamento humano que os computadores são capazes de executar, como tratamento de incerteza, consciência e humor. Apesar disso, para Bellman, ``o espírito humano se mantém muito acima de qualquer coisa que possa ser automatizada''. Essa é uma visão de que o objetivo da IA é replicar o pensamento humano.

Um outro ponto de vista coloca a IA como ferramenta de investigação da mente humana, ou seja, como ``o estudo das faculdades mentais através do uso de modelos computacionais'' \cite{charniak1985introduction}. Para essa definição estar correta é necessário que exista uma equivalência entre o processo mental humano e o processamento de um computador. Por conta disso, Charniak define o dogma central da Inteligência Artificial: ``O que o cérebro faz pode ser pensado em algum nível como um tipo de computação''. Caso o dogma se mostre verdadeiro, o uso de modelos computacionais para o estudo das faculdades mentais é valido. ``Faculdades mentas'', dentro dessa definição, são os mecanismos internos que recebem imagens e palavras (através da visão e da linguagem) e os converte em saídas na forma de ações robóticas e fala. Esse processamento interno inclui dedução, planejamento, aprendizado e outras técnicas.

As definições de IA podem ser divididas entre aquelas que defendem que os computadores devem pensar como humanos, e aquelas que defendem que os computadores devem agir como humanos \cite{russel2013artificial}. As duas definições apresentadas anteriormente estão no primeiro grupo. As definições que levam em conta que os computadores devem agir como humanos em geral possuem um aspecto mais prático. Por exemplo, a definição dos autores Rich e Knight diz que inteligência artificial é ``o estudo de como os computadores podem desempenhar tarefas que hoje são melhor desempenhadas pelas pessoas'' \cite{rich1991artificial}. Com isso podemos notar que esse lado mais prático da IA não precisa se preocupar tanto com as questões filosóficas por trás do pensamento humano, pois foca em resolver problemas reais através dos métodos existentes. 

Neste trabalho, utilizaremos uma definição nessa mesma linha de pensamento de que o computador deve agir como um ser humano, nesse caso específico, de maneira lógica: ``inteligência computacional é o estudo do desenvolvimento de agentes inteligentes'' \cite{poole1998computational}. Essa definição foi escolhida pois o HAIL se trata de um modelo desenvolvido para agentes inteligentes. A Seção \ref{section:agent} se dedica a definir o que é um agente.

Independente da definição utilizada, podemos afirmar que a IA é um campo vasto que intriga muitos pesquisadores. Esse campo possui diversas técnicas, que são utilizadas para resolver todo o tipo de problemas. Kurzweil apresenta a seguinte visão sobre a pesquisa de Inteligência Artificial:

\begin{displayquote}
    É nosso destino como pesquisadores de Inteligência Artificial nunca alcançar a cenoura pendurada à nossa frente. A inteligência artificial é inerentemente definida como a busca de problemas difíceis da ciência dos computadores que ainda não foram resolvidos \cite{kurzweil2000age}.
\end{displayquote}



\section{Agente inteligente}

\label{section:agent}

Um agente inteligente é uma entidade autônoma, capaz de tomar as próprias decisões para atingir seus objetivos \cite{wooldridge1999intelligent}. Apesar da definição intuitiva ser simples, assim como no termo inteligência artificial não existe um consenso da comunidade sobre o que é um agente. Definições mais simples categorizam agente como algo que age, e agente inteligente como aquele que age buscando o melhor resultado possível \cite{russel2013artificial}, enquanto definições mais fechadas buscam contextualizar a entidade em um ambiente:

\begin{displayquote}
    Um agente inteligente é um sistema que age de maneira inteligente: o que faz é apropriado para as circunstâncias e seus objetivos, é flexível para mudar ambientes e mudar objetivos, aprende com a experiência e faz escolhas apropriadas, dadas as limitações perceptivas e a computação finita \cite{poole1998computational}.
\end{displayquote}

Tais definições são bastante amplas, e enquadram diversos tipos de programas computacionais. Apesar disso, agentes possuem características específicas, principalmente provenientes da capacidade de interagir uns com os outros. As capacidades e características dos agentes inteligentes podem ser divididas em cinco tópicos \cite{lugerBook6th}: 

\begin{enumerate}
    \item \textbf{Agentes são autônomos ou semi-autônomos:} Os agentes são independentes, ou seja, cada agente é capaz de trabalhar em uma tarefa sem saber no que outros agentes estão trabalhando, ou sem saber como eles resolvem determinada tarefa. Além disso, eles podem tanto fazer algo efetivamente (agir) ou reportar seus resultados para outros agentes (se comunicar).
    \item \textbf{Agentes possuem escopo localizado:} Cada agente é sensível ao ambiente, e normalmente não possui conhecimento sobre aquilo que todos os outros agentes estão realizando. Portanto o conhecimento de um agente é limitado às tarefas que ele deve realizar, sem conhecimento amplo sobre seus limites.
    \item \textbf{Agentes são interativos:} Normalmente, agentes se agrupam em forma de sociedade, com o objetivo de colaborar para resolver um problema. E assim como na sociedade humana, o conhecimento, a responsabilidade, habilidades e outros recursos estão distribuídos entre os indivíduos.
    \item \textbf{As sociedades dos agentes são estruturadas:} Na maioria das abordagens de solução de problema orientada a agentes, cada agente, mesmo possuindo seu próprio conjunto de habilidades e objetivos, se coordena com outros agentes para a resolução geral de problemas. Portanto, a solução final não é apenas coletiva, mas também cooperativa.
    \item \textbf{O fenômeno da inteligência nesses ambientes é emergente:} A capacidade final da resolução de um problema por uma sociedade de agentes é maior do que a soma das capacidades individuais de trabalho. A inteligência é vista como um fenômeno residente e emergente de uma sociedade e não apenas uma propriedade de um agente individual.
    
\end{enumerate}

A noção de agente inteligente ainda pode ser caracterizada como forte ou fraca \cite{wooldridge1995intelligent}. A noção fraca de agente é utilizada para denominar hardware ou software que possui algumas características específicas, sendo elas autonomia, habilidade social, reatividade e pró-atividade. Já a noção forte de agente se refere a um sistema que, além das características citadas anteriormente, ou foi concebida ou foi implementada utilizando conceitos que normalmente se aplicam a humanos. O modelo explorado neste trabalho segue a noção forte de agente, pois se baseia em conceitos da psicologia e da filosofia para resolver um problema prático de agentes.

Com base na leitura das definições apresentadas anteriormente, para este trabalho vamos definir agente formalmente conforme apresentado na Definição \ref{def:agent}, de maneira que facilite a manipulação e formalização do modelo proposto.

\theoremstyle{definition}
\begin{definition}
    \label{def:agent}
    Um agente é uma tripla $Ag = \langle K, P, \gamma \rangle$, onde:
    \begin{itemize}
        \item $K$ é uma base de conhecimentos, tal que $K = K_i \cup K_p$, onde $K_i$ é o conjunto de conhecimentos inicias do agente e $K_p$ os conhecimentos adquiridos através das percepções. $K_i$ é iniciado com valores arbitrários de acordo com a necessidade do agente e $K_p$ é iniciado vazio. Uma base de conhecimentos é uma estrutura que representa fatos a respeito do mundo e apresenta formas de raciocinar a respeito desses fatos para deduzir novos conhecimentos \cite{hayes1983building};
        \item $P$ é o conjunto de planos do agente, sendo um plano definido como $plano = (\Psi, A, \Omega)$, onde $\Psi$ é o conjunto união formado pelas pré-condições das ações que compõem o plano, $A$ o conjunto de ações que compõe o plano e $\Omega$ o conjunto união formado pelas pós-condições das ações que compõem o plano. Por sua vez, uma ação é definida como $acao = (\psi, n, \omega)$, sendo $\psi$ um conjunto de pré-condições, $n$ um nome para a ação e $\omega$ um conjunto de pós-condições; e
        \item $\gamma$ é a função de percepção, definida como $ \gamma(p, K) \rightarrow P_i $, onde $p$ é o conjunto de percepções recebidas, $K$ a base de conhecimentos de $Ag$ e $P_i$ o retorno da função, que é um subconjunto próprio do conjunto $P$ de planos do agente.
    \end{itemize}{}
\end{definition}{}

A partir dessa definição, podemos construir o conceito de contexto, que será amplamente utilizado na formalização do modelo de revisão de percepções. O contexto de um agente é o conjunto de todos os símbolos compreendidos pelo agente:

\begin{definition}
    O contexto $c$ de um agente $Ag$ é o domínio de sua função $\gamma$.
    \label{definition::context}
\end{definition}{}



\iffalse
\section{Agente}

Essa seção é um esqueleto de uma definição de agente que pode ser útil para explicar o modelo proposto. O foco é a relação da função $\Lambda$ com o resto do modelo, portanto K, A e P precisam ser melhorados.

Para esse artigo vamos definir agente como a quádrupla $Ag = (K, A, P, \Lambda)$, onde:

\begin{itemize}
    \item K é o conjunto de conhecimentos que o agente possui, podendo ter adquirido ou já ter começado com eles. Essas conhecimentos podem se referir a coisas que o agente conhece do mundo, coisas que o agente quer mudar no mundo ou coisas abstratas.
    \item A é o conjunto de atuadores do agente, ou seja, coisas que o agente pode usar para causar mudança no mundo (e, portanto, nos conhecimentos K).
    \item P é o conjunto de percebedores do agente, que adquirem novas informações a respeito do ambiente, causando mudança em K.
    \item $\Lambda$ é a função de percepção, que opera sobre a entrada de P, baseado nas informações contidas em K e retorna um subconjunto de A.
    \[ \Lambda: P \times K \rightarrow 2^A\]
\end{itemize}

Eu não sei se o jeito que eu escrevi está certo, mas usando essa ideia o que foi chamado de Contexto no resto do artigo pode ser facilmente definido como a imagem da função $\Lambda$.

\fi

\section{Percepção}

Existem diversas definições para o termo ``percepção''. Podemos entender percepção como um conjunto de sensações que, através da maneira subjetiva que um dado agente o interpreta, representa determinadas entidades do ambiente \cite{gibson1950perception}. Ou seja, a percepção não é simplesmente a representação direta das entidades reais que existem no mundo, mas um processo complexo que varia para cada indivíduo.

Percepções podem ser divididas nos níveis baixo e alto \cite{chalmers1992high}. A percepção de baixo nível ocorre através de meios físicos, os órgãos ópticos para os humanos ou os sensores para os agentes. 
A percepção de alto nível trabalha com uma visão mais geral da informação, extraindo conceitos dos dados brutos, podendo envolver diversas faculdades como o reconhecimento de objetos e o relacionamento de entidades. Nos trabalhos de Inteligência Artificial, em geral, estamos interessados na percepção de alto nível, pois a percepção de baixo nível está mais relacionada a robótica (no caso de agentes que possuem hardware próprio) ou a simulação (no caso de agentes que possuem apenas software).

Ainda segundo Chalmers et. al. uma das principais características da percepção de alto nível é a extrema flexibilidade. Um mesmo objeto do ambiente pode ser percebido de diversas maneiras, de acordo com as características do observador. Para os autores, algumas das fontes da flexibilidade das percepções são a capacidade de serem influenciadas pelas crenças, objetivos e contexto externo. Além disso, percepções de um mesmo objeto podem ser radicalmente alteradas conforme o necessário.

Para o modelo que iremos propor, baseado na Definição \ref{def:agent}, o conceito de percepção pode ser simplesmente definido como a entrada $p$ da função de percepção $\gamma$ de um agente. Vale destacar a diferença entre percepção e contexto, pois o contexto é constituído pelas percepções que fazem parte do domínio da função $\gamma$, ou seja, uma percepção é toda informação produzida pelo ambiente que o agente recebe, e contexto é o subconjunto das percepções que o agente reconhece.

Em ambientes dinâmicos há possivelmente centenas de percepções por segundo \cite{hayes1992guardian}. Mas percepções não necessariamente precisam incluir representações corretas da realidade e podem variar de agente para agente \cite{janssen2005agent}. Percepções incorretas podem ocorrer por conta de limitações da capacidade do agente de perceber determinados objetos ou por conta de obstrução física dos sensores, por exemplo \cite{chrisman1991intelligent}.

\subsection{Refinamento}

Como o volume de percepções de um agente pode ser muito grande, e as percepções custam tempo para serem processadas, o número de percepções que chegam ao ciclo de raciocínio do agente pode ser reduzido para diminuir seu custo computacional. Neste trabalho, esse processo será chamado de refinamento, definido da seguinte maneira:

\begin{definition}
    \label{def:refinamento}
    Refinamento de percepções é uma função $\theta$ tal que, dado o conjunto de entradas de percepções $p$, reduz tais percepções para um subconjunto próprio $\rho$.
\end{definition}

Existem diversas maneiras de realizar refinamento. Uma das mais clássicas é o uso de filtros de percepções, que limitam que percepções serão processados pelo agente baseado em diversos critérios, como posição, distância e a velocidade do objeto percebido \cite{bordeux2001}. Outra maneira de restringir as percepções é a percepção ativa. Um observador pode ser categorizado como ativo se ele ativamente pode executar uma ação que altere as configurações geométricas de seus sensores, com objetivo de melhorar a qualidade de sua observação \cite{Aloimonos1988}. O processo realizado pela percepção ativa pode ser definido através da tupla: por quê, o que, quando, onde e como perceber  \cite{Bajcsy2018}. Desta maneira, o agente pode perceber apenas quando é necessário (por quê), escolhendo o que perceber (o que) e otimizar fisicamente a percepção (quando, onde e como).

\subsection{Anomalias}

Anomalias são as percepções consideradas inválidas, geradas por alguma falha no processo de percepção. O conjunto de anomalias de um agente é o conjunto de todas as percepções possíveis que não fazem parte de seu contexto. A seguinte definição será utilizada:

\begin{definition}
    Uma percepção $p$ de um agente $Ag$ com contexto $c$, é uma anomalia caso $p \notin c$ de $Ag$.
\end{definition}

Anomalias podem ser divididas em dois tipos: alucinações e ilusões. Nas seções seguintes, será descrito o que as alucinações e ilusões são para os estudos clássicos do problema da percepção \cite{Russell1912-RUSTPO-49} \cite{Price1933-PRIP-20} e como elas podem ser representadas dentro do processo de percepção de agentes inteligentes. A base desse estudo foi o artigo ``The Problem of Perception'' \cite{perception-problem}.

\subsubsection{Ilusão}

Uma ilusão, na definição clássica onde o objeto de estudo são os seres humanos, é qualquer situação perceptiva na qual um objeto físico é realmente percebido, mas ele aparenta ser outra coisa que ele não é \cite{Smith2002-SMITPO-17}. O Exemplo \ref{example::ilusao1} ilustra o que é uma ilusão.

%Essa definição será usada como base para criar nossa definição computacional para ilusão. Entretanto, é importante notar que essa é uma definição diferente de outras encontradas na literatura, como a usada por Lhelani et. al. \cite{Khemlani2017}, onde uma ilusão é uma falácia advinda de uma interpretação humana errada sobre um problema lógico resolvido pelo computador.

\begin{example}{}
Suponha um robô que possui a tarefa de empacotar diferentes itens do depósito de uma loja. Os itens passam por uma esteira, e o agente usa dois sensores para perceber o que os itens são. O primeiro sensor é uma câmera ligada ao topo do corpo do robô, usado para definir a forma do item. O segundo sensor é um detector de textura na lateral das esteiras. Assim, as informações advindas dos sensores formam predicados da forma \texttt{forma(textura)}, que descreve um item. Os itens podem ter a forma de um círculo, um quadrado ou um triângulo, e sua textura pode ser lisa ou listrada.
O agente possui duas cores de papéis para empacotar, vermelho e azul. O papel vermelho é usado para quadrados (de ambas as texturas) e círculos lisos. O papel azul é usado para círculos listrados e triângulos lisos. A loja não vende triângulos listrados, então não há nenhum item desse tipo no depósito. Podemos descrever esse comportamento com as seguintes regras:

\begin{center}
    \texttt{papel(vermelho) :- quadrado(\_) OR circulo(liso)}
\end{center}
\begin{center}
     \texttt{papel(azul) :- circulo(listrado) OR triangulo(liso)}
\end{center}

\begin{figure}[h!]
    \centering
    \caption{Funcionamento do agente empacotador.}
    \includegraphics[width=0.6\textwidth]{images/empacotador_novo.png}
    \legend{Fonte: Autor.}
    \label{fig:method}
\end{figure}
\label{example::robo}
\end{example}

\begin{example}
Agora vamos estender o exemplo \ref{example::robo} supondo que o sensor tátil não está funcionando corretamente, e ele irá sentir objetos lisos como se fossem ondulados. Para nosso exemplo, vamos considerar que o primeiro item será um triângulo liso. Nenhum erro ocorre quando a câmera percebe o triângulo, mas o sensor tátil indica que ele é ondulado. Nesse caso, teremos uma ilusão, pois triângulo é um objeto válido, mas ele foi percebido com uma propriedade inválida, que não existe no contexto do agente. Não há planos para quando o agente detecta esse tipo de erro, então ele pode executar um plano padrão para casos de erro, ou simplesmente não fazer nada.
\label{example::ilusao1}
\end{example}{}

Esse tipo de anomalia demonstrado no Exemplo \ref{example::ilusao1} será chamado de ilusão classe 1, onde o corpo do predicado, ou o objeto da percepção, é válido, mas possui um argumento ou uma característica inválida.

\begin{definition}{}
   Uma ilusão classe 1 é uma percepção do tipo \texttt{objeto(caracteristica)} ou equivalente, onde \texttt{objeto} é um elemento do contexto do agente e \texttt{característica} não é.
\end{definition}

\begin{example}
    Se considerarmos que as percepções do agente possuem formas erradas por conta de um defeito na câmera ou o software de reconhecimento de padrões que atua sobre ela, um objeto como um círculo liso pode ser reconhecido como um estrela listrada. Estrela não é um objeto válido, mas listrado é.
    \label{example::ilusao2}
\end{example}{}

Isso será chamado de ilusão classe 2, definido de maneira similar a ilusão classe 1.

\begin{definition}{}
   Uma ilusão classe 2 é uma percepção do tipo \texttt{objeto(caracteristica)} ou equivalente, onde \texttt{objeto} não é um elemento do contexto do agente e \texttt{característica} é.
\end{definition}

Portanto, podemos simplesmente definir ilusão da seguinte forma:

\begin{definition}{}
   Uma ilusão é uma percepção do tipo \texttt{objeto(caracteristica)} ou equivalente, que se caracteriza como uma ilusão classe 1 ou uma ilusão classe 2.
\end{definition}

\subsubsection{Alucinação}

O segundo tipo de anomalia é a alucinação. Uma alucinação é uma percepção recebida que poderia ser completamente válida, mas que na realidade não existe no ambiente.

\begin{example} {}
    Retornando ao exemplo \ref{example::ilusao1} do agente responsável por empacotar os itens, mas que agora apresenta também o comportamento defeituoso do exemplo \ref{example::ilusao2}. Uma percepção formalmente correta, do tipo \texttt{objeto(caracteristica)}, por conta dos erros que os sensores possuem, pode resultar na percepção \texttt{estrela(ondulada)}. Essa percepção poderia ser processada pelo agente, entretanto ela não faz parte de seu contexto e resultaria na execução de um plano padrão para erros ou na inação do agente.
    \label{exemple::alucinacao}
\end{example}

Assim, vamos definir alucinação como um tipo específico de ilusão classe 1 e classe 2, podendo acarretar os mais diversos tipos de erros dentro do raciocínio do agente, ou gerando problemas caso seja ignorada.

\begin{definition}{}
   Uma alucinação é uma percepção do tipo \texttt{objeto(caracteristica)} ou equivalente, onde nem \texttt{objeto} nem \texttt{característica} são elementos do contexto do agente.
\end{definition}

\section{Planejamento automatizado}

Planejamento automatizado é um dos problemas fundamentais da Inteligência Artificial. As motivações para usar o planejamento automatizado são a capacidade de utilizar recursos de planejamento acessíveis e eficientes e reproduzir uma parte do processo cognitivo humano com um componente totalmente integrado de comportamento deliberativo \cite{GHALLAB20041}. A maneira clássica de realizar planejamento automatizado é considerar esse um problema de dedução lógica, onde dado um estado inicial, ações que aferam esse estado e um conjunto de estados de objetivo, é necessário encontrar a sequência de ações que faziam com que o ambiente saísse de um estado inicial para um estado de objetivo \cite{MADANI20035}.

Uma forma alternativa de tratar o problema de planejamento automatizado é utilizando planejamento probabilístico \cite{KUSHMERICK1995239}. Essa abordagem pode ser necessária por conta do fato de que o agente provavelmente não tem conhecimento completo do mundo ao seu redor. Outra saída para o problema do planejamento automatizado são os processos de decisão de Markov \cite{Cassandra:1998:EAA:926710}, \cite{DBLP:journals/corr/abs-1105-5460}, \cite{article}.

Na Definição \ref{definition::autoplanning} é apresentada a noção abstrata de planejamento automático, descrita como um modelo conceitual simples que contém os elementos principais do problema, tendo sido originalmente apresentada por Ghallab et. al. \cite{GHALLAB20041}.

\begin{definition}{}
\label{definition::autoplanning}
   % USAR MAIS TARDE PARA DEFINIR O BLOCO DE AUTOPLANEJAMENTO An automated planning block is a instance of the conceptual model of automated planning, described through the interaction between three components bellow \cite{GHALLAB20041}:
   Um modelo conceitual de planejamento automatizado é descrito como a interação entre os seguintes três componentes:
   
    \begin{itemize}
        \item Um sistema de transição de estados $\Sigma$, especificado por uma função de transição de estados $\Gamma$, de acordo com os eventos e ações que ele recebe. 
        \item Um $controlador$, que dado uma entrada de estados $s$ do sistema, fornece como saída uma ação de acordo com algum plano.
        \item Um $planejador$, que dado uma entrada de uma descrição de sistema $Z$, uma situação inicial e alguns objetivos, sintetiza um plano para o controlador a fim de alcançar o objetivo.
    \end{itemize}
    
    Um sistema de transição de estados $\Sigma$ é uma quádrupla $\Sigma = \langle S, A, E, \Gamma \rangle$, onde:
    
    \begin{itemize}
        \item $S = \{s_1, s_2, ..., s_{n}\}$ é um conjunto finito ou recursivamente enumerável de estados;
        \item $A = \{a_1, a_2, ..., a_{n}\}$ é um conjunto finito ou recursivamente enumerável de ações;
        \item $E = \{e_1, e_2, ..., e_{n}\}$ é um conjunto finito ou recursivamente enumerável de eventos; e 
        \item $\Gamma: S \times A \times E \rightarrow 2^S$ é uma função de transição de estados. 
    \end{itemize}
     
\end{definition}

\section{Resumo}

Existem diversas definições de agente, mas para este trabalho será considerado que um agente inteligente é uma entidade, inserida em um ambiente, que possui autonomia para tomar suas decisões. Essa entidade possui um conjunto de conhecimentos a respeito do ambiente, e pode atualizar esses conhecimentos através da percepção, o processo de utilizar seus sensores para reconhecer o mundo ao seu redor. O funcionamento básico de um agente acontece através da entrada de percepções novas, o processamento delas para atualizar a base de conhecimentos e a escolher de quais ações tomar. Como o agente pode receber muitas percepções de uma vez, pode ser necessário criar medidas para reduzir esse volume de acordo com certos parâmetros -- chamamos isso de refinamento.

Por fim, as percepções que um agente recebe podem ser anomalias, i. e., inválidas ou corrompidas. Separamos essas anomalias entre ilusões e alucinações, de acordo com suas características semânticas. Neste trabalho, será proposto um modelo que recebe as percepções, refina, detecta e classifica possíveis anomalias e cria novos planos para lidar com essas percepções inválidas através de um processo de planejamento automatizado.

Esse processo de receber percepções, categorizá-las como percepções válidas ou anomalias e gerar novos artefatos de valor para o agente é o que chamamos neste trabalho de revisão de percepções. No Capítulo \ref{chapter:relacionados} alguns trabalhos que propõem processos similares a revisão de percepções serão apresentados.
%% \chapter{Trabalhos Relacionados}
\chapter{Revisão Bibliográfica}

Existem diversas abordagens para otimizar as percepções recebidas por um agente. O primeiro e o segundos artigo apresentado nesse capítulo implementam modelos para tratar de percepções imprecisas. O terceiro e quarto artigo são trabalhos relacionados a outros campos de estudo de percepção, mas que precisam resolver problemas relacionados a percepções imperfeitas também.

\section{Scalable Perception for BDI-Agents Embodied in Virtual Environments \cite{van2011scalable}}

Ambientes virtuais como jogos, simulações e treinamentos exigem cada vez mais complexidade dos agentes com os quais os participantes interagem. A arquitetura BDI provê a complexidade necessária para que os agentes virtuais desempenhem as tarefas avançadas necessárias. Porém, agentes BDI tradicionais possuem uma interface direta com o ambiente, enquanto os agentes dos jogos normalmente possuem um conjunto de sensores limitados para realizar suas percepções. O problema é que agentes BDI não possuem um mecanismo padrão para controlar seus sensores, decidindo quais percepções receber. Dessa maneira, o agente pode facilmente ficar sobrecarregado de informação. Para resolver esse problema, os autores desse trabalho criaram um \textit{framework} que fornece habilidades sensoriais e e atenção perceptiva para agentes BDI incorporados em um ambiente virtual, funcionando como um \textit{middleware} que atua entre o modelo cognitivo do agente e o ambiente.

Nesse \textit{framework} (representado na figura \ref{fig:bdiPerceptionModel}) toda informação do ambiente é representada pelo modelo de informação chamado \textit{Environment Object Model} (Modelo de Objeto de Ambiente) ou EOM. Objetos são definidos por classes e características, e existe uma hierarquia de objetos para agregar semântica. O \textit{middleware} é dividido entre a interface física e a interface cognitiva. A função da interface física é interagir com o ambiente para formar signos. Para tal, o processador sensorial primeiro recebe uma lista de possíveis percepções, que respeita filtros pré determinados implementados nos sensores. Após isso, esse processador determina se o agente está interessado nas informações recebidas, através do \textit{Interest Subscription Manager} (gerenciador de inscrição de interesse). Esses signos são passados para o agente na foma de percepções, que alteram os objetivos do agente. Os novos objetivos, por sua vez, são repassados para o sistema de atenção, que atualiza os interesses do agente no gerenciador de inscrição de interesse. A comunicação entre a interface física e a interface cognitiva é realizada através de mensagens, e há interfaces para que o \textit{middleware} possa se comunicar com o ambiente e com o agente, de maneira a mantê-lo independente de domínio. 

\begin{figure}[h!]
    \centering
    \includegraphics[width=\textwidth]{Images/bdiPerceptionModel.png}
    \caption{\textit{Framework} de percepção \cite{van2011scalable}.}
    \label{fig:bdiPerceptionModel}
\end{figure}

\section{PMK — A Knowledge Processing Framework forAutonomous Robotics Perception and Manipulation \cite{Diab_2019}}

As tarefas executadas por robôs vêm se tornando cada vez mais complexas. Para realizar essas tarefas, os robôs precisam passar por uma etapa de planejamento, na qual decidem quais ações tomar baseados no estado atual do ambiente ao seu redor. Alguns dos mecanismos clássicos de planejamento utilizam a Linguagem de Definição de Domínio de Planejamento (Planning Domain Definition Language ou PDDL) para descrever o ambiente no qual o agente está inserido. O problema é que essa abordagem assume um mundo fechado, i.e., que todos os fatos sobre o mundo são conhecidos, caso contrário o planejador pode falhar. Com essa limitação, um robô não é capaz de começar uma tarefa a não ser que todos os objetos do ambiente tenham sido reconhecidos e as ações que ele deve executar tenham sido definidas. Em outras palavras, a existência de alucinações e ilusões limita o funcionamento de tais sistemas.

Para resolver esse problema em situações onde o robô precisa realizar tarefas complexas de manipulação, foram criadas abordagens de planejamento baseadas no conhecimento, que utilizam reconhecimento semântico do cenário, conhecimento a respeito do comportamento físico de objetos e raciocínio sobre as possíveis ações de manipulação.

O trabalho de Diab et al. propõem um \textit{framework} de representação de conhecimento baseado em ontologias (uma especificação formal de conhecimento) chamado PMK (Perception and Manipulation Knowledge), apresentado na figura \ref{fig:pmk}. Esse modelo é genérico, para que possa ser utilizado em diversos domínios, e incorporado com outras ontologias. O PMK permite associar dados de percepção de baixo nível (proveniente dos sensores, na camada física do sistema) com conhecimento de alto nível (camada de raciocínio do agente).
Uma dos principais contribuições do artigo é criar um \textit{framework} que funcione como uma caixa preta para um planejador qualquer: o PMK é capaz raciocinar sobre os recursos do robô, suas restrições de ação, a viabilidade de ação e os comportamentos de manipulação. Para isso, o modelo utiliza análise situacional, avaliando a situação a situação dos objetos no ambiente com base em posicionamento espacial, acessibilidade do robô aos objetos, potencial área na qual o objeto será colocando entre outros. 

\begin{figure}
    \centering
    \includegraphics[width=0.9\textwidth]{Images/pmk-model.png}
    \caption{Framework PMK}
    \label{fig:pmk}
\end{figure}

\section{Understanding human intention by connecting perception and action learning in artificial agents \cite{kim2017understanding}}

Para desenvolver agentes capazes de realizar comportamentos complexos similares aos de seres humanos, é primeiro preciso entender como os seres humanos aprendem a perceber, pensar e agir em um mundo dinâmico. Diversos campos da inteligência artificial buscam replicar esses comportamentos, além de outros como a emoção e a cooperação. Essas habilidades parecem ser intrínsecas aos seres humanos, e tornam nossas relações mútuas únicas. Em particular, a capacidade de entender a intenção dos outros tem sido considerada a base da comunicação entre humanos. Nesse artigo, Kim, Yu e Lee propõem um modelo, chamado OA-SMTRNN, para entender a intenção do usuário e responder ativamente da maneira mais adequada, através do uso de redes neurais.. Para implementar o reconhecimento de intenção, são focados dois processos cognitivos, a percepção da disponibilidade de objetos e a previsão da ação humana.

Nos experimentos realizados pelos autores, diversos objetos precisaram ser percebidos pelo agente. Entretanto, alguns objetos poderiam estar sobrepostos, conforme demonstra a imagem \ref{fig:overlap}. Nesses casos, as percepções recebidas pelo agente poderiam estar incorretas. Para resolver este problema, o módulo responsável pelas ações foi implementado com a capacidade de relacionar a ação e os objetos. No artigo, é exemplificada a relação entre ``encher um copo d'água'' e ``fazer um café mocha''. Ou seja, o modelo, que foi previamente treinado, se mostrou capaz de associar a intenções como ``beber leite'' a determinadas ações (segurar um objeto, leva algo para a boca) para inferir que determinada anomalia (uma caixa de leite sobreposta por uma caneca) era uma caixa de leite.

\begin{figure}
    \centering
    \includegraphics[width=0.5\textwidth]{Images/overlap.jpg}
    \caption{Exemplo de sobreposição \cite{kim2017understanding}.}
    \label{fig:overlap}
\end{figure}

\section{Combining Perception and Knowledge Processing for Everyday Manipulation \cite{pangercic2010}}

Robôs autônomos implementados para realizar tarefas de manipulação de objetos do dia a dia precisam tomar diversas decisões que requerem a combinação de percepção e processamento de conhecimento. Este artigo de Panger et al. apresenta um sistema de programação lógica chamado K-CoPMan (Knowledge enabled Cognitive Perception for Manipulation, ou Percepção Cognitiva Ativada pelo Conhecimento para Manipulação). Esse modelo é capaz de testar e satisfazer pré-condições de conhecimento para manipulações do dia a dia. Para isso, ele fornece ao agente o conhecimento simbólico abstrato sobre as cenas percebidas, usa conhecimento simbólico abstrato para realizar tarefas de percepção e responde a novos tipos de consultas que exigem a combinação de percepção e processamento de conhecimento.

Um dos principais mecanismos do K-CoPMan é o componente de percepção passiva. Para se tornar consciente do ambiente, o agente que utiliza tal sistema pode escanear a cena em busca de áreas de interesse, como mesas ou cadeiras, utilizar os sensores para detectar objetos. Cada objeto recebe um identificador único, para então ser guardado na base de conhecimentos, juntamente com o contexto do momento em que a percepção foi realizada. O identificador é utilizado para que mais tarde seja possível examinar mais a fundo objeto, e possivelmente classificá-lo ou categorizá-lo. Portanto, o K-CoPMan permite que agentes inteligentes estejam conscientes do ambiente ao seu redor fazendo uma varredura completa do ambiente, uma vez que utiliza tanto percepção ativa quanto passiva, e guardando as anomalias detectadas para que possam ser tratadas mais tarde por um módulo próprio (o servidor de percepção).



\iffalse
\newpage
\section{SEÇÃO DE TRABALHOS RELACIONADOS (ORGANIZAÇÃO)}

Essa seção estão conteúdos ligados a pesquisa de trabalhos relacionados, mas será movida ou distribuída no artigo final.

\begin{itemize}
    \item O trabalho de John Anderson [9] propõem uma versão distribuída do simulador de agentes únicos Gensim. No artigo, Anderson discorre sobre como colocar a fonte das percepções completamente dentro do agente ou do ambiente é filosoficamente impreciso. Além disso, o autor descreve como isso também representa um problema prático, uma vez que a preparação sensorial é um elemento computacionalmente intensivo, portanto um equilíbrio deve ser encontrado.
    
    \item Para Włodzisław Duch [10], em seu trabalho sobre inteligência computacional, um dos grandes problemas atuais dessa área é seu foco em raciocínio como computação, e o uso da lógica como base do raciocínio, deixando de lado o caráter técnico de como símbolos precisam primeiro serem derivados de percepções reais. Segundo o autor, o cérebro humano é altamente especializado em análise de padrões naturais e outras técnicas que permitem o mapeamento de percepções a ações, mas apesar do grande avanço na área de inteligência computacional, sistemas projetados para resolver essas funções cognitivas de ordem inferior ainda estão muito distantes da capacidade do cérebro biológico.
    
    \item Bordeux et. al. [8] apresenta uma pipeline de percepção para agentes autônomos, propondo um processo de pré-processamento, processamento e pós-processamento utilizando filtros de percepções. filtro de agente, composto por um filtro de percepção, opcionalmente um filtro semi-reflexivo (que não será discutido por estar fora do escopo do trabalho atual) e uma lista de objetos selecionados. A ideia de guardar a lista de objetos selecionados conversa com o bloco avaliador de nosso modelo, pois guarda informações de determinadas percepções realizadas para serem mais tarde reaproveitadas para um ajuste fino.
    
    \item Em um artigo de revisão sobre arquiteturas cognitivas, Langley et. al [3] caracteriza a importância de tratar de percepções imperfeitas, e mostra que as arquiteturas podem conter elementos para tratar disso, uma vez que os sensores geralmente possuem interferências ou outros tipos de ruídos, que afetam a qualidade da percepção obtida. Segundo o autor, ambientes dinâmicos complicam ainda mais a situação, uma vez que o agente precisa rastrear alterações que ocorrem muitas vezes de maneira repentina no ambiente. A única solução que o autor propõem a isso é o “conhecimento perceptivo”, onde o agente decide sobre quais sensores utilizar, onde e quando focaliza-los e que interferências são plausíveis, se aproximando da percepção ativa.
    
    \item Mesmo em simulações virtuais, às percepções podem levar a erros por conta de sua imprecisão. Sichman [6] ilustra alguns aspectos essenciais de mecanismos de raciocínio sociais, baseado na noção de dependência social. O sistema proposto trata de um modelo de coalizões, onde agentes podem se juntar em prol de um objetivo em comum. O protocolo de formação de coalizões é formado por proposições, aceitações, recusas e mensagens de revisão. As mensagens de revisão são necessárias pois uma possível razão para um agente se recusar a participar de uma coalizão é porque o remetente tem uma crença falsa sobre suas capacidades, ou seja, acreditar que o agente pode executar uma ação, mas na realidade não pode. Isso pode acontecer pois fontes de informação, como as percepções, podem levar a erros.

    \item O trabalho de Pangercic et. al. [7] trata de percepção e processamento de conhecimento, usando como estudo de caso um robô encarregado de determinar quais objetos faltam em uma mesa durante uma refeição, através de inferências lógicas. A implementação resulta em um modelo estatístico, em que cada objeto que o robô conhece recebe uma porcentagem de chance de ser necessário. Apesar dessa alta volatilidade, os autores não tratam da possibilidade da inferência incorreta.
    
    \item Diab et. al. apresenta um framework de processamento de conhecimento para a manipulação de percepção de robôs autônomos, através de raciocínio de percepção, isto é, raciocínio relacionado às características perceptivas dos objetos no ambiente, como outros modelos também fazem, mas além disso adiciona raciocínio relacionado aos algoritmos que o sensor pode executar para extrair os recursos, aos sensores associados ao robô e as limitações físicas dos sensores. Segundo os autores, esse processo torna o robô mais inteligente e flexível. Essa flexibilidade pode ser útil para lidar com a falhas de um sensores, fornecendo alternativas, ou seja, o framework proposto tem flexibilidade para lidar com sistemas sensoriais de vários modelos.
    
\end{itemize}

\textbf{Problema abordado no artigo:} Conforme abordado pelo artigo [3], interferências ocorrem entre o processo físico de percepção e o processamento final dela, dentro do ciclo cognitivo do agente. Portanto, esse trabalho ataca essa lacuna que existe, com o objetivo de minimizar percepções incorretas ou falhas completas de percepção.

\textbf{Contribuição do artigo:} Nesse artigo é apresentado um modelo genérico para o tratamento de percepções, com o objetivo de evitar que informação potencialmente útil seja desperdiçada, criando novos planos quando o agente não está pronto para lidar com percepções que estão fora do seu planejamento inicial. Esse modelo foi construído para que possa ser acoplado a qualquer arquitetura cognitiva, independente do grau de abstração que ela implemente.

\textbf{fim da seção de organização dos trabalhos relacionados}
\fi
%%\chapter{Modelo de revisão de percepções}

\label{chapter:model}

Neste capítulo, será descrito e formalizado o modelo proposto nesse trabalho. Ele foi inspirado pelos conceitos de ilusão e alucinação apresentados no Capítulo \ref{conceitos-fundamentais}, que o nomeiam -- o nome HAIL vem da junção das palavras \textit{hallucination} e \textit{illusion}, alucinação e ilusão em inglês, respectivamente. Seu objetivo é identificar anomalias nas percepções recebidas por um agente qualquer e torná-las informação úteis na forma de novos planos.

\section{Visão geral do modelo}

O modelo HAIL foi desenvolvido para que seja possível adicioná-lo a qualquer agente, independente de arquitetura cognitiva, como um componente que conecta as percepções vindas do ambiente ao agente. O HAIL pode ser separado em dois módulos, como mostra a Figura \ref{fig:method}. De maneira geral, o funcionamento e a comunicação desses módulos se dá da seguinte maneira:

\begin{enumerate}
    \item As percepções recebidas pelo modelo são refinadas pelo módulo de refinamento;
    \item As percepções refinadas passam pelo módulo de alucinação e ilusão onde são categorizadas entre:  percepções válidas, alucinações, ilusões classe 1 e ilusões classe 2;
    \item As percepções válidas são encaminhadas para o raciocínio do agente, enquanto as anomalias continuam no módulo de alucinação e ilusão armazenadas em estruturas chamadas de bloco avaliador;
    \item Quando os requisitos estabelecidos pelo bloco avaliador são cumpridos, as anomalias são selecionadas para passarem pelo processo de planejamento automatizado, alimentando o agente com novos planos.\end{enumerate}

% Primeiro, o agente recebe um conjunto de percepções $p$ através de seus sensores. Depois disso, essas percepções passam através de uma função de refinamento, onde as percepções $p$ são refinadas em um subconjunto próprio $\rho$ de percepções refinadas. O conjunto $\rho$ é então usado como entrada para o módulo de alucinação e ilusão, onde cada percepção refinada passa por um processo de detecção de anomalia. As percepções de $\rho$ que forem consideradas válidas, irão constituir o subconjunto próprio $\varphi$ de percepções válidas, enquanto as anomalias formarão o subconjunto próprio $\sigma$ de anomalias. As percepções válidas são enviadas direto para o ciclo de raciocínio do agente, enquanto as anomalias são enviadas para uma lista ordenada, onde aguardarão para passarem por um processo de planejamento automatizado. Percepções válidas e anomalias podem ser utilizadas para realimentar o módulo de refinamento, de acordo com o tipo de módulo de refinamento implementado.

Para ajudar a compreender o funcionamento do HAIL integrado ao raciocínio de um agente, nós usaremos uma versão estendida do Exemplo \ref{example::robo}, com algumas adições para podermos demonstrar passo a passo como funciona o modelo.

\begin{example}
    Partindo do exemplo \ref{example::robo}, vamos supor que agora a loja vende estrelas lisas e listradas, mas elas não devem ser empacotadas. Além disso, o mesmo robô responsável por empacotar os itens que passavam por uma esteira é responsável por empacotar itens de três diferentes esteiras. As percepções são as mesmas que antes, mas agora ele é capaz de perceber os itens nas três esteiras, através de novos sensores táteis e de uma câmera que capta uma imagem aberta o suficiente para isso. As percepções continuam sendo do tipo \texttt{forma(textura)}.
    \label{example::robo2}
\end{example}{}

O Exemplo \ref{example::robo2} será estendido em casos específicos nas seções seguintes.

\begin{figure}[h!]
    \centering
    \caption{Visão geral do modelo HAIL.}
    \includegraphics[width=1\textwidth]{images/modelo_geral.png}
    \legend{Fonte: Autor.}
    \label{fig:method}
\end{figure}

\section{Módulo de Refinamento}

\label{refinamento}

O módulo de refinamento funciona como uma primeiro filtro para que percepções indesejadas pelo agente não cheguem até seu ciclo de raciocínio. O processo de refinamento é descrito pela Definição \ref{def:refinamento}.

O processo de refinamento não é obrigatório. Caso não seja de interesse de uma determina implementação do HAIL refinar suas percepções, basta que a função do módulo de refinamento seja a função identidade $f(x) = x$, possuindo assim $\rho = p$.

\begin{example}
    Continuando o exemplo \ref{example::robo2}, as estrelas não fazem parte da área de atuação desse robô, e portanto para otimizarmos o processo de empacotamento é possível utilizar o módulo de refinamento para reduzir a informação desnecessárias enviadas para o raciocínio do agente. Nesse exemplo, o HAIL pode ser implementado com uma função $\theta$ que realiza uma filtragem simbólica, removendo as percepções que não fazem parte do contexto. O conjunto de percepções $p$ passa pelo processo de filtragem e retorna $\rho$. Neste exemplo, sendo $s$ o conjunto de percepções possíveis envolvendo estrelas, a função de refinamento possui um comportamento tal que o conjunto $p$ sob a operação $\theta$ retorna $\rho = p \cap \overline{s}$, ou seja, o agente possui um filtro que remove as percepções que envolvem estrelas.
    
    Por exemplo, vamos supor que o agente recebe o conjunto $p_i$ de percepções, composto por $\{circulo(listrado), triangulo(liso), estrela(amarela)\}$. A operação $\theta$ vai remover de $p_i$ os elementos contidos no conjunto $s$ de possíveis percepções envolvendo estrelas, conforme foi descrito anteriormente. Portanto, a saída do bloco de percepções será $\rho = \{circulo(listrado), triangulo(liso)\}$.
    
\end{example}{}

\section{Módulo de alucinação e ilusão}

\begin{figure}[h]
    \centering
    \caption{Módulo de alucinação e ilusão.}
    \includegraphics[width=1\textwidth]{images/diagrama-modelo.png}
    \legend{Fonte: Autor.}
    \label{fig:model}
\end{figure}
 
 A Figura \ref{fig:model} apresenta um diagrama do funcionamento do módulo de alucinação e ilusão. Sua função é receber todas as percepções que passaram pelo processo de refinamento, e detectar quais delas são anomalias. Para isso, primeiro cada percepção $\rho(x)$ da entrada $\rho$ (o conjunto de percepções filtradas) é dirigida para o decisor 1. O primeiro decisor responde a pergunta: ``A percepção recebida faz parte do contexto do agente?''. Caso a resposta for sim, consideramos a percepção como válida, e ela é enviada para o raciocínio do agente. Caso a resposta seja não, consideramos $\rho(x)$ uma anomalia, e enviamos ela para o segundo decisor, que responde a pergunta: ``O corpo ou o argumento do predicado $\rho(x)$ faz parte do contexto do agente?''. Caso a resposta seja não, concluímos que a anomalia é uma alucinação. Caso contrário, ela é considerada uma ilusão e é enviada para o terceiro decisor. O terceiro decisor responde a pergunta: ``O corpo do predicado $\rho(x)$ faz parte do contexto do agente?''. Caso a resposta for sim, a ilusão é considerada uma ilusão classe 1, caso contrário, é considerado uma ilusão classe 2. Essa cadeia de decisores pode ser representada pelo Algoritmo \ref{algorithm:decisor}.

\begin{algorithm}[H]
\Entrada{contexto \textit{c} do agente, percepção $\rho(x)$}
\Inicio{
 \label{algorithm:decisor}
  \uSe{$\rho(x)$ está em \textit{c}}{
  $\rho(x)$ é uma percepção válida\;
  }\uSenaoSe{nem $\rho$ nem $x$ estão em c}{
  $\rho(x)$ é uma alucinação\;
  }\uSenaoSe{$\rho$ está em \texttt{c}}{
  $\rho(x)$ é uma ilusão classe 1\;
  }\Senao{$\rho(x)$ é uma ilusão classe 2}
}
 \caption{Funcionamento dos decisores do módulo de alucinação e ilusão.}
\end{algorithm}

% \begin{algorithm}[H]
% \SetKwInOut{Input}{input}
% \Input{agent context \textit{c}, perception $\rho(x)$}

%   \uIf{$\rho(x)$ is in \textit{c}}{
%   $\rho(x)$ is a valid perception\;
%   }\uElseIf{neither $\rho$ nor $x$ is in c}{
%   $\rho(x)$ is a hallucination\;
%   }\uElseIf{$\rho$ is in \texttt{c}}{
%   $\rho(x)$ is an illusion class 1\;
%   }\Else{$\rho(x)$ is an illusion class 2}
%  \label{algorithm:decisor}
%  \caption{Funcionamento dos decisores do módulo de alucinação e ilusão.}
% \end{algorithm}

\begin{example}
    Para entender como as percepções são tratadas pelos decisores, vamos supor algumas entradas possíveis para o agente de nossos exemplos. Vamos analisar dois casos, uma percepção válida e uma anomalia:
    
    \begin{itemize}
        \item \texttt{quadrado(riscado)} -- Essa é uma percepção completamente válida dentro do contexto do agente, pois ele possui um plano específico para tratá-la, que é empacotar o objeto com o papel vermelho, portanto vai fazer parte do conjunto de percepções refinadas $\rho$. Após sair do módulo de refinamento percepções, a percepção é recebida como entrada pelo decisor 1, que detecta que existe um plano específico para tratar da entrada, portanto a percepção é considerada válida e é diretamente enviada para o raciocínio do agente.
        
        \item \texttt{lua(serrilhada)} --  Lua não é um item que deveria ser percebido pelo agente, pois não faz parte dos itens que deveriam ser inseridos na esteira. Todavia, a função de refinamento de nosso agente simplesmente remove as percepções que envolvem estrelas, fazendo com que a percepção \texttt{lua(serrilhada)} chegue ao módulo de alucinação e ilusão. Dentro desse módulo, quando ela chega ao decisor 1, é classificada como anomalia, pois não faz parte do contexto do agente. Em seguida, a percepção é enviada para o decisor 2, que verifica que nem \texttt{lua} nem \texttt{serrilhada} fazem parte do contexto do agente, ou seja, é uma alucinação. Uma vez detectada a alucinação, a percepção segue para o bloco avaliador.
    \end{itemize}
    \label{example:modulo}

\end{example}{}

\subsection{Bloco avaliador}

Após uma anomalia ser classificada, ela é enviada para um bloco avaliador, que tratará de decidir qual anomalia pode ser considerada relevante para ser usada no planejamento automatizado e quais podem ser descartadas. De acordo com a implementação do módulo de refinamento, as percepções que são classificadas como anomalias podem ser utilizadas para otimizar o processo de refinamento.

Em nosso modelo, utilizamos três blocos avaliadores: um para alucinações, um para ilusões classe 1 e outro para ilusões classe 2. Eles são separados para permitir que a prioridade de tratamento de alucinações seja definida individualmente, de acordo com a necessidade do agente que implementa o modelo.

O objetivo do bloco avaliador é decidir quando alucinações e ilusões que foram recebidas devem ser processadas, evitando que o planejamento automatizado que será realizado em seguida tenha impacto no tempo de execução de um ciclo de raciocínio do agente. Para isso, utilizamos uma lista ordenada por peso como escalonador. O principio do funcionamento da lista ordenada por peso é o mesmo de uma fila, \textit{first in first out} (FIFO), mas atribui um peso a cada entrada que aumenta quando novos elementos iguais são inseridos. Quando um elemento é inserido pela primeira vez na fila, ele recebe o peso 1, e quando uma cópia do mesmo elemento é inserida o elemento tem seu peso aumentado em 1, como mostra a Figura \ref{filaPonderada}. Quando uma operação de remoção é executada, o elemento de maior peso é removido. Se dois ou mais elementos tiverem o maior peso, aquele que foi inserido primeiro é removido.

O bloco avaliador seleciona quando uma percepção deve ser tratada através de uma função matemática, levando em conta o tempo médio de processamento de uma percepção válida e de uma anomalia. Além desse funcionamento básico, o bloco avaliador ainda contém um mecanismo para remover anomalias classificadas com irrelevantes para o sistema, através de uma função de limpeza. Caso essa função retorne verdadeiro, todos os elementos de peso 1 da sua respectiva lista são removidos. Essas duas funções são descritas com mais detalhes na Seção \ref{section:formalizacao}.

\begin{figure}[h!]
    \centering
    \caption{Exemplo do funcionamento de uma lista ordenada por peso.}
    \includegraphics[width=0.8\textwidth]{images/filaPonderada.png}
    \legend{Fonte: Autor}
    \label{filaPonderada}
\end{figure}

\subsection{Bloco de planejamento automatizado}

O bloco de planejamento automatizado é potencialmente a parte mais custosa computacionalmente, o que pode ser um gargalo do sistema, principalmente caso o agente funcione em tempo real e receba um volume muito elevado de percepções por segundo. Um planejamento automatizado implementado de maneira puramente simbólica tende a ser complexo computacionalmente, uma vez que pode considerar milhares de alternativas para o estado de mundo atual, tentando chegar mais perto de seu objetivo. Um processo de planejamento automatizado conexionista é uma alternativa, uma vez que estamos tratando de uma análise incompleta do mundo. Caso seja possível, teorias com maior custo computacional (como criatividade computacional \cite{colton2012computational}) podem ser aplicadas aqui para um resultado ainda mais preciso.

Uma percepção chega ao bloco de planejamento automatizado uma vez que ela seja a primeira na fila ponderada e a função de processamento retorne verdadeiro em sua verificação. De um ciclo para outro, as percepções permanecem na fila, a não ser que sejam descartadas pelo mecanismo de limpeza. Nosso modelo não explicita qual é a ordem que os blocos avaliadores devem processar suas filas para mandar anomalias para o planejamento automatizado (isto é, se deve primeiro ser priorizada as anomalias do bloco avaliador de alucinações, ilusões classe 1 ou ilusões classe 2), ficando a cargo da implementação em questão tomar essa decisão.

\begin{example}
    Para mostrar o caminho que faz uma ilusão, vamos considerar que o agente recebe em duas esteiras a percepção \texttt{triangulo(listrado)}. Não existem triângulos listrados na loja, mas como $\theta$ não filtra essa percepção, ela vai chegar ao decisor 1. 
    
    Os planos do agente são: \texttt{papel(vermelho) :- quadrado(\_) OR circulo(liso)} e 
     \texttt{papel(azul) :- circulo(listrado) OR triangulo(liso)}, e como não existe um plano específico para tratar a percepção \texttt{triangulo(listrado)}, ela é considerada uma anomalia, e é encaminhada ao decisor 2. Como existe um plano para tratar de triângulos lisos, a percepção é então classificada como uma ilusão, e vai para o terceiro decisor. Nele, como o plano trata triângulos, a percepção é considerada uma ilusão classe 1. 
     
     Após a percepção ter sido classificada corretamente, ela é enviada para o bloco avaliador, e é inserida na fila ponderada com peso 1. Depois disso, a percepção da segunda esteira, que é igual a que já foi tratada, chega ao bloco de alucinação e ilusão. Essa percepção vai fazer o mesmo caminho até o bloco avaliador e o peso da anomalia já presente na fila é aumentada para 2. Como duas percepções foram consideradas anomalias, vamos considerar que a função de processamento seja satisfeita, e o bloco avaliador passe essa anomalia para o bloco de planejamento automatizado.
    
    Após uma percepção ser considerada relevante para ser enviada ao planejamento automatizado, ela deve gerar um novo plano para aquela percepção a ser adicionada ao conjunto de planos do agente, e essa percepção então deixará de ser uma anomalia. Ilusões e alucinações tem blocos de planejamento automatizado separados para permitir que duas implementações completamente diferentes sejam utilizadas de acordo com a função do agente e suas particularidades. 
    
    Retomando o exemplo \ref{example:modulo}, no segundo caso, quando o planejamento automatizado de alucinações recebe a percepção \texttt{lua(serrilhada)}, um processo de planejamento automatizado deve ser executado. Para esse exemplo, consideremos um planejamento automatizado puramente simbólico, que analisa uma grande quantidade de estados futuros possíveis para o ambiente em que o robô está inserido e seleciona aquele que será mais eficiente para que o agente se aproxime de seu objetivo principal (terminar de empacotar os itens). Isso é extremamente custoso, mas como alucinações são muito raras para esse agente em questão, pois ele já elimina as percepções inválidas reconhecidas no projeto durante o refinamento, não terá um impacto grande no desempenho do agente. No exemplo, é possível que a lua seja um objeto novo que está sendo vendido, portanto, o agente precisa embalar esse item. Ao detectar uma semelhança entre a lua serrilhada e o circulo listrado, o planejamento automatizado pode determinar que esse novo objeto deve ser embalado com o papel azul, criando assim um novo plano da forma \texttt{papel(azul) :- lua(serrilhada)} é adicionado, e \texttt{lua(serrilhada)} deixa de ser uma alucinação.

    No caso da percepção \texttt{triângulo(listrado)}, o agente poderia ter um bloco de planejamento automatizado baseado em uma rede bayesiana, uma vez que ilusões podem ser muito mais comuns e o agente já tem uma breve noção do que deve ser feita com objetos do tipo. \texttt{triângulo(listrado)} pode ser um novo item a venda na loja, e como já foi verificado em duas esteiras diferentes, faz sentido que ele não seja uma mera falha de sensores. Assim, o planejamento automatizado pode inferir o plano \texttt{triângulo(listrado) -> empacotar}, permitindo que o agente tenha um aprendizado dinâmico resultado da adição de um novo plano em seu conjunto de planos. Assim como no caso da ilusão, uma vez que esse plano novo foi adicionado a percepção original deixa de ser uma anomalia, uma vez que faz parte do contexto em que o agente está trabalhando.
    \label{example::planejamento}
\end{example}{}

\section{Raciocínio do agente}

O raciocínio do agente se refere a todos os fatores externos do modelo de revisão de percepções pertencentes ao agente, principalmente a arquitetura cognitiva utilizada para implementá-lo. Em um agente qualquer, o raciocínio é alimentado principalmente pelas percepções. Outras fontes de conhecimento, como comunicação, podem existir também, mas para o modelo HAIL consideramos toda forma de entrada como percepção. Em um agente que implemente o modelo HAIL, o raciocínio recebe tanto as percepções consideradas válidas pelo módulo de alucinação e ilusão quanto os novos planos criados a partir do planejamento automatizado.
%% \chapter{Implementação e Simulação}
\chapter{Material e Métodos}

A partir da construção do modelo proposto no capítulo anterior, a próxima etapa foi a sua formalização. Após termos um modelo formal, construímos um pseudocódigo que pode ser implementado em qualquer linguagem de programação. Para validar o modelo, foi feita uma implementação genéria (sem arquitetura cognitiva) em Python, apresentada no próximo capítulo).

\section{Formalização}

\label{section:formalizacao}

Nesta seção vamos formalizar o modelo HAIL através de um modelo de cascata, partindo dos conceitos mais gerais e afunilando para as definições mais específicas e fórmulas matemáticas utilizadas. Para isso, começamos com uma única tupla que define de maneira geral o que é o modelo que se desdobra para os demais conceitos. O objetivo disso é criar camadas de abstração, as quais podem ser modificadas de acordo com a necessidade de implementações específicas ou da integração com arquiteturas cognitivas ou outros modelos.

 O bloco básico do modelo de revisão de percepções proposto, chamado de HAIL, é composto por um módulo para alucinação e ilusão $M_{ai}$ e uma função de refinamento $\theta$, conforme descrito na Definição \ref{def:modeloHAIL}. O módulo de ilusão e alucinação é uma quádrupla, apresentada na definição \ref{def:illuHallu}. A função de refinamento é uma função abstrata, cuja entrada é obtida através dos sensores do agentes e a saída é a entrada do módulo de alucinação e ilusão, conforme já foi descrito anteriormente na Seção \ref{refinamento}.
 
 \begin{definition}{}
    O modelo de revisão de percepções HAIL é uma dupla $HAIL = \langle M_{ai}, \theta \rangle$, onde:
    
    \begin{itemize}
        \item $M_{ai}$ é o módulo de ilusão e alucinação; e
        \item $\theta$ é a função de refinamento $\theta(p) = \rho$, onde $p$ é um conjunto de percepções e $\rho$ é um subconjunto próprio de $p$.
    \end{itemize}{}
    \label{def:modeloHAIL}
\end{definition}

Após ter passado pela função $\theta$, as percepções $\rho$ irão passar pelo Algoritmo \ref{algorithm:decisor}, e serão encaminhadas de acordo com sua classificação. O bloco de ilusão e alucinação é descrito por uma quádrupla, com conjuntos de decisores, blocos e uma função de transição.

\begin{definition}
\label{def:illuHallu}
    O bloco de ilusão e alucinação é uma quádrupla $M_{ai} = \langle D, Ab, Ap, \Delta \rangle$, onde:
    
    \begin{itemize}
        \item $D$ é o conjunto de decisores $D = \{d_{a}, d_{h}, d_{i}\}$, onde:
             \begin{itemize}
                \item $d_{a}$ é o decisor de anomalias, definido pela função:
                \[ d_{a} = \left\{ \begin{array}{ll}
                0 & \mbox{se $\rho(x)$ está em $c$\footnotemark};\\
                1 & \mbox{se $\rho(x)$ não está em $c$}.\end{array} \right. \]
             
                \item $d_{h}$ é o decisor de alucinação, definido pela função:
                \[ d_{h} = \left\{ \begin{array}{ll}
                0 & \mbox{se nem $\rho$ nem $(x)$ está em $c$};\\
                1 & \mbox{se $\rho$ ou $(x)$ está em $c$}.\end{array} \right. \]
                
                \item $d_{i}$ é o decisor de ilusão, definido pela função:
                \[ d_{i} = \left\{ \begin{array}{ll}
                0 & \mbox{se $\rho$ está em $c$};\\
                1 & \mbox{se $(x)$ está em $c$}.\end{array} \right. \]
            \end{itemize}
        
        \footnotetext{ $c$ é o contexto do agente, de acordo com a definição \ref{definition::context}.}
        
        \item $Ab$ é o conjunto de blocos avaliadores $Ab = \{Ab_{h}, Ab_{i1}, Ab_{i2}\}$, onde $Ab_{h}$ é o bloco avaliador de alucinações, $Ab_{i1}$ é o bloco avaliador de ilusões classe 1 e $Ab_{i2}$ é o bloco avaliador de ilusões classe 2.
        
        \item $Ap$ é o conjunto de blocos de planejamento automatizado $Ap = \{Ap_{h}, Ap_{i}\}$, onde $Ap_{h}$ é o bloco de planejamento automatizado de alucinações e $Ap_{i}$ é o bloco de planejamento automatizado de ilusões.
        
        \item $\Delta$ é a função de transição definido pela tabela abaixo, onde $out$ é um estado final, que leva a percepção para fora do modelo de revisão de percepções, ou seja, pode tanto significar o encaminhamento de uma percepção válida para o raciocínio do agente quanto o fim da execução de um ciclo de revisão.
        
            \begin{table}[htb]
                \caption{Função de transição $\Delta$ do módulo de ilusão e alucinação.}
                \centering
                \begin{tabular}{c c c c} 
                    \toprule
                    \textbf{Estado} & \textbf{0} & \textbf{1} \\
                    \midrule
                    $d_{a}$     & $out$     & $d_{h}$       \\
                    $d_{h}$     & $Ab_{h}$  & $d_{i}$       \\
                    $d_{i}$     & $Ab_{i1}$ & $Ab_{i2}$     \\
                    $Ab_{h}$    & $out$     & $Ap_{h}$      \\
                    $Ab_{i1}$   & $out$     & $Ap_{i}$      \\
                    $Ab_{i2}$   & $out$     & $Ap_{i}$      \\
                    \bottomrule
                \end{tabular}
                \label{transition-table}
                \legend{Fonte: Autor.}
                
            \end{table}
    \end{itemize}{}
\end{definition}{}

O módulo de ilusão e alucinação é o artefato principal do modelo. Ele recebe uma entrada $\rho$, que é a saída da função de refinamento apresentada na definição 1, e processa cada um dos elementos $\rho(x)$ desse conjunto, através de decisores e blocos de avaliação, percorrendo o modelo de acordo com as transições descritas pela função de transição $\Delta$. Os três decisores do conjunto $D$ fazem a triagem para detectar se a percepção $\rho(x)$ é uma anomalia, e que tipo de anomalia é. Após passar pelos três decisores, saberemos se essa percepção é valida, é uma alucinação, é uma ilusão tipo 1 ou uma ilusão tipo 2. Após ter passado pelos decisores, a percepção ou é levada para a cognição do agente, caso seja válida, ou fica armazenada nos blocos avaliadores, caso seja uma anomalia.

\begin{definition}
    Um bloco avaliador é uma tripla $Ab_{x} = \langle L, Pf, Cf) \rangle$, $x \in \{h, i1, i2\}$, onde:

    \begin{itemize}
        \item $L$ é uma lista ordenada pelo número de vezes que uma mesma anomalia é dada como entrada;
        \item $Pf$ é a função de processamento, definida abaixo:
            
             \[ Pf = \left\{ \begin{array}{ll}
                        1 & \mbox{se $T_{m}(A) \leq T_{m}(V) * (|A| - |A_{pr}|)$;}\\
                        0 & \mbox{caso contrário}.\end{array} \right. \]
        
            
            Onde:
            
            \begin{itemize}
                \item $T_{m}$ é a função que retorna a média do tempo gasto para processar as percepções de um conjunto;
                \item $A$ é o conjunto de anomalias, $A(x)$ é um elemento específico $x$ e $|A|$ o número de anomalias do conjunto;
                \item $A_{pr}$ é o conjunto de anomalias que já foram validadas para serem processadas pela função de processamento neste ciclo de raciocínio ($A_{pr}$ é instanciada vazia a cada ciclo de raciocínio), e $|A_{pr}|$ o número de anomalias desse conjunto.
                \item $V$ é o conjunto de percepções válidas.
            \end{itemize}{}
        
        \item $Cf$ é a função de limpeza definida abaixo com auxílio da função equação de limpeza $Cf$, sendo $\alpha$ um coeficiente de limpeza variável que precisa ser definido pela instância implementada do modelo (por padrão, toma-se $\alpha = 1$):
        
        \[ Cf = \left\{ \begin{array}{ll}
                        1 & \mbox{se  $Ce = Verdadeiro$;}\\
                        0 & \mbox{caso contrário}.\end{array} \right. \]
            
            \[ Ce = \sum_{i=1}^{|L|} W_{n}(L_{i}) > \alpha \sum_{j=1}^{|L|} W_{1}(L_{j}) \]
            
            Onde:
            
            \begin{itemize}
                \item $L$ é a lista ordenada do bloco, sendo $|L|$ seu número de anomalias e $L_{i}$ a anomalia $i$ da lista.
                \item $W$ é a função peso da anomalia $L_{i}$ definida como $W(L_{i}) = |L_{i}|$, sendo $|L_{i}|$ o peso da anomalia especificada (número de entradas recebidas dessa mesma anomalia na lista). A função $W$ é utilizada para especificar as seguintes funções:
                \\
                
                    (i) $ W_{1}(L_{i}) = \left\{ \begin{array}{ll}
                        1 & \mbox{se $W(L_{i}) = 1$;}\\
                        0 & \mbox{caso contrário}.\end{array} \right. $
                \\
                
                    (ii) $ W_{n}(L_{i}) = \left\{ \begin{array}{ll}
                        W(L{i}) & \mbox{se $W(L_{i}) > 1$;}\\
                        0 & \mbox{caso contrário}.\end{array} \right. $
            \end{itemize}{}
    \end{itemize}
\end{definition}{}

A terceira definição é a de bloco de avaliação ($Ab$). Um $Ab$ é um módulo do modelo que é responsável por armazenar as anomalias detectadas e decidir se elas serão processadas pelo agente ou não. É descrito por uma tripla, constituída por uma lista ordenada $L$, uma função de processamento $Pf$ e uma função de limpeza $Cf$. $L$ é uma lista organizada pela recorrência de elementos inseridos nela, onde cada elemento só aparece uma vez e contém um número de vezes que o mesmo elemento já foi inserido nela, chamado de peso. Nesse modelo, os elementos são as anomalias percebidas pelo agente, e o peso é o número de vezes que o agente percebeu a anomalia.

$Pf$ é uma função que avalia se uma anomalia será processada nesse ciclo de raciocínio ou se será armazenada para ser processada no futuro. Para isso, ela precisa de uma função que retorne o tempo médio previsto para o processamento de uma percepção $T_m$, seja ela uma percepção válida ou uma anomalia. Com base nessa função, $Pf$ retorna 1 caso o tempo médio de processamento de uma anomalia seja menor que o tempo médio de processamento de uma percepção válida multiplicado pelo número de anomalias que fazem parte desse ciclo de raciocínio menos o número de anomalias que já foram aprovadas pelo bloco avaliador, e zero caso contrário.

De maneira simplificada, o objetivo dessa função é evitar que o modelo de revisão de percepção gaste mais tempo de processamento do que ele gastaria caso não estivesse sendo utilizado e todas as percepções fossem válidas. Para isso, o bloco avaliador permite processar apenas as anomalias que aparecem de maneira mais recorrente para o agente.

$Cf$ é uma função que realiza a limpeza de $L$. Conforme os ciclos de raciocínio forem passando, $L$ tende a possuir diversas anomalias que foram percebidas apenas uma única vez. Dessa maneira, uma grande quantidade de memória seria necessária para armazenar as possíveis centenas de anomalias que podem nunca ser processadas. Assim, a função $Cf$ verifica se a equação $Ce$ é verdadeira ou falsa. Ela é verdadeira quando o número de anomalias que apareceram uma única vez é maior que a soma dos pesos das anomalias que apareceram mais de uma vez (o peso é o número atrelado a cada anomalia, que representa quantas vezes elas já foram inseridas na lista). Quando a função for verdadeira, o bloco avaliador remove todas as anomalias de peso 1 da lista.

O coeficiente de limpeza $\alpha$ presente em $Ce$ pode ser utilizado na implementação do HAIL quando é necessário que a limpeza seja mais ou menos recorrente. Por padrão, caso a instância do HAIL não queira alterar a frequência da limpeza das filas, utiliza-se 1. Utilizando valores menores que 1 a função será verdadeira mais facilmente, e o contrário caso o valor seja maior que isso.

Quando $P_f$ retorna 1, a anomalia do topo da lista de algum dos três blocos avaliadores é removida, e usada como entrada do bloco de planejamento automatizado. Qual bloco avaliador tem a prioridade para ser utilizado é uma decisão que deve ser tomada pela implementação.

\begin{definition}
    Um bloco de planejamento automatizado é uma instância do modelo conceitual de planejamento automatizado (Definição \ref{definition::autoplanning}).
\end{definition}

Por fim, a saída do planejamento automatizado, que é um plano ou um conjunto de planos novos que o agente deve adicionar ao conjunto de planos que possui, é enviada para o raciocínio do agente, dando fim ao processo de revisão de percepções daquele ciclo.

\section{Generalidade}

O HAIL foi um modelo construído para ser genérico. Sua generalidade pode ser avaliada em dois quesitos: a capacidade de ser acoplado a qualquer arquitetura cognitiva ou modelo de agente inteligente e a capacidade de ser modificado de acordo com as necessidades da arquitetura à qual será acoplado.

Quanto a capacidade de acoplamento, podemos considerar o HAIL um modelo genérico uma vez que ele atua sobre as percepções do agente e não sobre seu processo cognitivo, ou seja, sua entrada é composta apenas por um subconjunto das percepções recebidas e os planos que podem ser gerados pelo módulo de planejamento automatizado. 

No Capítulo \ref{conceitos-fundamentais}, descrevemos um agente como uma entidade que recebe percepções do ambiente e o modifica através de ações. A Figura \ref{fig:generalidade} demostra a diferença entre a interação normal de um agente com o ambiente e quando há a adoção do modelo HAIL. Portanto, contanto que o agente utilize mecanismos de percepção e raciocínio que permitam definir bem seu contexto, permitindo que o HAIL possa ser implementado conforme definido na Seção \ref{section:formalizacao}, o modelo pode ser inserido e removido do processo de percepção do agente livremente.

\begin{figure}
    \centering
    \caption{Interação entre o ambiente e um agente sem e com o modelo HAIL.}
    \includegraphics[width=\textwidth]{images/generalidade.png}
    \legend{Fonte: Autor}
    \label{fig:generalidade}
\end{figure}

O outro aspecto da generalidade do modelo diz respeito a sua alterabilidade. As principais partes modificáveis do HAIL são o módulo de revisão de percepções, o coeficiente de limpeza $\alpha$ utilizado pela função de limpeza dos blocos avaliadores e os blocos avaliadores.

O módulo de revisão de percepções pode ser implementado para utilizar qualquer técnica de refinamento de percepções, ou então não utilizar técnica alguma. Isso foi definido pois dependendo do tipo de ambiente no qual o agente está inserido pode ser necessário analisar todas as percepções recebidas ou filtrar de maneira bastante rígida quais percepções serão processadas.

O coeficiente de limpeza foi criado para que seja possível ajustar a frequência com a qual as listas ponderadas são limpas, tornando possível que agentes em situações com muitas anomalias não utilizem tanta memória e agentes em situações críticas possam analisar anomalias dentro de um maior período de tempo.

Por fim, o bloco de planejamento automatizado não teve uma estratégia ou arquitetura definida pois esse tipo de mecanismo é bastante complexo e não era o foco do trabalho. Além disso, como foi mostrado no Exemplo \ref{example::planejamento}, deixar a implementação do planejamento automatizado em aberto permite que o desenvolvedor adéque o modelo conforme a necessidade.

\section{Pseudocódigo}

O pseudocódigo apresentado nessa seção é um esqueleto na qual implementações reais do modelo proposto possam se basear. A sequência de comandos apresentada deve ser aplicada entre o momento que a percepção é recebida pelo agente e a efetiva entrada da percepção no seu ciclo de raciocínio.

\begin{algorithm}[h]
\While{True}{
    $p \leftarrow$ sensors.percept\\
    $\rho \leftarrow \theta(p)$\\
    
    $c \leftarrow$ agent.getContext\\
    
    \ForEach{perception $\rho(x)$ in $\rho$}{
        decide $\rho(x)$ perception type based on $c$\\
        \If{$\rho(x)$ is hallucination}{\\
            $Ab_{h} \leftarrow \rho(x)$ \\
            $A \leftarrow \rho(x)$\\ 
        }
        \ElseIf{$\rho(x)$ is illusion class 1}{
            $Ab_{i1} \leftarrow \rho(x)$ \\
            $A \leftarrow \rho(x)$\\ 
        }
        \ElseIf{$\rho(x)$ is illusion class 2}{
            $Ab_{i2} \leftarrow \rho(x)$ \\
            $A \leftarrow \rho(x)$\\ 
        }
        \Else{
            $V \leftarrow \rho(x)$\\ 
        }
        
    }
    x $\leftarrow$ choose(i1, i2, h)\\
        
        \While{$Ab_x$ == True}{
            $t \leftarrow$ top of $Ab_x$'s L
            $plan \leftarrow Ap_{x}$.autoplan($t$)\\
            add $plan$ to P\\
            add $t$ to $A_{pr}$\\
        }
        \ForEach{y in (i1, i2, h)}{
            \If{$Ab_{y}.Cf$ == True}{
                clean $Ab_x$'s L
            }
        }
    }
    \caption{Algoritmo de avaliação de percepções}
    \label{algorithm:model}
\end{algorithm}

Para facilitar a compreensão do algoritmo, iremos explicá-lo linha a linha.

Colocamos o algoritmo dentro de um \emph{while true} para representar a continuidade do algoritmo, que executa todo ciclo de raciocínio. Nas linhas 2 e 3, o agente recebe as percepções através dos sensores, e aplica elas na função de refinamento $\theta$. Na linha 4, colocamos o contexto do agente na variável c. Isso é necessário pois sempre que esse algoritmo executa, é possível que o contexto mude, uma vez que novas percepções e planos passam pelo processo de cognição. Na linha 6, a percepção passa pelo processo de classificação, descrito em detalhes no algoritmo \ref{algorithm:decisor}. Entre a linha 7 e 21, caso a percepção seja uma anomalia, ela é adicionada ao bloco avaliador corresponde, e também ao conjunto de anomalias. Caso seja uma percepção válida, é apenas adicionada ao conjunto de percepções validas. Na linha 23, é realizado um processo de escolha de qual tipo de anomalia será tratada nesse ciclo de raciocínio. Essa escolha pode ser aleatória ou utilizar qualquer tipo de métrica escolhida pelo programador ao implementar o modelo. O tipo escolhido é avaliado na linha 24, onde o laço de repetição \emph{while} irá manter o planejamento automatizado recebendo o topo da lista ponderada enquanto a função de processamento permitir. Quando uma percepção é enviada ao planejamento automatizado, ela é adicionada ao conjunto de anomalias processadas. Nesse trecho de código foi realizado uma simplificação, pois não existe $Ap_{i1}$ e $Ap_{i2}$, apenas $Ap_{i}$, mas consideramos isso algo a ser tratado na implementação. Por fim, entre a linha 29 e a 33 é verificado se cada uma das listas ponderadas precisa ser limpa, utilizando a função de limpeza.

%% \chapter{Simulação e Resultados}
\chapter{Resultados e Discussão}

Para analisar as capacidades do modelo proposto, foram realizados três experimentos. Os dois primeiros experimentos foram implementados utilizando o design $2^k$ fatorial \cite{jain1990art}. Esse tipo de design consiste em variar $k$ fatores em 2 níveis diferentes, -1 e 1, que são extremos opostos. Por exemplo, em uma pesquisa ligada a um processador, um fator pode ser o número de núcleos, e seus níveis serem 1 núcleo e 8 núcleos. Portanto, o fator é uma variável livre, que é utilizada para analisar a variação de uma variável dependente qualquer. Os fatores e as variáveis livres utilizadas estão nas tabelas \ref{table:experiments_factors} e \ref{table:experiments_variables}, respectivamente. A análise do impacto dos fatores foi realizada utilizando a equação de regressão não linear do design $2^k$ fatorial. 
%O código implementado para essa análise está disponível no anexo [NÚMERO].

\begin{table}[h!]
    \begin{center}
        \caption{ Fatores utilizados nos experimentos. }
        \label{table:experiments_factors}
        \begin{tabular}{|c|c|c|c|}
        \hline
        \textbf{Fatores} & \textbf{Sigla} & \textbf{Nível -1} & \textbf{Nível 1} \\
        \hline
        Porcentagem de Percepções Inválidas & PPI & 5\% & 95\%  \\
        \hline
        Tempo Médio Gasto Pelo Autoplanejamento & TMA & 1/2 CR & 64 CR \\
        \hline
        Tempo Médio Gasto em um Ciclo de Raciocínio & TMC & 01 CR & 32 CR \\
        \hline
        Número de Percepções Recebidas por Ciclo & NPC & 01 & 16 \\
        \hline
    \end{tabular}{}
    \end{center}
\end{table}{}

\begin{table}[h!]
    \begin{center}
        \caption{ Variáveis dependentes analisadas nos experimentos. }
        \label{table:experiments_variables}
        \begin{tabularx}{\textwidth}{ |Y|Y| }
            \hline
            \textbf{Variáveis} & \textbf{Justificativa} \\
            \hline
            Tempo Virtual Decorrido & Medir desempenho geral do modelo \\
            \hline
            Planos Criados & Avaliar potencial do modelo de inserir aprendizado em arquiteturas que não o possuem \\
            \hline
            Percepções Processadas & Analisar a capacidade do modelo de ganhar desempenho ao longo do tempo\\
            \hline
        \end{tabularx}{}
    \end{center}{}
\end{table}

As simulações consistem na execução do modelo proposto, que foi implementado em Python, submetido a um grande volume de percepções. O agente utilizado na simulação segue os exemplos apresentados no capítulo 4 (robô embrulhador).
Uma simulação possui 5000 ciclos, sendo que cada ciclo pode possuir uma ou várias percepções. Essas percepções podem ser válidas (pertencentes ao contexto do agente) ou inválidas (não pertencentes ao contexto do agente), sendo que a proporção entre o tipo de percepções é definido pela PPI. As percepções são produzidas aleatoriamente por um gerador de percepções. As percepções válidas são geradas sorteando percepções que pertencem ao contexto do agente, e as percepções inválidas são geradas utilizando o pacote \texttt{RandomWords} \cite{pipRandomWords}. Cada ciclo da simulação segue os seguintes passos:

\begin{enumerate}
    \item Gerar as percepções da simulação;
    \item Iterar sobre cada um dos ciclos, passando as percepções para o modelo implementado;
    \item Salvar os resultados, o agente final (com novos planos gerados pelo módulo de planejamento automatizado) e as percepções de cada ciclo em arquivos CSV.
\end{enumerate}

\begin{figure}[h!]
    \centering
    \includegraphics[width=0.8\textwidth]{Images/diagrama-simulacao.png}
    \caption{Diagrama da execução de uma iteração de cada experimento.}
    \label{fig:diagrama-simulacao}
\end{figure}

Além da configuração dos valores dos fatores, é possível configurar se um agente será recarregado (ou seja, se ele deve manter os planos gerados por uma simulação anterior) e se o simulador deve gerar novas percepções ou não, pulando a etapa 1 de cada ciclo.

Como é um experimento feito para uma arquitetura cognitiva genérica, vamos supor que o tempo de processamento que o agente gasta em cada ciclo de raciocínio é constante, e o tempo gasto pelo módulo de planejamento automatizado será calculado em função desse valor, ou seja, se o tempo do ciclo de raciocínio é o mesmo do planejamento automatizado, vamos dizer que o tempo do planejamento automatizado é 1 CR (ciclo de raciocínio).

\section{Experimento 1}

O objetivo do primeiro experimento é analisar a variação no tempo do ciclo de raciocínio e a quantidade de planos criados pelo módulo de planejamento automatizado, de acordo variações nos quatro fatores apresentados. Cada simulação foi repetida dez vezes, para que uma média pudesse ser obtida. Apesar do uso de repetições, não foi usada a metodologia do design $2^k r$ fatorial, que considera as possíveis falhas referente as repetições, pois o resíduo dos experimentos é extremamente baixo, da ordem de grandeza de $e^{-12}$.

\subsection{Resultados}

A média dos resultados de cada simulação estão apresentados na tabela \ref{table:experimento1.}. Os valor final do tempo virtual decorrido varia bastante, principalmente nas simulações com alto nível de percepções inválidas. Em dois momentos o tempo virtual foi 5000. Isso ocorreu pois nessas simulações o TMC é 1, o TMA é 0.5 e o NPC é 16. Portanto, ciclos de percepções válidas consumiam 1 unidade de tempo (4750). O bloco avaliados sempre permitia o processamento de 2 percepções inválidas (pois cada uma toma 0.5 unidades de tempo), e quase sempre há percepções inválidas disponíveis na fila pois entram 16 percepções inválidas por ciclo de percepções inválidas. Portanto, cada vez que o modelo processava percepções inválidas, ele consumia 1 unidade de tempo, e isso aconteceu 250 vezes (5\% de 5000).

\begin{table}[h!]
    \begin{center}
        \caption{ Resultados do experimento 1}
        \label{table:experimento1}
        \begin{tabular}{ |c|c|c|c|c|c|c| }
            \hline
            \textbf{Simulação} & \textbf{PPI} & \textbf{TMC} & \textbf{TMA} & \textbf{NPC} & \textbf{Tempo Virtual} & \textbf{Planos Criados}\\
            \hline
            1 & 5\% & 1 & 1/2 & 1 & 4874.6 & 250.8\\
            \hline
            2 & 5\% & 1 & 1/2 & 16 & 5000.0 & 502.0\\
            \hline
            3 & 5\% & 1 & 64 & 1 & 20806.7 & 250.9\\
            \hline
            4 & 5\% & 1 & 64 & 16 & 20813.0 & 251.0\\
            \hline
            5 & 5\% & 32 & 1/2 & 1 & 152093.5 & 251.0\\
            \hline
            6 & 5\% & 32 & 1/2 & 16 & 153437.3 & 2932.2\\
            \hline
            7 & 5\% & 32 & 64 & 1 & 168028.8 & 250.9\\
            \hline
            8 & 5\% & 32 & 64 & 16 & 168032.0 & 251.0\\
            \hline
            9 & 95\% & 1 & 1/2 & 1 & 3229.2 & 3541.6\\
            \hline
            10 & 95\% & 1 & 1/2 & 16 & 5000.00 & 9303.8\\
            \hline
            11 & 95\% & 1 & 64 & 1 & 228668.9 & 3550.3\\
            \hline
            12 & 95\% & 1 & 64 & 16 & 304300.4 & 4750.8\\
            \hline
            13 & 95\% & 32 & 1/2 & 1 & 48521.5 & 3539.0\\
            \hline
            14 & 95\% & 32 & 1/2 & 16 & 140683.3 & 4073.8\\
            \hline
            15 & 95\% & 32 & 64 & 1 & 273609.6 & 3550.3\\
            \hline
            16 & 95\% & 32 & 64 & 16 & 312032.0 & 4751.0\\
            \hline
        \end{tabular}{}
    \end{center}{}
\end{table}

O número de planos criados se agruparam em certos conjuntos de valores, de acordo com os níveis dos fatores. Como esperado, as simulações com menor porcentagem de percepções inválidas criaram menos planos, pois houveram menos anomalias, logo menos material para o módulo de planejamento automatizado trabalhar. O gráfico da figura \ref{fig:pc_occurrences} mostra a ocorrência dos resultados em certos valores aproximados.

\begin{figure}[h!]
    \centering
    \includegraphics[width=0.8\textwidth]{Images/plans_created_occurrences.png}
    \caption{Recorrência do valor final de Planos Criados nas simulações realizadas.}
    \label{fig:pc_occurrences}
\end{figure}

\subsection{Análise de Fatores}

Conforme mostra a tabela \ref{tab:experimento1fatores}, na variável dependente de Tempo Virtual, os fatores TMC e TMA isolados tiveram uma alta porcentagem de impacto. Isso porque a contagem do tempo virtual acontecia em função justamente dos eventos de realizar planejamento automatizado e executar um ciclo de raciocínio. Portanto, a interferência nesses fatores influencia diretamente o resultado final. Além disso, a combinação dos fatores PPI + TMA teve um impacto bastante grande, de 24.13\%. Isoladamente, PPI e TMA possuem porcentagens altas (12.69 e 31.64, respectivamente), portanto era esperado que a sua
combinação também tivesse uma porcentagem alta. Entretanto, TMC possui 22.20 porcento de impacto, mas a combinação PPI + TMC possui apenas 4.16. Isso demonstra como uma combinação de alta taxa de percepções inválidas com um tempo de planejamento automatizado alto podem ser impactantes no resultado de um agente que executa o modelo.

\begin{table}
    \begin{center}
        \caption{Análise dos fatores do experimento 1}
        \label{tab:experimento1fatores}
        \begin{tabular}{ |c|c|c| }
            \hline
            \textbf{Fator} & \textbf{Efeito Tempo Virtual} & \textbf{Efeito Planos Criados}\\  
            \hline
            PPI & 12.69\% & 66.10\%\\
            \hline
            TMC & 22.20\% & 0.50\%\\
            \hline
            TMA & 31.64\% & 2.95\%\\
            \hline
            NPC & 1.44\% & 8.67\%\\
            \hline
            PPI + TMC & 4.16\% & 3.76\%\\
            \hline
            PPI + TMA & 24.13\% & 0.05\%\\
            \hline
            PPI + NPC & 1.39\% & 2.13\%\\
            \hline
            TMC + TMA & 0.55\% & 0.50\%\\
            \hline
            TMC + NPC & 0.10\% & 0.50\%\\
            \hline
            TMA + NPC & 0.01\% & 2.99\%\\
            \hline
            PPI + TMC + TMA & 0.53\% & 3.76\%\\
            \hline
            PPI + TMC + NPC & 0.09\% & 3.75\%\\
            \hline
            PPI + TMA + NPC & 0.02\% & 0.06\%\\
            \hline
            TMC + TMA + NPC & 0.54\% & 0.50\%\\
            \hline
            PPI + TMC + TMA + NPC & 0.52\% & 3.76\%\\
            \hline
        \end{tabular}{}
    \end{center}{}
\end{table}


Na variável dependente Planos Criados, a fator que mais impactou o resultado foi o PPI. Isso era o esperado, pois quanto mais percepções inválidas o modelo recebe, mais planos podem ser criados. Os outros fatores tiveram impactos bem mais baixos. O NPC teve 8.67\% de influência pois simulações com mais percepções inválidas por ciclo preenchem a fila ponderada mais rápido, sempre tendo combustível para alimentar o bloco de planejamento automatizado.

%%%%%%%%%%%%%%%%%%%%%%%%%%%%%%%%%%%%%%%%%%%%%%%%%%%%%%%%%%%%%%%%%%%%

\section{Simulação 2}

A segunda simulação foi realizada para analisar o ganho de performance de um agente ao aplicar os planos criados com o bloco de autoplanejamento. Esse segundo experimento segue os moldes do primeiro, porém foi realizado apenas uma iteração para cada configuração de fatores, e após ser realizado uma simulação com uma dada configuração, foi realizada uma nova simulação com os mesmos fatores, mas usando o agente resultante da primeira. Assim, os planos criados pelo agente foram reaproveitados.

Essa simulação pode ter sua configuração de fatores separadas em dois grupos: (I) o tempo de processamento de um ciclo de raciocínio é menor do que o tempo do autopĺanejamento; (II) o tempo do autoplanejamento é menor que o tempo de processamento de um ciclo de raciocínio. Essa separação pode ser feita pois quanto mais planos criados temos, menos percepções inválidas serão recebidas pelo agente. Portanto, se o tempo de processamento de um ciclo for maior que o de autoplanejamento, o tempo virtual gasto total usando um agente que já aprendeu vários planos será maior. Isso é um \textit{trade off} pois agora essas novas
percepções (que eram antes anomalias) estão efetivamente sendo utilizadas pelo agente.

\subsection{Resultados}

Os resultados estão separados em tabelas por variável dependente e por grupo, como descrito anteriormente. As tabelas contém o índice da simulação, os níveis dos fatores, os resultados da primeira e da segunda iteração (R1 e R2, respectivamente) e a relação percentual entre os resultados, ou seja, a proporção de R2 em relação a R1, obtido através do cálculo $(R2 * 100) / R1$.

Em todas as tabelas, podem ser observadas um menor efeito na relação percentual nas simulações que utilizaram o nível $-1$ do fator PPI (5\% de percepções inválidas). Isso se dá ao fato do menor impacto do modelo em ambientes de baixo volume de percepções inválidas, uma vez que o modelo apenas age quando há percepções inválidas na fila ponderada.

Na tabela \ref{table:vtaltv1}, a média da relação percentual foi de 75.26\%, se destacando a simulação 5 com uma relação percentual de 4.77\%. Com o tempo de autoplanejamento alto, tempo de processamento de um ciclo de raciocínio baixo e apenas uma percepção por ciclo, é notável o ganho de desempenho. Um comportamento similar (porém invertido) pode ser observado na tabela \ref{table:vtaltv2}, cuja média da relação percentual foi de 136.33\%. Nessas simulações, o esperado era que o tempo virtual subisse conforme a quantidade de planos criados aumentasse, resultando nesse comportamento similar porém invertido, conforma explicado anteriormente. 

\begin{table}
    \begin{center}
        \caption{ Alteração do Tempo Virtual no Grupo I }
        \label{table:vtaltv1}
        \begin{tabular}{ |c|c|c|c|c|c|c|c| }
            \hline
            \textbf{Simulação} & \textbf{PPI} & \textbf{TMC} & \textbf{TMA} & \textbf{NPC} & \textbf{R1} & \textbf{R2} & \textbf{Relação Percentual}\\
            \hline
            1.1 & 5\% & 1 & 64 & 1 & 20750.0 & 20687.0 & 99.70\%\\
            \hline
            1.2 & 5\% & 1 & 64 & 16 & 20813.0 & 20750.0 & 99.70\%\\
            \hline
            1.3 & 5\% & 32 & 64 & 1 & 168032.0 & 167968.0 & 99.96\%\\
            \hline
            1.4 & 5\% & 32 & 64 & 16 & 168032.0 & 168000.0 & 99.98\%\\
            \hline
            1.5 & 95\% & 1 & 64 & 1 & 227579.0 & 10859.0 & 4.77\%\\
            \hline
            1.6 & 95\% & 1 & 64 & 16 & 304313.0 & 177998.0 & 58.49\%\\
            \hline
            1.7 & 95\% & 32 & 64 & 1 & 273536.0 & 162400.0 & 59.37\%\\
            \hline
            1.8 & 95\% & 32 & 64 & 16 & 312032.0 & 250048.0 & 80.14\%\\
            \hline
            
        \end{tabular}{}
    \end{center}{}
\end{table}

\begin{table}
    \begin{center}
        \caption{ Alteração do Tempo Virtual no Grupo II }
        \label{table:vtaltv2}
        \begin{tabular}{ |c|c|c|c|c|c|c|c| }
            \hline
            \textbf{Simulação} & \textbf{PPI} & \textbf{TMC} & \textbf{TMA} & \textbf{NPC} & \textbf{R1} & \textbf{R2} & \textbf{Ganho Percentual}\\
            \hline
            2.1 & 5\% & 1 & 1/2 & 1 & 4874.5 & 4876.5 & 100.04\%\\
            \hline
            2.2 & 5\% & 1 & 1/2 & 16 & 5000.0 & 5000.0 & 100\%\\
            \hline
            2.3 & 5\% & 32 & 1/2 & 1 & 152093.5 & 152219.5 & 100.08\%\\
            \hline
            2.4 & 5\% & 32 & 1/2 & 16 & 153435.5 & 152887.0 & 99.64\%\\
            \hline
            2.5 & 95\% & 1 & 1/2 & 1 & 3228.5 & 4955.0 & 153.48\%\\
            \hline
            2.6 & 95\% & 1 & 1/2 & 16 & 5000.00 & 4944.0 & 98.88\%\\
            \hline
            2.7 & 95\% & 32 & 1/2 & 1 & 48458.5 & 157385.5 & 324.78\%\\
            \hline
            2.8 & 95\% & 32 & 1/2 & 16 & 160000.0 & 140631.0 & 113.77\%\\
            \hline
            
        \end{tabular}{}
    \end{center}{}
\end{table}

Apesar do aumento do tempo virtual das simulação, pode ser usado uma terceira variável, a quantidade de percepções válidas processadas, para analizar o desempenho do modelo. Isso porque de uma simulação para outra, o sistema de autoplanejamento armazena novos planos, e mesmo que o tempo virtual aumente, o agente está sendo capaz de processar mais informação. A média de percepções válidas processadas nas primeiras simulações de cada configuração de fatores foi de 30563.0625 percepções, enquanto nas segundas simulações foi de 40599.75, totalizando um ganho aproximado de 32.84\% de desempenho para essa variável.

As tabelas \ref{table:plansaltv1} e \ref{table:plansaltv2} mostram a variação do ganho de planos criados entre as simulações. Ambas apresentam uma queda drástica nessa variável nas simulações com 95\% de percepções inválidas, pois foram nessas simulações que o modelo pode criar mais planos (pois o autoplanejamento é utilizado mais vezes). As simulações 1.6 e 1.8 possuem um resultado acima das
simulações 1.5 e 1.7, apesar das quatro possuírem PPI de 95\%, pois possuem mais percepções (16 vezes mais, pois são 16 percepções por ciclo) e processam apenas uma percepção inválida por ciclo (pois o TMA é maior que o TMC).
A tabela \ref{table:plansaltv2} em especial possui valores ainda menores na relação percentual pois muitas percepções inválidas são processadas por ciclo, em especial na simulação 2.8, onde nenhum plano novo foi criado na segunda simulação.

\begin{table}
    \begin{center}
        \caption{ Alteração nos Planos Criados no Grupo I }
        \label{table:plansaltv1}
        \begin{tabular}{ |c|c|c|c|c|c|c|c| }
            \hline
            \textbf{Simulação} & \textbf{PPI} & \textbf{TMC} & \textbf{TMA} & \textbf{NPC} & \textbf{R1} & \textbf{R2} & \textbf{Relação Percentual}\\
            \hline
            1.1 & 5\% & 1 & 64 & 1 & 250.0 & 249.0 & 99.60\%\\
            \hline
            1.2 & 5\% & 1 & 64 & 16 & 251.0 & 250.0 & 99.60\%\\
            \hline
            1.3 & 5\% & 32 & 64 & 1 & 251.0 & 249.0 & 99.20\%\\
            \hline
            1.4 & 5\% & 32 & 64 & 16 & 251.0 & 250.0 & 99.60\%\\
            \hline
            1.5 & 95\% & 1 & 64 & 1 & 3533.0 & 93.0 & 2.63\%\\
            \hline
            1.6 & 95\% & 1 & 64 & 16 & 4751.0 & 2746.0 & 57.80\%\\
            \hline
            1.7 & 95\% & 32 & 64 & 1 & 3548.0 & 75.0 & 2.11\%\\
            \hline
            1.8 & 95\% & 32 & 64 & 16 & 4751.0 & 2814.0 & 59.23\%\\
            \hline
            
        \end{tabular}{}
    \end{center}{}
\end{table}


\begin{table}
    \begin{center}
        \caption{ Alteração nos Planos Criados no Grupo II }
        \label{table:plansaltv2}
        \begin{tabular}{ |c|c|c|c|c|c|c|c| }
            \hline
            \textbf{Simulação} & \textbf{PPI} & \textbf{TMC} & \textbf{TMA} & \textbf{NPC} & \textbf{R1} & \textbf{R2} & \textbf{Ganho Percentual}\\
            \hline
            2.1 & 5\% & 1 & 1/2 & 1 & 251.0 & 247.0 & 98.41\%\\
            \hline
            2.2 & 5\% & 1 & 1/2 & 16 & 502.0 & 500.0 & 99.60\%\\
            \hline
            2.3 & 5\% & 32 & 1/2 & 1 & 251.0 & 247.0 & 98.41\%\\
            \hline
            2.4 & 5\% & 32 & 1/2 & 16 & 2935.0 & 878.0 & 29.91\%\\
            \hline
            2.5 & 95\% & 1 & 1/2 & 1 & 3543.0 & 90.0 & 2.54\%\\
            \hline
            2.6 & 95\% & 1 & 1/2 & 16 & 9312.0 & 260.0 & 2.79\%\\
            \hline
            2.7 & 95\% & 32 & 1/2 & 1 & 3541.0 & 83.0 & 2.34\%\\
            \hline
            2.8 & 95\% & 32 & 1/2 & 16 & 4078.0 & 0.0 & 0.0\%\\
            \hline
            
        \end{tabular}{}
    \end{center}{}
\end{table}

\section{Simulação 3}

O objetivo da simulação 3 é observar a capacidade de aprendizado do agente ao longo do tempo. Para isso, foram executadas 5 simulações com os mesmos fatores (como demonstra a tabela \ref{table:experiment3factors}), mas entre uma simulação e outra o agente não foi recarregado, ou seja, manteve o aprendizado das simulações anteriores. As percepções foram geradas de maneira igual aos experimentos anteriores.

\begin{table}[h!]
    \begin{center}
        \caption{ Valor dos fatores no experimento 3. }
        \label{table:experiment3factors}
        \begin{tabular}{ |c|c| }
            \hline
            \textbf{Fator} & \textbf{Valor}\\
            \hline
            PPI & 50\%\\
            \hline
            TMC & 16\\
            \hline
            TMA & 32\\
            \hline
            NPC & 8\\
            \hline
        \end{tabular}{}
    \end{center}{}
\end{table}

Os resultados estão expostos na tabela \ref{table:experiment3results}. Nela, pode-se notar o comportamento dos três fatores observados. Enquanto os planos criados diminuem a cada simulação, pois menos percepções são inválidas (uma vez que o agente aprende com as simulações anteriores), a quantidade de percepções válidas processadas aumenta. Esse comportamento é demonstrado na figura \ref{fig:perceptions_v_plans-experiment3}. Além disso, o tempo virtual cai na mesma proporção que os planos criados diminuem, uma vez que a criação de um plano é duas vezes mais custosa em tempo do que o processamento de uma percepção válida.

\begin{table}
    \begin{center}
        \caption{ Resultados obtidos no experimento 3. }
        \label{table:experiment3results}
        \begin{tabular}{ |c|c|c|c| }
            \hline
            \textbf{Simulação} & \textbf{Planos Criados} & \textbf{Percepções Válidas Processadas} & \textbf{Tempo Virtual}\\
            \hline
            1 & 2501 & 21788 & 120016 \\
            \hline
            2 & 2454 & 30558 & 119264 \\
            \hline
            3 & 946 & 38641 & 95136 \\
            \hline
            4 & 40 & 39959 & 80640 \\
            \hline
            5 & 0 & 40000 & 80000 \\
            \hline
        \end{tabular}{}
    \end{center}{}
\end{table}



\begin{figure}
    \centering
    \includegraphics[width=\textwidth]{Images/perceptions_vs_plans.png}
    \caption{Evolução das percepções válidas processadas e dos planos novos criados ao longo das simulações}
    \label{fig:perceptions_v_plans-experiment3}
\end{figure}
%\chapter{Conclusão}

Neste trabalho, foi apresentado um modelo de revisão de percepções. Tal modelo, inspirado pelos conceitos de alucinação e ilusão, é capaz de reduzir as percepções recebidas, detectar percepções anômalas (que fogem do escopo de trabalho do agente), classificá-las de acordo com suas características como ilusão (do tipo 1 ou 2, conforme definido) ou alucinação e tratá-las, através do mecanismo de autoplanejamento.

O funcionamento do modelo foi descrito no capítulo 3, e sua formalização e implementação apresentadas no capítulo 4. Os experimentos realizados demonstram que o funcionamento do modelo é como esperado, sendo capaz de absorver as anomalias recebidas do ambiente e melhorar o comportamento do agente baseado nelas. Os resultados dos experimentos foram apresentados no capítulo 5. 
Analisando os resultados, é possível observar que o modelo funciona melhor em cenários onde há uma grande quantidade de percepções anômalas, possibilitando uma maior criação de planos novos. Além disso, o maior desafio do modelo é manter o desempenho do agente em cenários onde o tempo gasto pelo módulo de autoplanejamento é muito elevado.

Apesar dos experimentos terem resultados positivos, as simulações estavam limitadas a um cenário sem escopo, ou seja, sem objetivo no mundo real. Por conta disso, quase não haviam fatores aleatórios, tornando a análise extremamente precisa (com erros da ordem de grandeza de $e^{-12}$) mas também bastante limitada.
Portanto, é necessário ainda validar o modelo em mais três etapas: (i) acoplando-o a uma arquitetura cognitiva implementada, para verificar como ela se comporta em uma interação mais complexa com o ambiente; (ii) implementação de experimentos mais complexos, com simulações de alguma situação real que possa ter resultados mais palpáveis; (iii) a aplicação do modelo em agentes físicos, para verificar se ele é capaz de melhorar o desempenho de agentes em ambientes reais.




%% -------------- Elementos Pós-Textuais -----------------%%
\postextual  
\bibliography{bibliografia} % Referências bibliográficas
% %-------------------------------------------------------------
%---------------------- Apêndices ----------------------------
%-------------------------------------------------------------

\begin{apendicesenv}
\partapendices  % Indica o início dos Apendices

\chapter{Diferença entre Anexo e Apêndice}


Os apêndices ``São textos ou documentos elaborados pelo autor, a fim de complementarem sua argumentação, sem prejuízo da unidade nuclear do trabalho'' \cite{NBR14724:2011}. Podem ser incluídos nos apêndices:  os questionários da pesquisas, as tabulação de dados, ilustrações e outros documentos que necessariamente foram preparados pelo autor. Já os anexos, em conformidade com a norma \cite{NBR14724:2011}``são textos ou documentos não elaborados pelo autor, que servem de fundamentação, comprovação ou ilustração à parte do trabalho'', como por exemplo leis, ilustrações, demonstrações de formulas, tabulações de dados de trabalhos referenciados, etc.



\chapter{Formatação}

Os apêndices devem ser identificados por letras maiúsculas consecutivas (APÊNDICE A, APÊNDICE B, etc), travessão e os respectivos títulos, devendo estar
centralizados na folha. 

\end{apendicesenv}

% %----------------------------------------------------------------
%---------------------- Anexos ----------------------------------
%----------------------------------------------------------------

\begin{anexosenv}
\partanexos   % indica o início dos anexos
% ----------------------------------------------------------
\chapter{Os benefícios da IC no aprendizado e formação científica}
% ----------------------------------------------------------

Trabalhar nesse projeto, que durou ao todo dois anos, foi uma grande oportunidade para aprender sobre o funcionamento da pesquisa científica, desenvolvendo diversas capacidades multidisciplinares. No projeto de 2019/2020, especialmente, os enfoques na implementação do projeto e na análise estatística dos resultados trouxe uma grande bagagem para minha formação acadêmica, principalmente na programação em Python, uso das bibliotecas gráficas para geração de gráficos e os conceitos estatísticos usados no design de experimentos 2k fatorial.


\end{anexosenv}

\phantompart  \printindex  % Índice Remissivo
% ----------------------------------------------------------
\end{document}  % fim do documento
