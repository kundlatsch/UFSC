%--------------------------------------------------------------------------
%--------------------- Resumo em Português --------------------------------
%--------------------------------------------------------------------------

\setlength{\absparsep}{18pt} % ajusta o espaçamento dos parágrafos do resumo
\begin{resumo}
Percepções são a forma mais simples de uma entidade se comunicar com o ambiente. Cada pessoa possui uma maneira diferente de perceber e interpretar o mundo. Entretanto, sabe-se que na percepção humana existem ilusões e alucinações, sendo que a primeira são percepções de objetos presentes no mundo mas com características inadequadas ou características corretas em objetos inadequados, e a segunda são percepções falsas de coisas reais. Dito isso, como podemos saber se nossas percepções são reais ou se são apenas fruto de nossa imaginação? E a questão derivada disso é: e computadores? Agentes possuem diversos sensores para reconhecerem o mundo a sua volta, e esses sensores podem falhar. Nesse trabalho, apresentamos um modelo genérico de revisão de percepções, capaz de tratar de percepções inválidas recebidas pelo agente, e criar novos planos para se adaptar ao ambiente.

 \vspace{\onelineskip}
 \noindent
 \textbf{Palavras-chave}: Agentes. Percepção. Ilusão. Alucinação.

\end{resumo}

%--------------------------------------------------------------------------
%--------------------- Resumo em Inglês --------------------------------
%--------------------------------------------------------------------------
\iffalse
\begin{resumo}[Abstract]
 \begin{otherlanguage*}{english}
   This is the english abstract.


   \vspace{\onelineskip}
   \noindent 
   \textbf{Keywords}: Keywords1, Keywords2, Keywords3.
 \end{otherlanguage*}
\end{resumo}
\fi