\section*{Proposta de Trabalho}

O algoritmo que foi sorteado para o trabalho, e que será seguido, é o Classic McEliece. A principal fonte para o desenvolvimento das demais entregas será o \emph{website} do algoritmo \cite{ClassicMcEliece}, que armazena as três submissões feitas ao concurso do NIST, com o código e os documentos (guias e especificações) atrelados.

Para as demais entregas, as propostas de trabalho são as seguintes:

\begin{itemize}
    \item[Doc2] Respostas para as oito questões requisitadas para a entrega, utilizando os documentos fornecidos pelos autores nas submissões ao NIST. Será dado um enfoque nos parâmetros que o algoritmo pode receber, uma vez que essa é a principal maneira de garantir que ele tenha robustes em um cenário de criptografia pós-quântica, e nos ataques clássicos utilizados contra o McEliece, pois é um sistema bastante antigo e diversos métodos de ataque já foram propostos.
    
    \item[Doc3] Texto explicando o funcionamento do algoritmo, utilizando a ferramenta \textit{libmcleece}, que implementa os algoritmos de geração de chave, criptografar e descriptografar do Classic McEliece. Os principais pontos do código serão utilizados para explicar o funcionamento dos algoritmos.
    
    \item[Vídeo] Produção de um vídeo expositivo com duração de 10 a 20 minutos seguindo as explicações, exemplos e demonstração do Doc3.
\end{itemize}