\section{Introdução}

O Classic McEliece é um sistema criptográfico de criptografia assimétrica baseado em códigos lineares de correção de erros, que inclui geração de chaves, criptografia e descriptografia de mensagens. Nesse trabalho, vamos apresentar na prática como ele funciona, utilizando a ferramenta libmcleece disponível em \href{https://github.com/sz3/libmcleece}{sua página no github}.

\section{\texttt{libmcleece}}

A libmcleece foi desenvolvida em C++, e implementa uma interface de linha de comando para a implementação original do Classic McEliece feita para sua sumbissão no concurso de algoritmos de criptografia de chave-pública pós-quânticos do NIST, disponível no site oficial do algoritmo \cite{ClassicMcEliece}.

Para instalar a libmcleece, basta cloná-la do github, instalar as dependências contidas no arquivo \texttt{README.md} e executar os comandos:

\begin{minted}{shell}

    cmake .
    make -j4 install

\end{minted}

O resultado do build estará disponível, por padrão, na basta \texttt{dist/}, e conterá um executável \texttt{mcleececli}, que será utilizado nos experimentos que serão realizados.

\section{Sistema criptográfico}

Nessa seção vamos descrever como funciona cada uma das etapas de criptografia do Classic McEliece: a geração de chave, criptografia e descriptografia de mensagens. Para exemplificar, vamos utilizar os personagens tradicionais de exemplos de criptografia Alice e Bob. Primeiro, Bob vai gerar uma chave e Alice vai encriptar uma mensagem utilizando a chave pública de Bob. Dessa maneira, apenas ele será capaz de descriptografar a mensagem.

Para facilitar o entendimento, primeiro serão descritos cada um dos processos e como eles foram implementados na submissão para o NIST, e depois será apresentado um exemplo prático com parâmetros pequenos, para fins didáticos.

\subsection{Geração de Chave}

Para Bob gerar um par de chaves, os seguintes passos são executados:

\begin{enumerate}
    \item Primeiro, ele deve gerar um polinômio irredutível mônico aleatório $g(z) \in {F}_{2^m}[z]$ de grau $t$, sendo ${F}_{2^m}$ um campo de extensão binária (um elemento da aritmética de campos finitos).
    \item Seleciona um conjunto uniformemente aleatório $\{\alpha_1, \alpha_2, ..., \alpha_n\} \subseteq {F}_{2^m}$ com todos os seus elementos sendo distintos.
    \item Computar uma matriz de tamanho $t \times n$ definida como $\tilde{H} = \{h_{i, j}\}$ sobre ${F}_{2^m}$, onde $ \{h_{i, j}\} = a_{j}^{i-1}g(\alpha_i)^{-1}$ para $i = 1, 2, ..., t$ e $j = 1, 2, ..., n$.
    \item Substituir cada entrada da matriz $\tilde{H}$ (elementos de ${F}_{2^m}$) com vetores de ${F}_{2}^m$ (arranjados em colunas) usando o isomorfismo de espaço vetorial entre ${F}_{2^m}$ e ${F}_{2^m}$ para chegar a uma matriz $\hat{H}$ de tamanho $mt \times n$.
    \item Aplicar eliminação gaussiana em  $\hat{H}$ para obter uma matriz sistemática $H = (I_{mt} | T_{mt \times (n-mt)})$ se possível, caso contrário deve retornar para a etapa 1.
    \item Gerar uma cadeia uniforme de bits aleatórios de tamanho $n$ nomeada $s$.
    \item \textbf{Chave Pública:} $T$.
    \item \textbf{Chave Privada:} $\{s, g(z), \alpha_1, \alpha_2, ..., \alpha_n\}$.
    
\end{enumerate}

Apesar de ser um pouco difícil de entender, pois utiliza notação matemática pesada e aritmética de campos finitos, a ideia da geração é criar, a partir de um código linear de correção de erros, uma matriz geradora para do código selecionado, e transformar essa matriz em um polinômio.

A implementação da geração de chave privada proposta na submissão do McEliece é a seguinte:


\iffalse
A ideia por trás da geração de chave do McEliece é escolher um código linear $C$ e tornar sua matriz geradora pública para que qualquer um possa usá-lo para codificar mensagens, mas manter o algoritmo de decodificação secreto (chave pública e privada, respectivamente). Conhecer $C$ apenas não é suficiente para quebrar o segredo do algoritmo, pois é necessário saber os parâmtros utilizados na geração.

Um exemplo disso são códigos Goppa binários, que são a versão dos códigos Goppa descritos originalmente por Valerii Denisovich Goppa \cite{berlekamp1973goppa} (códigos lineares gerados a partir de uma curva sobre um campo finito). Para decodificar um códigos Goppa binários, é necessário saber o polinômio Goppa utilizado e os localizadores de código. Assim, é possível publicar uma matriz geradora de $C$ publicamente ofuscando essas informações.

O passo a passo para Alice gerar uma chave seria o seguinte:

\begin{enumerate}
    \item Alice seleciona um código binário $C$ de dimensão $n$ por $k$ capaz de corrigir (eficientemente) um número definido de erros $t$, como um código Goppa binário, descrito anteriormente. Esta escolha deve dar origem a um algoritmo de decodificação eficiente $A$. A partir dessas escolhas, uma matrix geradora $G$ deve ser escolhida (cada código linear possui diversas matrizes geradores, por isso essa escolha precisa ser feita).
    
    \item Alice seleciona uma matriz binária não singular $k \times k$ aleatória chamada $S$.
    
    \item Alice seleciona uma matriz de permutação $n \times n$ aleatória chamada $P$.
    
    \item Alice computa a matriz $\hat{G} = SGP$ de dimensão  $k \times n$.
    
    \item A chave pública de Alice será ($\hat{G}$, $t$), e sua chave privada ($S$, $P$, $A$).
    
\end{enumerate}



\fi

\begin{minted}{cpp}
int genpoly_gen(gf *out, gf *f)
{
  int i, j, k, c;

  gf mat[ SYS_T+1 ][ SYS_T ];
  gf mask, inv, t;

  // fill matrix

  mat[0][0] = 1;

  for (i = 1; i < SYS_T; i++)
    mat[0][i] = 0;

  for (i = 0; i < SYS_T; i++)
    mat[1][i] = f[i];

  for (j = 2; j <= SYS_T; j++)
    GF_mul(mat[j], mat[j-1], f);

  // gaussian

  for (j = 0; j < SYS_T; j++)
  {
    for (k = j + 1; k < SYS_T; k++)
    {
      mask = gf_iszero(mat[ j ][ j ]);

      for (c = j; c < SYS_T + 1; c++)
        mat[ c ][ j ] ^= mat[ c ][ k ] & mask;

    }

    if ( mat[ j ][ j ] == 0 ) // return if not systematic
    {
      return -1;
    }

    inv = gf_inv(mat[j][j]);

    for (c = j; c < SYS_T + 1; c++)
      mat[ c ][ j ] = gf_mul(mat[ c ][ j ], inv) ;

    for (k = 0; k < SYS_T; k++)
    {
      if (k != j)
      {
        t = mat[ j ][ k ];

        for (c = j; c < SYS_T + 1; c++)
          mat[ c ][ k ] ^= gf_mul(mat[ c ][ j ], t);
      }
    }
  }

  for (i = 0; i < SYS_T; i++)
    out[i] = mat[ SYS_T ][ i ];

  return 0;
}
\end{minted}

Nessa implementação, é possível perceber o preenchimento da matriz e a eliminação gaussiana. Para realizar esses procedimentos, é utilizada uma estrutura de dados \texttt{gf} que implementa funções para aritmética de campo.

A partir da geração da chave privada, a chave pública é gerada da seguinte maneira:

\begin{minted}{cpp}
int pk_gen(unsigned char * pk, unsigned char * sk,
uint32_t * perm, int16_t * pi)
{
  int i, j, k;
  int row, c;

  uint64_t buf[ 1 << GFBITS ];

  unsigned char mat[ PK_NROWS ][ SYS_N/8 ];
  unsigned char mask;
  unsigned char b;

  gf g[ SYS_T+1 ]; // Goppa polynomial
  gf L[ SYS_N ]; // support
  gf inv[ SYS_N ];

  //

  g[ SYS_T ] = 1;

  for (i = 0; i < SYS_T; i++) { g[i] = load_gf(sk); sk += 2; }

  for (i = 0; i < (1 << GFBITS); i++)
  {
    buf[i] = perm[i];
    buf[i] <<= 31;
    buf[i] |= i;
  }

  uint64_sort(buf, 1 << GFBITS);

  for (i = 1; i < (1 << GFBITS); i++)
    if ((buf[i-1] >> 31) == (buf[i] >> 31))
      return -1;

  for (i = 0; i < (1 << GFBITS); i++) pi[i] = buf[i] & GFMASK;
  for (i = 0; i < SYS_N;         i++) L[i] = bitrev(pi[i]);

  // filling the matrix

  root(inv, g, L);
    
  for (i = 0; i < SYS_N; i++)
    inv[i] = gf_inv(inv[i]);

  for (i = 0; i < PK_NROWS; i++)
  for (j = 0; j < SYS_N/8; j++)
    mat[i][j] = 0;

  for (i = 0; i < SYS_T; i++)
  {
    for (j = 0; j < SYS_N; j+=8)
    for (k = 0; k < GFBITS;  k++)
    {
      b  = (inv[j+7] >> k) & 1; b <<= 1;
      b |= (inv[j+6] >> k) & 1; b <<= 1;
      b |= (inv[j+5] >> k) & 1; b <<= 1;
      b |= (inv[j+4] >> k) & 1; b <<= 1;
      b |= (inv[j+3] >> k) & 1; b <<= 1;
      b |= (inv[j+2] >> k) & 1; b <<= 1;
      b |= (inv[j+1] >> k) & 1; b <<= 1;
      b |= (inv[j+0] >> k) & 1;

      mat[ i*GFBITS + k ][ j/8 ] = b;
    }

    for (j = 0; j < SYS_N; j++)
      inv[j] = gf_mul(inv[j], L[j]);

  }

  // gaussian elimination

  for (i = 0; i < (PK_NROWS + 7) / 8; i++)
  for (j = 0; j < 8; j++)
  {
    row = i*8 + j;      

    if (row >= PK_NROWS)
      break;

    for (k = row + 1; k < PK_NROWS; k++)
    {
      mask = mat[ row ][ i ] ^ mat[ k ][ i ];
      mask >>= j;
      mask &= 1;
      mask = -mask;

      for (c = 0; c < SYS_N/8; c++)
        mat[ row ][ c ] ^= mat[ k ][ c ] & mask;
    }

    if ( ((mat[ row ][ i ] >> j) & 1) == 0 ) 
    // return if not systematic
    {
      return -1;
    }

    for (k = 0; k < PK_NROWS; k++)
    {
      if (k != row)
      {
        mask = mat[ k ][ i ] >> j;
        mask &= 1;
        mask = -mask;

        for (c = 0; c < SYS_N/8; c++)
          mat[ k ][ c ] ^= mat[ row ][ c ] & mask;
      }
    }
  }

  for (i = 0; i < PK_NROWS; i++)
    memcpy(pk + i*PK_ROW_BYTES, mat[i] + PK_NROWS/8, PK_ROW_BYTES);

  return 0;
}
\end{minted}

\subsection{Encriptação}

Para Alice realizar o processo de encriptação, ela deve seguir as seguintes etapas:

\begin{itemize}
    \item Gera um vetor aleatório uniforme  $e \in {F}_{2}^n$ com peso de Hamming $t$.
    
    \item \textbf{Codificação Niederreiter:} Usando a chave pública $T$ de Bob, Alice deve primeiro computar a matriz $H$, e depois o vetor $C_0 = He = (I | T)e$ de comprimento $mt$ em ${F}_{2}^mt$.
    
    \item Computar $C_1 = H(2, e)$ e gerar o texto cifrado $C = (C_0, C_1)$ de comprimento $mt + 256$ bits.
    
    \item Computar uma sessão de chave de 256 bits $K = H(1, e, C)$.
\end{itemize}

Na implementação de referência, o vetor $e$ é gerado com o seguinte algoritmo abaixo, que gera valores aleatórios continuamente até encontrar um vetor onde não haja repetição, então realiza as operações necessárias.

\begin{minted}{C++}
static void gen_e(unsigned char *e)
{
  int i, j, eq;

  uint16_t ind[ SYS_T ];
  unsigned char bytes[ sizeof(ind) ];
  unsigned char mask;  
  unsigned char val[ SYS_T ];  

  while (1)
  {
    randombytes(bytes, sizeof(bytes));

    for (i = 0; i < SYS_T; i++)
      ind[i] = load_gf(bytes + i*2);

    // check for repetition

    eq = 0;

    for (i = 1; i < SYS_T; i++) 
      for (j = 0; j < i; j++)
        if (ind[i] == ind[j]) 
          eq = 1;

    if (eq == 0)
      break;
  }

  for (j = 0; j < SYS_T; j++)
    val[j] = 1 << (ind[j] & 7);

  for (i = 0; i < SYS_N/8; i++) 
  {
    e[i] = 0;

    for (j = 0; j < SYS_T; j++)
    {
      mask = same_mask(i, (ind[j] >> 3));

      e[i] |= val[j] & mask;
    }
  }
}
\end{minted}

Na implementação da codificação de Niederriter, é importante notar que esse algoritmo usa uma síndrome como texto cifrado e a mensagem é um padrão de erro. Isso pode ser visto no trecho de código abaixo:

\begin{minted}{cpp}
static void syndrome(unsigned char *s, 
const unsigned char *pk, unsigned char *e)
{
  unsigned char b, row[SYS_N/8];
  const unsigned char *pk_ptr = pk;

  int i, j;

  for (i = 0; i < SYND_BYTES; i++)
    s[i] = 0;

  for (i = 0; i < PK_NROWS; i++)  
  {
    for (j = 0; j < SYS_N/8; j++) 
      row[j] = 0;

    for (j = 0; j < PK_ROW_BYTES; j++) 
      row[ SYS_N/8 - PK_ROW_BYTES + j ] = pk_ptr[j];

    row[i/8] |= 1 << (i%8);
    
    b = 0;
    for (j = 0; j < SYS_N/8; j++)
      b ^= row[j] & e[j];

    b ^= b >> 4;
    b ^= b >> 2;
    b ^= b >> 1;
    b &= 1;

    s[ i/8 ] |= (b << (i%8));

    pk_ptr += PK_ROW_BYTES;
  }
}
\end{minted}

\subsection{Decriptação}

Para Bob decriptar a mensagem encriptografada enviada por Alice, os seguintes passos devem ser seguidos:

\begin{enumerate}
    \item Primeiro, ele deve dividir $C$ como $(C_0, C_1)$, com $C_0 \in {F}_{2}^{mt}$ e $C_1 \in {F}_{2}^{256}$.
    
    \item Atribuir $b \leftarrow 1$.
    
    \item Decodificação:
        \begin{enumerate}
            \item \texttt{Entrada:} $C_0$ e a chava privada $\{s, g(z), \alpha_1, \alpha_2, ..., \alpha_n\}$.
            \item Estender $C_0$ para $v = (C_0, 0, ..., 0 \in  {F}_{2}^{mt}$ inderindo nele $n - mt$ zeros.
            \item Através da decodificação de Niederreiter, achar a palavra-código $c$ no código Goppa definido por $\Gamma = \{g(z), \alpha_1, \alpha_2, ..., \alpha_n\}$.
            \item Caso a palavra-código $c$ exista, atribuir o vetor $e = v + c$. Se $wt(e) = t$ e $C_0 = He$, retorna $e$. Caso contrário, retorna $\perp$.
        \end{enumerate}
    
    \item Se a decodificação retornou $\perp$, atribuir $e \leftarrow s$ e $b \leftarrow 0$.
    
    \item Computar $C_1' = H(2, e)$, e verificar se $C_1' = C_1$. Se não for, definir $e \leftarrow s$ e $b \leftarrow 0$.
    
    \item Computar a sessão de chave $K' = H(b, e, C)$.
\end{enumerate}

Se não houver falha em nenhuma das etapas durante o processo de decriptação e $C_1' = C_1$, então certamente a sessão de chave $K'$ será idêntica a $K$. De maneira equivalente, se Bob receber um texto cifrado válido $C$, ou seja, $C = (C_0, C_1)$ com $C_0 = He$ para algum $e \in {F}_{2}^{n}$ de peso $t$ e $C_1 = H(2, e)$, a decodificação sempre vai resultar no vetor $e$.

Na implementação de referência, foi implementado a decriptação de Niederreiter com o decodificador de Berlekamp, um algoritmo decodificador de correção de erros. O seguinte trecho de código recebe a chave privada e o texto crifrado, e retorna o vetor de erro $e$, conforme descrito anteriormente.

\begin{minted}{cpp}
int decrypt(unsigned char *e, 
const unsigned char *sk, const unsigned char *c)
{
  int i, w = 0; 
  uint16_t check;  

  unsigned char r[ SYS_N/8 ];

  gf g[ SYS_T+1 ];
  gf L[ SYS_N ];

  gf s[ SYS_T*2 ];
  gf s_cmp[ SYS_T*2 ];
  gf locator[ SYS_T+1 ];
  gf images[ SYS_N ];

  gf t;

  //

  for (i = 0; i < SYND_BYTES; i++)       r[i] = c[i];
  for (i = SYND_BYTES; i < SYS_N/8; i++) r[i] = 0;

  for (i = 0; i < SYS_T; i++) 
  { g[i] = load_gf(sk); sk += 2; } g[ SYS_T ] = 1;

  support_gen(L, sk);

  synd(s, g, L, r);

  bm(locator, s);

  root(images, locator, L);

  //
  
  for (i = 0; i < SYS_N/8; i++) 
    e[i] = 0;

  for (i = 0; i < SYS_N; i++)
  {
    t = gf_iszero(images[i]) & 1;

    e[ i/8 ] |= t << (i%8);
    w += t;

  }

#ifdef KAT
  {
    int k;
    printf("decrypt e: positions");
    for (k = 0;k < SYS_N;++k)
      if (e[k/8] & (1 << (k&7)))
        printf(" %d",k);
    printf("\n");
  }
#endif
  
  synd(s_cmp, g, L, e);

  //

  check = w;
  check ^= SYS_T;

  for (i = 0; i < SYS_T*2; i++)
    check |= s[i] ^ s_cmp[i]; 

  check -= 1;
  check >>= 15;

  return check ^ 1;
}
\end{minted}

\section{Aplicação Prática}

É possível utilizar essas três funções da implementação de referência através da libmcleece. Primeiro, para gerar uma chave, basta executar o binário gerado com o build com os seguintes parâmetros:

\begin{minted}{bash}
  mcleececli generate-keypair --key-path=/tmp/key
\end{minted}

Com isso, a chave será guardada em um arquivo apontado pelo \texttt{--key-path}. Com a chave gerada, podemos encriptar qualquer arquivo através da ferramenta de linha de comando:

\begin{minted}{bash}
  mcleececli encrypt /caminho/arquivo.ext 
    --key-path=/tmp/key.pk > encriptado.bin
\end{minted}

Por fim, qualquer pessoa que possua o arquivo encriptado e a chave privada, pode decriptar o arquivo com o comando:

\begin{minted}{cpp}
  mcleececli decrypt encriptado.bin 
    --key-path=/tmp/key.sk > decriptado
\end{minted}

\section{Conclusão}

Apesar da definição formal do algoritmo Classic McEliece ser bastante complexa, utilizando aritmética de campos finitos e bastante notação matemática, os princípios empregados são relativamente simples: utilizar a complexidade computacional de decodificar códigos lineares de correção de erros para encriptar as mensagens. Isso, em conjunto com uma ferramenta que transporta a implementação de referência para uma ferramenta de interface de linha de comando como o libmcleece, torna o uso do Classic McEliece simples e palpável para aplicações práticas.