Esse trabalho mostrou-se um desafio para o grupo, tanto pela curva de aprendizado que o software Blender apresenta como pela aptidão "artística" dos integrantes. 

Em um primeiro momento, foram necessárias algumas horas de prática e de vídeos assistidos; muitas das vezes o programa não é tão intuitivo, assim como seus atalhos. Entretanto, com o passar do tempo tornou-se mais simples chegar ao produto esperado. Após dominar algumas das principais ferramentas de modelagem, como loop cut e extrusão, foi bastante intuitivo o processo necessário para replicar as estruturas necessárias. A diferença entre os objetos modelados no começo (casco e grades do convés, por exemplo) é bastante notável em relação aos objetos modelados após algumas horas trabalhando no navio (como as janelas).

Já em relação à modelagem em si, um dos principais desafios foi (tentar) manter as proporções apresentadas tanto na planta como nas imagens de exemplo.

Abaixo, segue uma lista com alguns dos principais tutoriais e vídeo aulas seguidos durante a modelagem:

\begin{itemize}
    \item Tutorial de como modelar um navio pirata (usado para fazer o casco):\\ https://www.youtube.com/watch?v=7TjmeA7A0HM
    \item Como dar expessura para os objetos: \\https://www.youtube.com/watch?v=yIV59mmIF8c
    \item Como criar buracos para janelas e portas: \\https://www.youtube.com/watch?v=kZD8sSX7NtU
    \item Como utilizar arrays: \\https://www.youtube.com/watch?v=S2Mw2lbMas0
    \item Tutorial de texturas: \\https://www.youtube.com/watch?v=XI-pZshRp8g
    \item Com utilizar múltiplas texturas em um objeto: \\https://www.youtube.com/watch?v=ms89wP8m4ZU
    \item Como criar textura de vidro: \\https://www.youtube.com/watch?v=SN9lkS7K04k
    \item Navegação da câmera I: \\https://www.youtube.com/watch?v=1oD3gSX3ICM
    \item Navegação da câmera II: \\https://www.youtube.com/watch?v=mkD7S7wLx1I
\end{itemize}