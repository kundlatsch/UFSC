Este relatório apresenta uma visão geral da experiência da execução do Trabalho 4. O sofwatere Blender foi utilizado para realizar a união do modelo hierárquico com os arquivos de MoCap disponibilizados. Para o desenvolvimento do jogo, foi utilizado Godot Engine.

Foi desenvolvido um jogo de labirinto, onde é necessário coletar duas vacinas para conseguir escapar. A movimentação do personagem é feita com as teclas WASD. Para animação e movimentação de pulo, a tecla B; para animação de cambalhota, tecla C. A movimentação da câmera é realizada a partir do mouse. Detalhes do desenvolvimento e execução são apresentados nos vídeos.

Foram produzidos três vídeos de apresentação, hospedados no Youtube, em que é possível visualizar o resultado final. O primeiro apresenta o que foi realizado no Blender. O segundo apresenta uma visão geral do que foi desenvolvido no Godot Engine. O terceiro consiste no jogo sendo executado. Além dos arquivos de vídeo, disponibilizamos no Google Drive os arquivos fontes referentes ao projeto: arquivo de modelagem do Blender, texturas utilizadas e projeto do Godot. O jogo foi exportado para Linux (\texttt{ine5420\_t4.x86\_64}), Windows (\texttt{ine5420\_t4.exe}) e HTML5, todos disponíveis na pasta \texttt{game/exec}.

\begin{itemize}
    \item Vídeo 1: \\
    \url{https://www.youtube.com/watch?v=aBP6V5frjW4}
    \item Vídeo 2: \\
    \url{https://www.youtube.com/watch?v=_5n6FwW0v5Q}
    \item Vídeo 3: \\
    \url{https://www.youtube.com/watch?v=bwAlkCxctp0}
    \item Arquivos fonte: \\
    \url{https://drive.google.com/drive/folders/1uZFP0W9izQKMXGewkxJBSbMcILs-5a6f?usp=sharing}
\end{itemize}

