%%%%%%%%%%%%%%%%%%%%%%%%%%%%%%%%%%%%%%%%%%%%%%%%%%%%%%%%%%%%%%%%%%%%%%
% How to use writeLaTeX: 
%
% You edit the source code here on the left, and the preview on the
% right shows you the result withi.n a few seconds.
%
% Bookmark this page and share the URL with your co-authors. They can
% edit at the same time!
%
% You can upload figures, bibliographies, custom classes and
% styles using the files menu.
%
%%%%%%%%%%%%%%%%%%%%%%%%%%%%%%%%%%%%%%%%%%%%%%%%%%%%%%%%%%%%%%%%%%%%%%

\documentclass[12pt]{article}

\usepackage{sbc-template}

\usepackage{graphicx,url}
\usepackage{amsthm}

%\usepackage[brazil]{babel}   
\usepackage[utf8]{inputenc}  

     
\sloppy

\title{Prova da NP Completude de
\\The Legend of Zelda: A Link to the Past}

\author{Gustavo Kundlatsch e Gustavo Raimundo}


\address{Departamento de Informática e Estatística (INE)
\\Universidade Federal de Santa Catarina(UFSC)}

\begin{document} 

\maketitle

\begin{abstract}
  This paper has the objective to prove the NP completude of the game The Legend of Zelda: a Link to The Past,
  a classic game of the serie, released for the Super Nintendo Entertainment System (SNES).
  That is, if the game was played by an algorithm, the complexity to solve certain problems
  during it's run would be NP-Complete, so the game as a whole is NP-Complete. For this, we 
  will use a proof framework, used to make similar proofs to other games of the same company.
\end{abstract}
     
\begin{resumo} 
  Este artigo tem como objetivo provar a NP completude do jogo The Legend Of Zelda: a Link to The Past,
  um dos clássicos da franquia, lançado para o Super Nintendo Entertainment System (SNES). 
  Isto é, caso fosse jogado por um algoritmo, a complexidade
  para resolver certos problemas durante seu decorrer são NP-Completo, e assim,
  o jogo como um todo pode ser dito NP-Completo. Para isso, utilizaremos um framework
  que é utilizado para realizar provas semelhantes em diversos jogos da mesma empresa.
\end{resumo}

\section{Introdução}
\subsection{Motivação}
	Internet das Coisas (ou IoT, sigla em inglês para Internet of Things) é não só um tópico que nos últimos anos vem ganho destaque na mídia como tem também ganho os holofotes dos especialistas em tecnologia. IoT é um conceito ligado ao crescimento da interconexão digital de objetos cotidianos com a internet, como eletrodomésticos, microcontroladores e variedades ainda mais simples, como lâmpadas e tomadas. Em resumo, é a conexão dos objetos, mais do que das pessoas, a internet. A interconexão dos sistemas permite um ambiente diário totalmente inteligente e eficiente. Portanto, um dos maiores impactos da IoT é, a partir da interpretação de dados recebidos pelos sensores dos dispositivos, a capacidade dos objetos de se comunicar com os usuários, promovendo melhorias na qualidade de vida, maior produtividade e agilidade nos processos. Além do impacto imediato na mudança do estilo de vida dos usuários, essa nova onda de produtividade permitirá que as pessoas foquem naquilo que não pode ser automatizado.
	Essa tecnologia tem ganhado espaço no vida das pessoas \cite{fan2018blockchain}. No entanto, essas tecnologias também representam graves riscos de privacidade e preocupações com a segurança dos dados \cite{dwivedi2019decentralized}. Suponha uma loja em que todos os sistemas estejam conectados, desde os as lâmpadas e a rede elétrica até os dados de compra e vendas do lugar. Sendo assim, se um hacker encontrar uma vulnerabilidade em algum desses sistemas e conseguir invadir a tomada, por exemplo, o mesmo poderá usá-la de porta de entrada para acessar outros sistemas dentro da loja. Portanto, garantir a segurança desse tipo de rede é algo essencial.


\subsection{Justificativas}
	A crescente no uso de dispositivos conectados a rede faz com que os fabricantes de IoT se concentraram na funcionalidade e nos recursos dos mesmos e que a segurança se tonasse algo secundário \cite{falco2019neuromesh}. Como dispositivos desse tipo geralmente possuem um processamento reduzido para um baixo consumo de energia, assim como possuem uma memória pequena para manter o custo barato, a tarefa de torná-los seguros é bastante árdua, e é um desafio ainda em aberto para a academia. Nos últimos anos, abordagens que utilizam blockchain tem se tornado uma boa alternativa para resolver esse problema, e diversas pesquisas tem surgido na área, mas apesar de fornecerem segurança e privacidade descentralizadas, envolvem energia significativa, atraso e sobrecarga computacional que é não é adequado para a maioria dos dispositivos IoT com recursos limitados \cite{dorri2017blockchain}. Então, apesar de blockchain ser uma alternativa bastante interessante para ser tomada, é preciso reduzir o custo computacional e elétrico envolvido para ter um custo benefício que valha a pena. A urgência da necessidade de modelos seguros de IoT pode ser sentida quando analisarmos os grandes players que estão no mercado de dispositivos para casa, como Google, Amazon e Apple, que já possuem milhares de casas com aparelhos inteligentes, como lâmpadas, assistentes pessoais, janelas automatizadas, robôs para limpeza do chão e diversos outros dispositivos que precisam ser seguros, caso contrário ataques de hackers podem deixar em perigo toda a infraestrutura das casas das pessoas que os utilizam. Proteger a rede IoT é um dos objetivos importantes de projetar novas arquiteturas distribuídas \cite{sharma2018software}.

\subsection{Objetivos}
\subsubsection{Gerais}
	O aumento acentuado dos aplicativos da Internet das Coisas (IoT), requer soluções robustas para o problema da violabilidade de seus dados. Por tanto, é preciso utilizar métodos computacionais que permitem tal robustês, como blockchain, para garantir a segurança não só da vida online de seus proprietários como até a segurança física que pode ser comprometida dependendo do tipo de aparelhos instalados em sua casa \cite{dorri2017towards}. Portanto o principal objetivo é garantir a segurança dos dispositivos IoT utilizando blockchain.

\subsubsection{Específicos}
\begin{itemize}
\item Compreender Internet das Coisas e blockchain;
\item Obter uma compreensão abrangente sobre como blockchain pode ser usado para garantir a segurança em IoT;
\item Fornecer uma visão geral sobre segurança em IoT em geral;
\item Identificar abordagens existentes, seus principais casos de uso, identificação dos principais problemas, e encaminhar possíveis soluções.
\end{itemize}

\subsection{Organização do Artigo}
	A seção 2 apresenta os conceitos básicos de internet das coisas, segurança e blockchain. Na seção 3 são apresentados a revisão bibliográfica sistemática, com a tabela de buscas, e os trabalhos correlatos, exemplificando a situação do estado da arte no uso de blockchain como solução para segurança em redes de aparelhos de internet das coisas. Na seção 4 são expostos os aspectos relevantes que remetem tanto ao problema quanto a solução, contendo uma discussão sobre as blockchains que já operam hoje em dia. Na seção 5 discutimos os problemas existentes no uso de blockchain para soluções de segurança em IoT, explorando aspectos de redes blockchain e sua aplicação no mercado. Na seção 6 apresentamos possíveis soluções, utilizando como base um caso de estudo de uma casa inteligente que utiliza redes blockchain para manter a segurança do sistema. Na seção 7, abstraímos o exemplo da seção 6 para um modelo teórico, que pode ser utilizado para modelar outras redes de internet das coisas, e mostramos resultados práticos e indicativos que colaboram para validar o modelo proposto. Por fim na seção 8 temos a conclusão e sugestões de trabalhos futuros a serem desenvolvidos dentro do tema do artigo.
	

\section{Apresentação dos Problemas}

Como a prova apresentada nesse artigo utiliza um framework, ela pode se
tornar um pouco abstrata demais. Nessa sessão apresentamos os problemas envolvidos e exemplos de cada um.

\subsection{The Legend of Zelda: a Link to The Past}

The Legend of Zelda: a Link to The Past é um jogo desenvolvido para Super Nintendo 
(que a partir desse momento será chamado apenas de Zelda, para facilitar a leitura), publicado pela
primeira vez em 1991, no Japão. Nele, controlamos o protagonista Link, um garoto que tem como objetivo
salvar a terra de Hyrule, impedir que o mago Ganon se torne o novo governante e salvar a princesa Zelda.
Para isso, Link deve obter a Triforce, uma relíquia sagrada espalhada pelo mapa, que é composta por 
três triângulos equiláteros, que trazem ao portador sabedoria, coragem e poder.

Em termos de design, o jogo é um mapa aberto, em que o jogador pode andar livremente. A câmera é posicionada acima
do plano, dando visão de uma parte do mapa total, sendo que quando o jogador se desloca para fora desse enquadramento
a câmera se desloca para exibir uma nova parte do mapa. O jogo é uma mistura de combates e puzzles. Para avançar
no jogo, é preciso explorar cavernas e construções, conversar com personagens e derrotar inimigos, assim obtendo
novos itens e habilidades. Os itens podem ser achados dentro de baús, sendo que existem itens especiais que quando são
encontrados seus baús permanecem abertos. Itens comuns podem voltam a ter seus baús fechados quando o jogador se move para
outro pedaço do mapa e retorna ao pedaço original (inimigos têm o mesmo comportamento, ou seja, a não ser que sejam chefes
especiais, os inimigos retornam ao mapa após morrerem caso haja essa troca de tela. Chefes especiais também liberam itens especiais).
Como Zelda é um jogo de mapa aberto, qualquer inimigo pode ser enfrentado a qualquer momento, desde que seja possível chegar até ele, dando grande liberdade ao jogador.

Tais itens e habilidades não são em si importantes para a prova que iremos realizar,
mas certamente poderiam ser utilizados para provar a NP-Completude por outros caminhos, como por exemplo realizando
uma redução ao problema 1-Push 2D \cite{demaine2000pushpush}.

\begin{figure}[!htb]
     \centering
     \includegraphics[scale=0.3]{zelda.jpg}
     \includegraphics[scale=0.3]{link.jpg}
     \caption{Imagens do jogo}
\end{figure}

Diversos elementos do jogo podem ser utilizados para provar sua NP-Completude. Nesse jogo específico da franquia existe
um elemento bastante importante para esta prova, que o diferencia de seus anteriores: um gancho que Link pode utilizar
para se mover em direção a qualquer bloco do cenário.

\begin{figure}[!htb]
     \centering
     \includegraphics[scale=0.8]{hookshot.png}
     \caption{Exemplo de uso do gancho}
\end{figure}

\subsection{3-SAT}

O problema de satisfazibilidade booleana (SAT) é um problema de decisão, cuja instancia é uma escrita com operadores lógicos AND, OR, NOT, variáveis, e parênteses. A questão desse problema de decisão é: dada uma expressão, há alguma atribuição de valores verdadeiros e falsos para as variáveis que torne toda a expressão verdadeira? Uma fórmula da lógica proposicional é dita satisfazível se e somente se é possível atribuir valores lógicos a suas variáveis de tal maneira que eles tornem a fórmula verdadeira. A prova da NP Completude do problema SAT é dada pelo teorema de Cook-Levin \cite{cook1971complexity}.

O problema 3-SAT é um subconjunto do problema SAT, onde cada expressão lógica terá apenas três literais.

\begin{figure}[!htb]
     \centering
     \includegraphics[scale=0.8]{cook.png}
     \caption{Exemplo do problema 3-SAT}
\end{figure}

\section{Framework}

O framework aqui apresentado foi originalmente proposto por \cite{aloupis2015classic}, e tem
como objetivo provar a NP-Hardness de jogos de plataforma. Para isso, ele implementa diversos dispositivos.
Apesar de The Legend of Zelda: a Link to The Past não ser um jogo de plataforma, mostraremos como ele se encaixa neste modelo.

\begin{figure}[!htb]
     \centering
     \includegraphics[scale=0.5]{framework.png}
     \caption{Framework geral para prova da NP Completude}
\end{figure}

Essa framework funciona da seguinte maneira: O jogador começa na posição Start. 
Cada dispositivo de variável obriga o jogador a fazer uma escolha
exclusiva de "verdadeiro" (\(x\)) ou "falso" (\(\lnot x\)) como valor para a variável para uma fórmula booleana. Ambas as escolhas
permitem ao jogador seguir caminhos que levam aos dispositivos de Cláusula, correspondentes as clausulas
que contém aquela literal (\(x\) ou \(\lnot x\)). Esses caminhos podem se cruzar, mas o dispositivo de Crossover
previne que o jogador troque entre caminhos cruzados. Ao visitar o dispositivo de cláusula, o jogador pode
desbloquear a cláusula (um estado permanente de mudança), mas não pode alcançar nenhum dos outros caminhos
conectados ao dispositivo de cláusula. Por fim, depois de atravessar através de todos os dispositivos de variáveis,
chegará a posição final. O jogador pode passear o caminho de checagem se e somente se cada cláusula for desbloqueada
por algum literal. Portanto, basta implementar os dispositivos citados para provar a NP-Hardness de qualquer jogo de plataforma.

Os dispositivos devem seguir as seguintes propriedades:

\textbf{Começo e Fim: } Os dispositivos de começo e fim contém o ponto inicial e o objetivo do personagem, respectivamente.

\textbf{Variável: }Cada dispositivo de variável precisa forçar o jogador a escolher um entre dois caminhos, correspondendo
a (\(xi\)) ou sua negação \(\lnot xi\) sendo escolhidos como o literal satisfeito, como por exemplo o caminho que foi escolhido ou o
caminho que não pode ser transposto. Cada dispositivo de variável deve ser acessível por apenas e tão somente o dispositivo de 
variável anterior, independentemente da escolha feita no dispositivo anterior, no qual o caminho a entrada de um literal não
permita a travessia de volta para a negação do literal.

\textbf{Cláusula e Checagem: }Cada dispositivo de cláusula deve ser acessível do (e inicialmente, apenas do) caminho vindo do
literal correspondente aos literais que aparecem na cláusula da fórmula Booleana. Além disso, quando um jogador
visita o dispositivo de cláusula dessa maneira, ele deve realizar alguma ação que "destrave" o dispositivo. O caminho de checagem
atravessa todo dispositivo de cláusula em sequência, e o jogador pode passar em todos os dispositivos de cláusula através do 
caminho de checagem se e apenas se o dispositivo de cláusula estiver desbloqueado. Assim o caminho de verificação pode ser
totalmente atravessado apenas se todos os dispositivos de variável tiverem sido visitadas dos caminhos de literais. Se o jogador
atravessar todo o caminho de verificação, ele pode acessar o dispositivo de fim.

\textbf{Crossover: }O dispositivo Crossover deve permitir a travessia através de duas passagens que se cruzam,
de tal forma que não há vazamento entre elas.


\section{Prova da NP-Completude}

Para realizar a prova da NP-Completude de Zelda, vamos primeiro demonstrar que tal problema é
NP-Hard, reduzindo-o para o problema 3SAT. Depois, mostraremos que é NP. Como por definição o conjunto
de problemas NP-Completo é a intersecção dos conjuntos de problemas NP com o conjunto de problemas NP-Hard,
teremos provado que Zelda é NP-Completo.

\subsection{NP-Hard}

\newtheorem*{theorem}{Teorema 1}

\begin{theorem}
    É NP-Hard decidir quando uma posição final é alcançável a partir de uma dada posição
    inicial em uma versão generalizada de The Legend Of Zelda: A Link to the Past.
\end{theorem}


\begin{proof}
    Existem várias maneiras de demonstrar esse teorema, uma vez que se uma parte do jogo
    for NP-Completa, todo o jogo também será. Para isso, utilizaremos apenas baús de tesouro
    e blocos, que servirão como alvo do gancho (vamos assumir que Link começa com esse item).
    Para realizar a prova, basta descrevermos o ambiente do jogo de acordo com os dispositivos
    do framework apresentado. Como esse não é um jogo de plataforma, não precisamos de um dispositivo
    específico de começo e fim, podendo o começo ser uma posição arbitrária dentro de uma caverna
    (onde normalmente ficam os puzzles que precisam do gancho) e uma posição arbitrária do mapa aberto,
    respectivamente.
    
    O dispositivo de variável, mostrado na figura 4, funciona da seguinte maneira: Link se aproxima ou
    do canto superior esquerdo ou do canto superior direito, dependendo de qual valor foi escolhido na
    variável anterior. Então Link usa o gancho para ir até um baú no centro superior, e por fim utiliza o
    gancho em um dos dois baús de baixo. Uma vez que Link alcançou um dos baús de baixo, o outro se torna
    inalcançável. Observe como diversas barreiras ao redor dos corredores impedem link de usar o gancho em outros
    baús de direções indesejáveis.
    
    \begin{figure}[!htb]
        \centering
        \includegraphics[scale=0.8]{proof1.png}
        \caption{Dispositivos de variável e cláusula, respectivamente}
    \end{figure}
    
    O dispositivo de cláusula é ilustrado também na figura 4. Os três corredores de cima correspondem aos literais que aparecem
    na cláusula. Quando link visita um desses corredores, ele deve empurrar o bloco para frente, o que permite a ele 
    utilizar o gancho em um dos blocos da direita depois, quando estiver atravessando o caminho de checagem (figura 5).
    Note que a barreira mais a esquerda do dispositivo de cláusula previne Link de "pular" clausulas não satisfeitas, mesmo
    se ele puder usar o gancho arbitrariamente a longas distâncias.
    
    \begin{figure}[!htb]
        \centering
        \includegraphics[scale=0.7]{proof2.png}
        \caption{Caminho de checagem}
    \end{figure}
    
    Por fim, o dispositimo de crossover já é nativamente implementado no jogo, como mostra a figura 6.
    
    \begin{figure}[!htb]
         \centering
         \includegraphics[scale=0.7]{proof3.png}
         \caption{Dispositivo de Crossover}
    \end{figure}
    
\end{proof}

\subsection{NP}

\newtheorem*{theorem2}{Teorema 2}

\begin{theorem2}
    Decidir quando uma posição final é alcançável a partir de uma dada posição
    inicial em uma versão generalizada de The Legend Of Zelda: A Link to the Past é NP.
\end{theorem2}

\begin{proof}
    Para concluir a prova da NP-Completude de Zelda, basta provarmos que finalizar o jogo é um problema NP.
    Se mostrarmos um caminho que é sempre possível,
    então há um algoritmo de solução de localização de caminho com um tempo de execução que é
    no máximo polinomial no tamanho da entrada, ou seja,
    podemos mostrar por certificados que o problema é verificável em tempo polinomial, o que por definição o torna um problema NP.
    
    Então, suponde que existe tal solução, vamos considerar qualquer caminho que Link pode fazer no mapa.
    Uma vez que só é possível abrir os baús que contém itens únicos do jogo uma vez, e que se matarmos
    todos os inimigos pelo menos uma vez teremos matado todos os inimigos que soltam itens especiais e também o chefe final do jogo, podemos dizer que
    o caminho que passa por cada um dos baús atualmente alcançáveis, e elimina cada um dos inimigos alcançáveis
    uma vez (lembrando que todos os inimigos podem ser derrotados desde o primeiro momento do jogo), então
    existe uma solução polinomial ao tamanho da entrada.
    
    Tal demonstração é similar as feitas por \cite{gabrielsen2012video} para outros jogos da Nintendo
    (Super Mario Bros., Donkey Kong Country e Metroid).

\end{proof}    



\section{Conclusão}

Neste artigo, mostramos como jogar The Legend of Zelda: A Link to the Past é um
problema NP-Completo. O framework de prova utilizado e a demonstração que tal problema
é NP poderiam ser utilizados para demonstrar que qualquer jogo da série é NP-Completo, contanto
que utilizem os mesmos elementos (mapa aberto, gancho de movimentação e restrições de caminhos).

Esse tipo de trabalho é interessante para mostrar como pode ser computacionalmente custoso
criar um algoritmo para jogar certos jogos, expondo a necessidade de se buscar soluções
alternativas como por exemplo algoritmos genéticos, redes neurais ou programação de agentes para
obter uma solução boa.

Por fim, é interessante avaliar como é acessível esse tipo de prova, mostrando que um estudante
que tenha a base da teoria da computação pode demonstrar que jogar seu jogo favorito é (ou não) NP-Completo.

%\section*{Agradecimentos}

\bibliographystyle{sbc}
\bibliography{sbc-template}

\end{document}
