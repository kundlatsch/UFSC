\section{Introdução}

Para compreender o funcionamento do UFS (que também é chamado de Berkeley Fast File System, BSD Fast File System ou Fast File System), precisamos compreender como era o sistema de arquivos de versões mais antigos do Unix. O sistema de arquivos original, descrito no clássico artigo de Thompson que fazia a descrição do funcionamento do Unix \cite{thompson1978unix}, era chamado apenas de FS, e era formado apenas por um bloco de inicialização, superbloco, um conjunto de inodes e os blocos de dados. Nos primórdios do Unix, esse sistema de arquivos era efetivo apenas por conta da baixa capacidade dos discos, quando não precisava de métodos muito robustos para garantir eficiência. Com o passar do tempo, a tecnologia avançou e a capacidade dos discos aumentou, e os movimentos dos cabeçotes para se deslocar através os inodes e os blocos de dados acabaram sendo caracterizados como causadores de problemas de performance \cite{tanenbaum2015modern}, num processo conhecido como Thrashing - situação onde muito recurso computacional é utilizado para realizar uma quantidade mínima de trabalho.

O UFS trouxe conceitos novos com o objetivo de melhorar o desempenho e impedir o Thrashing, com a criação de grupos de cilindros e a divisão do disco em pedaços menores, cada um com seu próprio conjunto de inodes e blocos de dados. O UFS é uma reimplementação do antigo FS, e fornece taxas de transferência consideravelmente superiores usando políticas de alocação mais flexíveis que permitem uma melhor localização de referência e podem ser adaptadas a uma ampla gama de características do processador. Esse sistema de arquivos foi proposto por McKusick et. al. \cite{mckusick1984fast}, e sua implementação original teve taxas de acesso a arquivos de até dez vezes mais rápido que o sistema de arquivos UNIX tradicional.

