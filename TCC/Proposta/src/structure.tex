%------------------------------------------------------------
%------------    Estrutura do texto   -----------------------      

% Pacotes Básicos:
\usepackage[T1]{fontenc}		% Seleção de códigos de fonte.
\usepackage[utf8]{inputenc}		% Codificação do documento (conversão automática dos acentos)
\usepackage{lastpage}		    % Usado pela Ficha catalográfica
\usepackage{indentfirst}		
\usepackage{color}				% Controle das cores
\usepackage{graphicx}			% Inclusão de gráficos
\usepackage{tabularx}
\usepackage{microtype} 		  	% Melhorias da justificação
\usepackage{pdfpages}           %inserir páginas em PDF
% Pacotes Extras:
\usepackage{amsmath,amsthm}     % Símbolos Matemáticos
\usepackage[portuguese, ruled, linesnumbered,commentsnumbered, algo2e, vlined, lined, boxed, algochapter]{algorithm2e} % Algoritmos 
\usepackage{hyperref}      % Criação de links.

% Escolha da formatação das referências Bibliográficas: 
\usepackage[alf,abnt-etal-list=0,abnt-etal-cite=3]{abntex2cite}	% Citações padrão ABNT  (AUTOR, ANO)
\usepackage{etoolbox}
%\usepackage[num]{abntex2cite}  % Citações numéricas (1)
%\citebrackets[] % Usar este comando para a citação numérica aparecer com [].

% Numeração das Figuras e Tabelas
\counterwithin{figure}{chapter}
\counterwithin{table}{chapter}

% Defininção de Cores:
\definecolor{blue}{RGB}{25,25,112}
\makeatletter % informações do PDF
\hypersetup{
    %pagebackref= false,
	pdftitle={\@title}, 
	pdfauthor={\@author},
    %pdfsubject={\imprimirpreambulo},
	pdfcreator={LaTeX with abnTeX2},
	pdfkeywords={abnt}{latex}{abntex}{abntex2}{trabalho acadêmico}, 
	colorlinks=true,    % false: boxed links; true: colored links
    linkcolor=blue,     % color of internal links
    citecolor=blue,     % color of links to bibliography
    filecolor=magenta, 	% color of file links
	urlcolor=blue,
	bookmarksdepth=4
}
\makeatother

% -------------------------------------------- 
% Espaçamentos entre linhas e parágrafos 
\setlength{\parindent}{1.3cm} % O tamanho do parágrafo

% Controle do espaçamento entre um parágrafo e outro:
\setlength{\parskip}{0.2cm}  % tente também \onelineskip

% Definição de ambientes matemáticos em português 
\newtheorem{teorema}{Teorema}[chapter]
\newtheorem{axioma}{Axioma}[chapter]
\newtheorem{corolario}{Corolário}[chapter]
\newtheorem{lema}{Lema}[chapter]
\newtheorem{proposicao}{Proposição}[chapter]
\newtheorem{definicao}{Definição}[chapter]
\newtheorem{exemplo}{Exemplo}[chapter]
\newtheorem{observacao}{Observação}[chapter]

% Novos Comandos
\usepackage{tgtermes}
\renewcommand{\ABNTEXchapterfont}{\rmfamily\bfseries}

% Variáveis adicionais
\providecommand{\imprimirautorcite}{}
\newcommand{\autorcite}[1]{\renewcommand{\imprimirautorcite}{#1}} 
\providecommand{\imprimirsubtitulo}{}
\newcommand{\subtitulo}[1]{\renewcommand{\imprimirsubtitulo}{#1}} 
\providecommand{\imprimirsigla}{}
\newcommand{\sigla}[1]{\renewcommand{\imprimirsigla}{#1}}
\providecommand{\imprimiruf}{}
\newcommand{\uf}[1]{\renewcommand{\imprimiruf}{#1}}
\providecommand{\imprimircurso}{}
\newcommand{\curso}[1]{\renewcommand{\imprimircurso}{#1}}
\providecommand{\imprimirinstituto}{}
\newcommand{\instituto}[1]{\renewcommand{\imprimirinstituto}{#1}}
\providecommand{\imprimirdepartamento}{}
\newcommand{\departamento}[1]{\renewcommand{\imprimirdepartamento}{#1}}
\providecommand{\imprimirano}{}
\newcommand{\ano}[1]{\renewcommand{\imprimirano}{#1}}
\providecommand{\imprimirgrau}{}
\newcommand{\grau}[1]{\renewcommand{\imprimirgrau}{#1}}
\providecommand{\imprimirexaminadorum}{}
\newcommand{\examinadorum}[1]{
    \renewcommand{\imprimirexaminadorum}{#1}}
\providecommand{\imprimirexaminadordois}{}
\newcommand{\examinadordois}[1]{
    \renewcommand{\imprimirexaminadordois}{#1}}
\providecommand{\imprimirexaminadortres}{}
\newcommand{\examinadortres}[1]{
    \renewcommand{\imprimirexaminadortres}{#1}}
\providecommand{\imprimirexaminadorquatro}{}
\newcommand{\examinadorquatro}[1]{
    \renewcommand{\imprimirexaminadorquatro}{#1}}
\providecommand{\imprimirttorientador}{}
\newcommand{\ttorientador}[1]{
    \renewcommand{\imprimirttorientador}{#1}} 
\providecommand{\imprimirttcoorientador}{}
\newcommand{\ttcoorientador}[1]{
    \renewcommand{\imprimirttcoorientador}{#1}}
\providecommand{\imprimirttexaminadorum}{}
\newcommand{\ttexaminadorum}[1]{
    \renewcommand{\imprimirttexaminadorum}{#1}}
\providecommand{\imprimirttexaminadordois}{}
\newcommand{\ttexaminadordois}[1]{\renewcommand{
        \imprimirttexaminadordois}{#1}}
\providecommand{\imprimirttexaminadortres}{}
\newcommand{\ttexaminadortres}[1]{
    \renewcommand{\imprimirttexaminadortres}{#1}}
\providecommand{\imprimirttexaminadorquatro}{}
\newcommand{\ttexaminadorquatro}[1]{
    \renewcommand{\imprimirttexaminadorquatro}{#1}}
		
%----------------------------------------------------
\renewcommand{\imprimircapa}{ % Capa 
\begin{capa}
        \begin{center}
                \begin{DoubleSpace}
                \MakeUppercase{\imprimirinstituicao } \\
                 \MakeUppercase{\imprimirinstituto } \\
                \MakeUppercase{\imprimirdepartamento} \\
                \end{DoubleSpace}
                \vspace{5cm}
				\MakeUppercase{\imprimirautor}  \\
                \imprimirorientadorRotulo ~\imprimirorientador \\
                \imprimircoorientadorRotulo ~\imprimircoorientador \\
                        				
				\vspace{5cm}
             \textbf{{\large\MakeUppercase{\imprimirtitulo}}} \\
			 \textbf{{\large \MakeUppercase{\imprimirsubtitulo}}} \\
				\vfill
        {\large{\imprimirlocal, ~\imprimiruf \\ \imprimirano  }}
        \end{center}
\end{capa}   
} % Capa



%----------------------------------------------------
\renewcommand{\imprimirfolhaderosto}{% folha de rosto
       \begin{center}
                \MakeUppercase{\imprimirinstituicao } \\
                 \MakeUppercase{\imprimirinstituto } \\
                \MakeUppercase{\imprimirdepartamento} \\
                
                \vspace{4cm}
				\MakeUppercase{\imprimirautor}  \\
				\vspace{2cm}
			    \begin{DoubleSpace}
                \MakeUppercase{\textbf{\imprimirtitulo} } \\
                \MakeUppercase{\textbf{\imprimirsubtitulo}} \\
                \end{DoubleSpace} 
      \end{center}
    \vfill 
    \begin{flushright} 
    \parbox{0.6\linewidth}{
    Proposta de TCC submetida para a aprovação na matéria de Introdução ao TCC, requisito parcial para obtenção do grau de Bacharel em Ciências da Computação pela Universidade Federal de Santa Catarina.\\ \\
% 		\imprimirtipotrabalho~ apresentada ao Curso de \imprimircurso~ da \imprimirinstituicao~ como parte dos
% 		requisitos necessários para a obtenção do grau de \imprimirgrau. \\ \\
		\textbf{\imprimirorientadorRotulo}~\imprimirorientador \\
		
		\textbf{\imprimircoorientadorRotulo}~\imprimircoorientador}
   \end{flushright} 
		\vfill
   \begin{center}
   {\large{\imprimirlocal, ~ \imprimiruf \\
   \imprimirano} }
   \end{center} }  % folha de rosto

%----------------------------------------------------


