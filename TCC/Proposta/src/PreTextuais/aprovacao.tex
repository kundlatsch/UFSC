\begin{table}[H]
\centering
\begin{tabular}{ll}
\multicolumn{2}{c}{\cellcolor[HTML]{C0C0C0}\textbf{FOLHA DE APROVAÇÃO DE PROPOSTA DE TCC}}                \\ \hline
\multicolumn{1}{|l|}{\textbf{Acadêmico}}            & \multicolumn{1}{l|}{Gustavo Emanuel Kundlatsch}              \\ \hline
\multicolumn{1}{|l|}{\textbf{Título do trabalho}} &
  \multicolumn{1}{l|}{\begin{tabular}[c]{@{}l@{}}Revisão de Percepções\end{tabular}} \\ \hline
\multicolumn{1}{|l|}{\textbf{Curso}}                & \multicolumn{1}{l|}{Ciência da Computação/INE/UFSC} \\ \hline
\multicolumn{1}{|l|}{\textbf{Área de Concentração}} & \multicolumn{1}{l|}{Inteligência Artificial}                    \\ \hline
\end{tabular}%
\end{table}
\noindent
\textbf{Instruções para preenchimento pelo \underline{ORIENTADOR DO TRABALHO}:}

\noindent - Para  cada  critério avaliado,  assinale  um  X  na  coluna  SIM  apenas  se  considerado  aprovado.

\noindent Caso contrário, indique as alterações necessárias na coluna Observação.
\begin{table}[H]
\resizebox{\textwidth}{!}{%
\begin{tabular}{|l|
>{\columncolor[HTML]{C0C0C0}}l |
>{\columncolor[HTML]{C0C0C0}}l |
>{\columncolor[HTML]{C0C0C0}}l |
>{\columncolor[HTML]{C0C0C0}}l |l|}
\hline
\multicolumn{1}{|c|}{\cellcolor[HTML]{C0C0C0}} &
  \multicolumn{4}{c|}{\cellcolor[HTML]{C0C0C0}\textbf{Aprovado}} &
  \cellcolor[HTML]{C0C0C0} \\ \cline{2-5}
\multicolumn{1}{|c|}{\multirow{\cellcolor[HTML]{C0C0C0}\textbf{Critérios}}} &
  \multicolumn{1}{c|}{\cellcolor[HTML]{C0C0C0}\textbf{Sim}} &
  \multicolumn{1}{c|}{\cellcolor[HTML]{C0C0C0}\textbf{Parcial}} &
  \multicolumn{1}{c|}{\cellcolor[HTML]{C0C0C0}\textbf{Não}} &
  \multicolumn{1}{c|}{\cellcolor[HTML]{C0C0C0}\textbf{\begin{tabular}[c]{@{}c@{}}Não \\ se aplica\end{tabular}}} &
  \multirow{\cellcolor[HTML]{C0C0C0}\textbf{Observação}} \\ \hline
\begin{tabular}[c]{@{}l@{}}1. O trabalho é adequado para um TCC no \\ CCO/SIN (relevância / abrangência)?\end{tabular} &
   &
   &
   &
   &
   \\ \hline
2. O titulo do trabalho é adequado? &
   &
   &
   &
   &
   \\ \hline
3. O tema de pesquisa está claramente descrito? &
   &
   &
   &
   &
   \\ \hline
\begin{tabular}[c]{@{}l@{}}4. O problema/hipóteses de pesquisa do\\ trabalho está claramente identificado?\end{tabular} &
   &
   &
   &
   &
   \\ \hline
5. A relevância da pesquisa é justificada? &
   &
   &
   &
   &
   \\ \hline
\begin{tabular}[c]{@{}l@{}}6. Os objetivos descrevem completa e\\ claramente o que se pretende alcançar neste trabalho?\end{tabular} &
   &
   &
   &
   &
   \\ \hline
\begin{tabular}[c]{@{}l@{}}7. É definido o método a ser adotado no\\ trabalho? O método condiz com os objetivos e \\ é adequado para um TCC?\end{tabular} &
   &
   &
   &
   &
   \\ \hline
\begin{tabular}[c]{@{}l@{}}8. Foi definido um cronograma coerente com \\ o método definido (indicando todas as\\ atividades) e com as datas das entregas\\ (p.ex. Projeto I, II, Defesa)?\end{tabular} &
   &
   &
   &
   &
   \\ \hline
\begin{tabular}[c]{@{}l@{}}9. Foram identificados custos relativos \\ à execução deste trabalho (se houver)?\\ Haverá financiamento para estes custos?\end{tabular} &
   &
   &
   &
   &
   \\ \hline
\begin{tabular}[c]{@{}l@{}}10. Foram identificados todos os envolvidos\\ neste trabalho?\end{tabular} &
   &
   &
   &
   &
   \\ \hline
\begin{tabular}[c]{@{}l@{}}11. As formas de comunicação foram\\ definidas (ex: horários para orientação)?\end{tabular} &
   &
   &
   &
   &
   \\ \hline
\begin{tabular}[c]{@{}l@{}}12. Riscos potenciais que podem causar\\ desvios do plano foram identificados?\end{tabular} &
   &
   &
   &
   &
   \\ \hline
\begin{tabular}[c]{@{}l@{}}13.  Caso o TCC envolva a produção de um \\ software ou outro tipo de produto e seja \\ desenvolvido também como uma atividade \\ realizada numa empresa ou laboratório, \\ consta da proposta uma declaração (Anexo 3)\\ de ciência e concordância com a entrega do\\ código fonte e/ou documentação produzidos?\end{tabular} &
   &
   &
   &
   &
   \\ \hline
\end{tabular}%
}
\end{table}

\begin{table}[H]
\centering
\begin{tabular}{|l|lll|}
\hline
\textbf{Avaliação}          & \textbf{[ ] Aprovado}      & \textbf{}                & \textbf{[ ] Não Aprovado} \\ \hline
 & \textit{Nome} & \textit{Data} & \textit{Assinatura}                                                                                     \\ \hline
\textbf{Professor Responsável} & \multicolumn{1}{l|}{Elder Rizzon Santos} & \multicolumn{1}{l|}{ } &  \\ \hline
\textbf{Orientador externo} & \multicolumn{1}{l|}{Thiago Ângelo Gelaim} & \multicolumn{1}{l|}{} &                      \\ \hline
\end{tabular}
\end{table}