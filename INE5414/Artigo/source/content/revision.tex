\section{Trabalhos Correlatos}
O foco da pesquisa, por conta do tema do presente artigo, resultou em uma quantidade volumosa de trabalhos, por conta da generalidade de seu assunto. As pesquisas foram feitas no Google Scholar, acessado no dia 30 de Março de 2019. As pesquisas estão categorizadas em índices de 1 a 4, de acordo com a generalidade, sendo 1 o mais geral e 4 o mais específico. Por conta do grande volume de artigos encontrados, vamos trabalhar com alguns dos mais relevantes e que mais bem se enquadram dentro da proposta de estudar soluções de segurança blockchain para internet das coisas. O critério de relevância considerado é o número de citações.

\begin{table}[H]
\centering
\large
\caption{Pesquisas realizadas, categorizadas por generalidade}\label{tab1}
\begin{tabular}{|l|c|c|}
\hline
\textbf{Palavra chave} & \textbf{Total} & \textbf{Categoria}\\
\hline
Internet of Things & 3.180.000 & 1\\
IoT, Security & 147.000 & 2\\
Blockchain & 48.400 & 2\\
IoT, Blockchain & 11.700 & 2\\
IoT, Blockchain, Security & 11.600 & 3\\
IoT, Blockchain, Attacks & 6.140 & 3\\
IoT, DDoS, Blockchain Security & 1.060 & 4\\
\hline
\end{tabular}
\end{table}

\subsection{NeuroMesh: IoT Security Enabled by a
Blockchain Powered Botnet Vaccine}

    Esse artigo de 2019 \cite{falco2019neuromesh} discute o fato de que apesar do crescente número de dispositivos IoT, seus fabricantes se concentraram na funcionalidade e nos recursos dos dispositivos e tornaram a segurança uma reflexão tardia. Como os dispositivos de IoT têm pequenas capacidades de memória e processadores de baixa potência, muitas empresas de segurança não conseguiram desenvolver software anti-malware para esses dispositivos. Segundo os autores, as soluções para segurança em IoT de hoje são pesadas e custosas, além de normalmente não serem muito eficientes. Nesse trabalho, eles apresentam uma solução de segurança IoT leve que usa ferramentas de hackers contra os hackers - em essência, uma vacina para IoT. O software desenvolvido fornece gerenciamento segurança e inteligência para dispositivos IoT usando uma botnet “amigável" operado através de uma infraestrutura de comunicação comprovada e existente para sistemas distribuídos - o blockchain Bitcoin. Essa solução desenvolvida é a NeuroMesh que consiste nos componentes principais da proteção de ponto de extremidade NeuroNode, servidores Rendezvous, Servidor de comando e controle NeuroCloud (CnC) e o centro de operações de segurança NeuroPrime (SOC). NeuroMesh é tem esse nome por conta de seu uso de redes neurais e de malha para proteger dispositivos IoT. A solução NeuroMesh detecta e remove proativamente dispositivos IoT infecrados, e listas negras ou listas brancas permitem controle de acesso baseado em IP e permitem comunicações seguras e atualizações para dispositivos IoT sobre o protocolo Bitcoin de comunicação.
    A solução NeuroMesh é uma abordagem multifacetada e abrangente que visa cada uma das deficiências de segurança da IoT descritas no artigo. Essa arquitetura leve fornece segurança de terminal que é difícil de contornar. Além disso, o sistema pode ser implementado em uma ampla variedade de dispositivos, incluindo IoT industrial, onde é necessária uma maior segurança para evitar ataques maliciosos generalizados nas indústrias mais sensíveis tais como sistemas médicos, fabricação industrial, sistemas de distribuição de energia e veículos autônomos.

\subsection{A Decentralized Privacy-Preserving Healthcare Blockchain for IoT}

    Neste outro artigo de 2019 \cite{dwivedi2019decentralized} começa com uma discussão sobre como o atendimento médico tornou-se uma das partes mais indispensáveis da vida humana, levando a aumento dramático em grandes dados médicos. Segundo os autores, para agilizar o processo de diagnóstico e tratamento, os profissionais estão agora adotando a tecnologia wearable baseada na Internet das Coisas (IoT). Nos últimos anos, testemunhou-se um grande aumente desses dispositivos, e hoje chegamos aos bilhões de sensores, dispositivos e veículos conectados através da Internet. Na área médica, a tecnologia de monitoramento remoto do usuário se tornou comum para o tratamento e cuidado dos pacientes. No entanto, essas tecnologias também representam graves riscos de privacidade e trazem preocupações com a segurança, transferência e registro de transações de dados. Esses problemas de segurança e privacidade de dados médicos poderia resultar de um atraso no progresso do tratamento, podendo até mesmo chegar a  colocar em risco a vida do paciente. Tendo em vista esse problema em aberto, os autores propõem o uso de um blockchain para fornecer gerenciamento e análise seguros de big data de saúde. Contudo, blockchains são computacionalmente caros, exigem alta largura de banda e poder computacional extra, e, portanto, não são completamente adequados para a maioria dos dispositivos de IoT com recursos restritos. Neste trabalho, eles tentam resolver os problemas mencionados acima ao usar blockchain com dispositivos IoT, e é proposta uma nova estrutura de modelos blockchain modificados para dispositivos IoT que dependem da natureza distribuída e de outras propriedades adicionais de privacidade e segurança da rede. Essas propriedades adicionais de privacidade e segurança no modelo são baseadas em criptografia avançada. As soluções fornecidas tornam os dados e transações do aplicativo IoT mais seguros e anônimos em, características provenientes da natureza de uma rede baseada em blockchain.
    
\subsection{Blockchain based Proxy Re-Encryption Scheme for Secure IoT Data Sharing}

    Dados são o coração de um ecossistema de Internet das coisas. A maioria dos sistemas atuais de IoT estão usando sistemas de compartilhamento de dados baseados em nuvem, que serão difíceis de escalar para atender às demandas dos futuros sistemas de IoT. Envolvimento de provedor de serviços de terceiros também exige confiança de usuário do sensor e usuário de dados do sensor. Além disso, taxas precisam ser pagas pelos seus serviços. Para resolver os problemas de escalabilidade e confiança e para automatizar os pagamentos, este artigo de 2018 \cite{manzoor2018blockchain}   apresenta um proxy baseado em um esquema blockchain de re-criptografia. O sistema armazena os dados da IoT em um nuvem distribuída após o processo de encriptação. Para compartilhar os dados coletados pela IoT, o sistema estabelece contratos inteligentes dinâmicos em tempo de execução entre o sensor e o usuário de dados sem o envolvimento de uma terceira parte confiável. Ele também usa um esquema de re-criptografia de proxy muito eficiente que permite que os dados só sejam visíveis pelo proprietário e a pessoa presente no contrato inteligente. Para os autores, essa nova combinação de contratos inteligentes com re-criptografia de proxy fornece uma plataforma eficiente, rápida e segura para armazenamento, negociação e gestão de dados de sensores. O sistema proposto é implementado em um testbed baseado Ethereum para analisar o desempenho e as propriedades de segurança.
    Os autores não só propõem uma negociação baseada em uma plataforma blockchain com a combinação de um esquema de reencriptação de proxy livre para garantir a transferência segura dos dados dos sensores para o usuário, asim como também validam um modelo de prova de conceito em um testbed privado Ethereum e demonstram a praticidade do sistema usando laptops prontos para uso e raspberry pis. Além disso, os experimentos e análises realizadas verificam que a combinação do esquema de re-criptografia de proxy com o blockchain permitem uma plataforma segura para negociação e compartilhamento do dados de sensores. O uso de blockchain aumenta o atraso, mas mantém um registro de toda a interação entre as entidades e elimina a necessidade de um terceiro usuário confiável. Portanto, esse trabalho apresenta uma estrutura que fornece uma plataforma eficiente, rápida e segura para armazenamento, negociação e gestão de dados de sensores.

\subsection{A software defined fog node based distributed blockchain cloud architecture for IoT}

    Esse trabalho de 2018 \cite{sharma2018software} apresenta um tema um pouco diferente dos demais artigos aqui discutitos, pois trata de blockchain em arquiteturas de IoT em cloud. Segundo os autores, a recente expansão da Internet das Coisas e a consequente explosão no volume de dados produzidos por dispositivos inteligentes levaram à terceirização de dados para centros de dados designados. Contudo, para gerenciar essas enormes massas de dados, os centros de dados centralizados, como o armazenamento em nuvem, não podem permitir caminhos auspiciosos. Existem muitos desafios que devem ser abordados na arquitetura de rede tradicional devido ao rápido crescimento na diversidade e número de dispositivos conectados à internet, que não é projetado para fornecer alta disponibilidade, entrega de dados em tempo real, escalabilidade, segurança, resiliência e baixa latência. Para lidar com essas questões, esse artigo propõe uma nova arquitetura de nuvem distribuída baseada em blockchain com uma rede definida por software (software-defined networking, ou SDN) que permite que os fog nodes do controlador na borda da rede atendam aos princípios de design requeridos. O modelo proposto é uma arquitetura de nuvem distribuída baseada na tecnologia blockchain, que oferece acesso de baixo custo, seguro e sob demanda às infraestruturas de computação mais competitivas em uma rede IoT. Com a criação de uma infraestrutura de nuvem distribuída, o modelo proposto permite uma relação custo-benefício computação de alto desempenho. Além disso, para trazer recursos de computação para a borda da rede IoT e permitir acesso de baixa latência a grandes quantidades de dados de maneira segura, nós fornecemos uma arquitetura de fog node que usa técnicas de SDN e blockchain. Fog nodes são computação de fog distribuída em entidades que permitem a implantação de serviços de fog e são formados por vários recursos de computação na borda da rede IoT. Os autores não só propõem o modelo, como para validar o desempenho da arquitetura proposta a comparam com modelos existentes usando várias medidas de desempenho. Os resultados da avaliação mostram que o desempenho é melhorado reduzindo o atraso induzido, reduzindo o tempo de resposta, aumentando o rendimento, e a habilidade para detectar ataques em tempo real na rede IoT com overheads de baixo desempenho.

\subsection{Diferencial}

Foram apresentados nesse seção 4 artigos estado da arte no uso de técnicas de blockchain para resolver diversos problemas de segurança em internet das coisas. Todos esses artigos são trabalhos recentes, datando de 2018 e de 2019, mostrando a atualidade dessa área de pesquisa. O diferencial do trabalho atual para esses artigos é a proposta de um modelo genérico, ou seja, buscamos mostrar uma forma ampla de aplicar blockchain em qualquer sistema que utiliza internet das coisas, independente do escopo.

Apesar de trabalhos similares já terem sido feitos, como foi tratado nas subseções anteriores, tentar resolver esse problema de maneira livre de escopo, para que possa ser implementado por qualquer cientista que queira avançar a pesquisa ou até mesmo impresas que tenham interesse em desenvolver a área, é algo que não foi encontrado na revisão bibliográfica realizada.

Além disso, a revisão bibliográfica realizada buscou ser imparcial e abranger artigos de diversas áreas dentro das implementações propostas (utilizando blockchain), tentando abranger desde trabalhos puramente matemáticos e teóricos até trabalhos completamente aplicados para resolver um problema em específico. Os que foram escolhidos para aparecer aqui foram aqueles com mais potencial de se enquadrar na proposta apresentada, independente do escopo original que apresentava.