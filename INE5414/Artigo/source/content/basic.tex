
\section{Conceitos básicos}

A seção de conceitos básicos é destinada a apresentar elementos fundamentais para a revisão bibliográfica proposta. No caso específico desse artigo, optamos por apresentar os seguintes três temas: Internet das Coisas, Segurança e Blockchain. Como esses são conceitos relativamente avançados de computação, é necessário garantir que eles sejam bem compreendidos para que tanto a revisão do estado da arte quanto a proposta de solução sejam bem sucedidos. Além de serem conceitos relativamente avançados de computação, eles também se comportam como blocos básicos para construir conceitos mais avançados. Por exemplo, como queremos mostrar as soluções de publicações atuais sobre o uso de Blockchain em Segurança, é preciso compreender a fundo o que é uma rede Blockchain para saber quais os benefícios mas também as desvantagens de tomar essa solução para resolver o problema de segurança em uma rede de Internet das Coisas.

\subsection{IoT}


Internet das Coisas, possuí uma carga teórica bastante pesada para ser profundamente compreendida. Apesar de seu conceito poder ser facilmente passado para o público leigo como ``as coisas do dia a dia conectadas a internet'' como máquinas de lavar roupa, cafeteiras, robôs de limpeza do chão ou até lâmpadas, seu conceito teórico envolve uma série de tecnologias de ponta que envolvem redes, dispositivos embarcados e, para nosso caso em especial, segurança. Um sistema embarcado (ou sistema embutido) é um sistema que possui um microprocessador e em que o computador é completamente ligado a computação do dispositivo que ele controla \cite{ganssle2003embedded}. Em computadores de propósito geral o sistema pode ser programado para fazer qualquer coisa que tenha capacidade técnica para realizar. Um sistema embarcado, por outro lado, realiza um conjunto de tarefas que foram predefinidas em sua confecção, com limitações tanto em hardware (componentes lógicas para tarefas específicas) quanto em software (programa dedicado ou sistema operacional compilado de acordo com a necessidade). A ideia de ter um sistema que realiza apenas um objetivo específico é a capacidade de otimização computacional, que reflete na capacidade de otimizar custo, gerando um preço muito melhor para o usuário final.

O conceito de internet das coisas está intimamente ligado aos sistemas embarcados, pois surge da junção de objetos do cotidianos, que a partir de uma placa embarcada consegue se conectar na internet \cite{iotDef} para realizar as tarefas necessárias, tanto para a otimização da própria tarefa que o objeto realiza quanto para conseguir uma boa performance em relação ao conjunto que o objeto compõem, como em uma casa inteligente em que todos os objetos se comunicam para agregar qualidade de vida aos moradores.

A internet das coisas é um conceito que não é novo, pois as empresas de tecnologia e especialistas discutem a ideia há décadas, sendo que a primeira torradeira conectada à internet foi revelada em uma conferência em 1989.
O ecossistema da internet das coisas (IoT) é composto por um
grande número de dispositivos interconectados que coletam, processam
gerar e compartilhar grandes quantidades de (possivelmente sensíveis e
informação critica \cite{brotsis2019blockchain}. Ou seja, IoT denota os dispositivos eletrônicos ou elétricos de muitos tamanhos e capacidades diferentes conectados à Internet \cite{miraz2018blockchain}.

Diversos objetos do nosso cotidiano podem ter suas versões conectadas a internet, como por exemplo eletrodomésticos, dispositivos médicos, câmeras, e todos os tipos de sensores. Isso abre as portas para inovações que facilitar novas interações entre os próprios objetos e os seres humanos, e pode abrir caminho para inovações ainda mais interessantes, como cidades inteligentes e carros autônomos. A crescente de objetos conectados a internet é resultado do barateamento da internet por si só, e também da popularização de dispositivos como smartphones, que torna concebível para o usuário final dispositivos ainda mais diferentes conectados a internet e se comunicando.
Um exemplo de como dispositivos IoT devem se tornar ainda mais presentes no futuro são os wearables, dispositivos inteligentes feitos para você vestir. O exemplo mais básico são os smartwatches, que no seu pulso são capazes de avisar sobre notificações recebidas no celular, batimentos cardíacos, dar informações detalhadas de GPS e outras diversas aplicações. Wearables ainda mais ousados como pulseiras, tênis e até bijuterias devem se tornar comum nas próximas décadas, por conta do seu custo relativamente baixo e miríade de aplicações.

\subsection{Segurança}

 Segurança é um tópico especialmente profundo, pois envolve desde coisas mais práticas e até simples (porém de uma grande base matemática) como a criptografia até conceitos mais complexos como ataques distribuídos ou injeções em programas desprotegidos. Segurança da computação certas vezes se torna até mesmo um tabu pela imagem negativa que tem sido continuamente passada pela mídia de \textit{hackers} e \textit{crackers}, programadores que utilizam seu conhecimento avançado em computação para realizar ataques danosos, seja por motivos pessoais como prestígio e crimes virtuais para estorquir ou roubar, ou por motivos mais autruístas como o \textit{cyber} ativismo. Apesar de ter esse conotação muitas vezes tomada como negativa, profissionais de segurança da informação trabalham justamente para evitar que pessoas mal intencionadas se aproveitem de falhas que passaram em branco tanto pela equipe de programação quanto pela equipe de testes de uma empresa.

A segurança da informação (SI) tem como objetivo proteger um dado conjunto de informações, como o próprio nome sugere, no sentido de preservar o valor que possuem para um determinado indivíduo, uma empresa ou qualquer tipo de organização, como o próprio governo. São propriedades básicas da segurança da informação: confidencialidade, integridade, disponibilidade e autenticidade. O escopo da SI não é restrito a tão somente sistemas computacionais, informações de bancos de dados e outros sistemas de armazenamento, pois o conceito é aplicado para todos os aspectos da proteção de informações e dados, como por exemplo um livro de presenças de uma escola, que caso ser violado pode causar dano as pessoas que assinaram ele (por ter suas assinaturas inspecionadas) ou a instituição (caso o livro seja manipulado por alguém). Além disso, existe o conceito adicional de Segurança Informática, que é diretamente ligado ao conceito de Segurança da Informação, mas voltado não só para os dados mas para o próprio sistema em si.
A maioria das definições de Segurança da Informação (SI) pode ser sintetizada como a proteção contra o uso ou acesso não-autorizado as informações do usuário, assim como a proteção de diversos tipos de ataque como a negação de serviços a usuários autorizados, enquanto se mantém a integridade dessa informação e seu sigilo. Podem ser estabelecidas métricas (com o uso ou não de ferramentas) para a definição do nível de segurança existente e, com isto, para verificar se o sistema anualizado está dentro do necessário para ser seguro para uso ou não. A segurança de uma informação sempre está restrito ao usuário que a mantém, podendo ser extraída por pessoas mal intencionadas através de técnicas de engenharia social, como falsidade ideológica e estelionato. Um exemplo de ataque que pode ser bastante pesado para sistemas como são os sistemas de internet das coisas é o ataque distribuído de negação de serviço (também conhecido como DDoS, um acrônimo em inglês para Distributed Denial of Service), onde um computador mestre pode ter sob seu comando até milhares de computadores zumbis. Nesse caso, as tarefas de ataque de negação de serviço são distribuídas a um "exército" de máquinas escravizadas.

Grandes empresas precisam de técnicas muito sofisticadas para garantir que a integridade de seus dados seja mantida, uma vez que todo o tipo de invasor pode se interessar por eles, por conta do valor agregado. Grandes sites que possuem diversas informações precisam ter  ainda mais cuidado, pois um vazamento pode liberar nomes, documentos e contas bancárias de seus usuários, podendo gerar muito prejuízo e um grande transtorno para milhares de pessoas. Mecanismos para o controle dos sistemas envolvem mecanismos físicos, mecanismos lógicos, mecanismos de encriptação, assinaturas digitais, controle de acesso, certificação e diversos outros sistemas que tem sido desenvolvidos ao longo das últimas décadas para tentar tornar o acesso as informações por pessoas não autorizadas impossível. 

\subsection{Blockchain}
	Uma blockchain é um livro digital inviolável e de evolução contínua \cite{manzoor2018blockchain}.
	São bancos de dados distribuídos compartilhados, onde os usuários
    pode adicionar ou ler transações sem que uma única entidade tenha controle total, evitando assim manipulações fraudulentas. Blockchains são interessantes quando usados para aprimorar a segurança de ambientes não confiáveis e descentralizados, pois eles fornecem operacionalidade sem uma autoridade central. As transações adicionadas a um bloco na blockchain são criptografadas através de uma função hash e assinadas para garantir a integridade e confiabilidade da operação realizada. A assinatura digital e o registro de data e hora permitem que as operações sejam rastreadas, determinando assim a sua origem. Uma transação é então transmitida para todos os nós da rede para obter consenso, fornecendo integridade de dados. Os blockchains também suportam a execução confiável de código na forma de contratos inteligentes. Se um acordo existe (uma condição é atendida), então um contrato (conjunto de operações) é executado \cite{taylor2019systematic}.
    
    Apesar de blockchain ter sido difundido pelo seu uso em criptomoedas, como a bitcoin, existem diversas aplicações, para problemas totalmente diferentes, as vezes até inusitados. Por exemplo, o aplicativo de streaming de músicas Spotify adquiriu a startup de blockchain Mediachain Labs, como o objetivo de ajudar a desenvolver soluções por meio de um banco de dados descentralizado que pudesse melhor conectar artistas e acordos de licenciamento com as faixas disponíveis. Outro exemplo, dessa vez voltado para o varejo, é o aplicativo Waranteed, que é um aplicativo blockchain que permite, aos consumidores, acessarem facilmente informações sobre os produtos que adquiriram e obter assistência em caso de mau funcionamento. Grandes empresas também estão presentes nesse mercado. A IBM possui o IBM Blockchain. É essencial conhecer o status e a condição de todos os produtos de sua cadeia de suprimentos, desde a matéria-prima até a distribuição. A tecnologia blockchain da IBM para este segmento permite transparência por meio de um registro compartilhado de propriedade e localização de peças e produtos em tempo real.