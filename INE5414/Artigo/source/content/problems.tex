\section{Problemas Existentes}

Na sessão anterior tratamos de um aspecto que também é um problema relevante da segurança em IoT: Baixo poder computacional. Os avanços ao nível da miniaturização e da nanotecnologia significam que cada vez mais objetos pequenos terão a capacidade de interagir e se conectar, e que cada vez mais precisaremos de processadores menores. Entretanto, sabemos que a lei de Moore não é mais válida para a progressão na indústria de processadores, ou seja, essas processadores miniaturizados terão ainda menos capacidade de processamento, e soluções ainda melhores de segurança serão necessários nos próximos anos. Além disso, a complexidade de algoritmos de criptografia pode ser bastante elevada, e isso pode ser ainda mais problemático se a segurança não se restringir a simples criptografia das senhas, mas também abranger detecção de ataques de negação de serviço, tentativas de invasão e outros métodos mais complexos de tentativa de causar dano a rede em que o dispositivo está inserido.

Blockchain pode ser uma boa saída para esse problema de segurança, no entretanto, adotar uma solução blockchain no contexto da IoT não é simples e implica vários desafios significativos, tais como: alta demanda de recursos para resolver o POW (Proof of Work), latência longa para confirmação de transação, e baixa escalabilidade que é resultado de transações de transmissão e bloqueia toda a rede. Portanto, nesse trabalho tentamos investigar soluções blockchain utilizados em problema de segurança em IoT, mas as vezes a solução está intimamente ligada ao problema para o qual é proposto, dentro do escopo específico da rede, os dos aspectos físicos em que os dispositivos estão inseridos. O consumo energético é um problema uma vez que temos muitos dispositivos ligados em rede evidentemente possuem um alto consumo de energia, afinal de contas teremos uma grande quantidade de computadores ligados o dia inteiro, todos os dias do ano. Quando se trata de bitcoin, esse consumo chega a trazer grandes discussões do campo da moral por conta dos efeitos que o consumo de energia de milhões de computadores pode causar a natureza. Por outro lado, o estudo de \cite{mccook2014order} mostra que caso as moedas atuais fossem substituídas pelo bitcoin, o consumo de energia seria menor (além de outros benefícios como menor necessidade de alocação humana, ou de gastos com a manutenção da representação física do dinheiro).

Além disso, é um problema comum quando trabalhando com blockchain existir uma dificuldade em integrar uma solução existente a arquitetura blockchain. As aplicações da blockchain oferecem soluções que exigem mudanças significativas, ou a substituição completa de sistemas existentes. A fim de realizar a troca, as empresas precisam desenvolver uma estratégia de transição. Ou seja, soluções IoT que já existem hoje em dia podem ser completamente reestruturadas para suportarem uma aplicação blockchain. Historicamente, podemos traçar um paralelo com o caso do processador Intel Itanium \cite{crawford2000introducing}. Apesar da arquitetura da Intel ser superior a dos concorrentes da época, trazendo uma reestruturação dos sistemas tradicionais de 32 bits para um novo formato de 64 bits, conforme começavam a exigir as memórias com capacidades de armazenamento que não estavam cobertas pela extensão de um endereço de 32 bits. Entretanto, a apesar da arquitetura proposta ser teoricamente melhor, os concorrentes correram com o lançamento de processadores que traziam a arquitetura 32 bits antiga com certas adaptações não muito eficientes para suportar aplicações 64 bits. Por conta disso, a Intel se viu obrigada a lançar processadores similares ao concorrente, inferiores ao que a sua arquitetura nova propunha. Podemos traçar um paralelo direto disso com a implementação de redes blockchain para garantir a segurança em ambientes de internet das coisas: caso uma empresa tente trazer essa solução para o consumidor final, não é necessário apenas que ela tenha efetivamente um sistema teoricamente melhor que os concorrentes, é preciso fazer isso em tempo hábil para poder lançar na mesma velocidade que as outras empresas lançam seus produtos, com qualidade superior, métricas de avaliação equivalente e de preferência com custo final maior que a concorrência, conforme manda o capitalismo.

A mão de obra especializada necessária para implementar um sistema de internet das coisas que utiliza blockchain para garantir integridade de dados, sigilo e operabilidade do sistema pode ainda enfrentar o problema humano da falta de mão de obra qualificada para trabalhar no desenvolvimento, uma vez que atinge alguns nichos bastante específicos mas diferenciados entre si. Ter uma equipe fluente em todas as tecnologias apresentadas pode ser algo bastante difícil, pois envolve a área de segurança diretamente com a implementação do blockchain, da área de embarcados por conta dos sensores inteligentes dos dispositivos IoT, a área de programação distribuída e redes. Isso pode escalonar ainda mais caso queira ser implementado junto com o projeto inicial os conceitos de Fog computing, que adicionam ainda mais uma característica de rede, onde os dispositivos responderão a um terminal central e esse terminal fará um pré processamento para tudo aquilo que será enviado para a nuvem. Essa possivelmente é a maior dificuldade de implantar esse modelo em larga escala, uma vez que ainda não há maneira genéricas e abrangentes o suficiente para fazer todas essas ligações necessárias entre diferentes tecnologias. Isso provavelmente irá mudar em poucos anos, com o interesse que a indústria mostra nessas áreas todas e na ciẽncia de borda ligada a tudo isso.