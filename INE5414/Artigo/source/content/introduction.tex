\section{Introdução}
\subsection{Motivação}
	Internet das Coisas (ou IoT, sigla em inglês para Internet of Things) é não só um tópico que nos últimos anos vem ganho destaque na mídia como tem também ganho os holofotes dos especialistas em tecnologia. IoT é um conceito ligado ao crescimento da interconexão digital de objetos cotidianos com a internet, como eletrodomésticos, microcontroladores e variedades ainda mais simples, como lâmpadas e tomadas. Em resumo, é a conexão dos objetos, mais do que das pessoas, a internet. A interconexão dos sistemas permite um ambiente diário totalmente inteligente e eficiente. Portanto, um dos maiores impactos da IoT é, a partir da interpretação de dados recebidos pelos sensores dos dispositivos, a capacidade dos objetos de se comunicar com os usuários, promovendo melhorias na qualidade de vida, maior produtividade e agilidade nos processos. Além do impacto imediato na mudança do estilo de vida dos usuários, essa nova onda de produtividade permitirá que as pessoas foquem naquilo que não pode ser automatizado.
	Essa tecnologia tem ganhado espaço no vida das pessoas \cite{fan2018blockchain}. No entanto, essas tecnologias também representam graves riscos de privacidade e preocupações com a segurança dos dados \cite{dwivedi2019decentralized}. Suponha uma loja em que todos os sistemas estejam conectados, desde os as lâmpadas e a rede elétrica até os dados de compra e vendas do lugar. Sendo assim, se um hacker encontrar uma vulnerabilidade em algum desses sistemas e conseguir invadir a tomada, por exemplo, o mesmo poderá usá-la de porta de entrada para acessar outros sistemas dentro da loja. Portanto, garantir a segurança desse tipo de rede é algo essencial.


\subsection{Justificativas}
	A crescente no uso de dispositivos conectados a rede faz com que os fabricantes de IoT se concentraram na funcionalidade e nos recursos dos mesmos e que a segurança se tonasse algo secundário \cite{falco2019neuromesh}. Como dispositivos desse tipo geralmente possuem um processamento reduzido para um baixo consumo de energia, assim como possuem uma memória pequena para manter o custo barato, a tarefa de torná-los seguros é bastante árdua, e é um desafio ainda em aberto para a academia. Nos últimos anos, abordagens que utilizam blockchain tem se tornado uma boa alternativa para resolver esse problema, e diversas pesquisas tem surgido na área, mas apesar de fornecerem segurança e privacidade descentralizadas, envolvem energia significativa, atraso e sobrecarga computacional que é não é adequado para a maioria dos dispositivos IoT com recursos limitados \cite{dorri2017blockchain}. Então, apesar de blockchain ser uma alternativa bastante interessante para ser tomada, é preciso reduzir o custo computacional e elétrico envolvido para ter um custo benefício que valha a pena. A urgência da necessidade de modelos seguros de IoT pode ser sentida quando analisarmos os grandes players que estão no mercado de dispositivos para casa, como Google, Amazon e Apple, que já possuem milhares de casas com aparelhos inteligentes, como lâmpadas, assistentes pessoais, janelas automatizadas, robôs para limpeza do chão e diversos outros dispositivos que precisam ser seguros, caso contrário ataques de hackers podem deixar em perigo toda a infraestrutura das casas das pessoas que os utilizam. Proteger a rede IoT é um dos objetivos importantes de projetar novas arquiteturas distribuídas \cite{sharma2018software}.

\subsection{Objetivos}
\subsubsection{Gerais}
	O aumento acentuado dos aplicativos da Internet das Coisas (IoT), requer soluções robustas para o problema da violabilidade de seus dados. Por tanto, é preciso utilizar métodos computacionais que permitem tal robustês, como blockchain, para garantir a segurança não só da vida online de seus proprietários como até a segurança física que pode ser comprometida dependendo do tipo de aparelhos instalados em sua casa \cite{dorri2017towards}. Portanto o principal objetivo é garantir a segurança dos dispositivos IoT utilizando blockchain.

\subsubsection{Específicos}
\begin{itemize}
\item Compreender Internet das Coisas e blockchain;
\item Obter uma compreensão abrangente sobre como blockchain pode ser usado para garantir a segurança em IoT;
\item Fornecer uma visão geral sobre segurança em IoT em geral;
\item Identificar abordagens existentes, seus principais casos de uso, identificação dos principais problemas, e encaminhar possíveis soluções.
\end{itemize}

\subsection{Organização do Artigo}
	A seção 2 apresenta os conceitos básicos de internet das coisas, segurança e blockchain. Na seção 3 são apresentados a revisão bibliográfica sistemática, com a tabela de buscas, e os trabalhos correlatos, exemplificando a situação do estado da arte no uso de blockchain como solução para segurança em redes de aparelhos de internet das coisas. Na seção 4 são expostos os aspectos relevantes que remetem tanto ao problema quanto a solução, contendo uma discussão sobre as blockchains que já operam hoje em dia. Na seção 5 discutimos os problemas existentes no uso de blockchain para soluções de segurança em IoT, explorando aspectos de redes blockchain e sua aplicação no mercado. Na seção 6 apresentamos possíveis soluções, utilizando como base um caso de estudo de uma casa inteligente que utiliza redes blockchain para manter a segurança do sistema. Na seção 7, abstraímos o exemplo da seção 6 para um modelo teórico, que pode ser utilizado para modelar outras redes de internet das coisas, e mostramos resultados práticos e indicativos que colaboram para validar o modelo proposto. Por fim na seção 8 temos a conclusão e sugestões de trabalhos futuros a serem desenvolvidos dentro do tema do artigo.
	