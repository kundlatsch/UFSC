\section{Aspectos Relevantes}

Segurança é algo essencial em IoT, pois como já foi visto no presente trabalho, uma falha em um simples sensor de uma lâmpada, por exemplo, pode gerar perigo para a própria integridade física dos usuários, uma vez que essa simples falha pode ser a porta de acesso para outros dispositivos conectados a internet, como trancas das portas da casa ou o portão da frente de entrada de carros, permitindo um indivíduo mal intencionado invadir a casa em questão. Certos dispositivos IoT são tão simples, de processador tão fraco e com uma capacidade de memória tão pequena, que certos desenvolvedores cometem falhas grosseiras, como guardar senhas de wi-fi sem criptografia por falta de recurso computacional para rodar um algoritmo de encriptação.

Essas são as duas principais características a serem consideradas quando queremos criar uma solução para segurança em IoT: relevância de uma solução robusta devido ao caráter crítico das aplicações, e o baixo custo computacional necessário em tais aplicações, para que possam ser executadas e qualquer dispositivo de IoT. Infelizmente, isso é um ideal longe de ser real, e portanto o máximo que pode ser feito é tentar otimizar ao máximo as soluções para que se aproximem do esperado.

O importante de criar uma solução da maneira mais abstrata e genérica possível, mas passível de ser implementada na prática, é abrangir o grande número de dispositivos existentes em uma rede IoT que podem ser considerados voláteis: por conta do custo baixo de certos sensores inteligentes, é necessário prever que estes dispositivos podem estragar a qualquer momento, seja por uma falha mecânica, um curto circuito, uma ação externa do ambiente como um vento forte que derruba o equipamento ou qualquer outro problema que pode eventualmente surgir. Uma arquitetura que suporta essa volatilidade da rede é necessária, para que dispositivos possam ser removidos e inseridos sem que toda a rede seja afetada ou até deixe de funcionar (essa solução segue o princípio de disponibilidade de serviço, vindo da segurança).

Outro tópico tratado nesse artigo é blockchain, que por sua vez possui alguns aspectos interessantes. Blockchain é uma tecnologia que recentemente tem ganhado muita força na indústria\cite{bcindustry}, que usa registro distribuído visando a descentralização como medida de segurança. São bases de registros e dados distribuídos e compartilhados que têm a função de criar um índice global para todas as transações que ocorrem em um determinado escopo. Sua principal proposta e diferencial é criar consenso e confiança na comunicação direta entre duas partes, ou seja, sem o intermédio de terceiros.

Um ponto a ser considerado quando estamos trabalhando com a tecnologia blockchain é o algoritmo de consenso \cite{zheng2017overview}. No blockchain, o algoritmo de consenso é um algotimo cuja finalidade é resolver problemas de confiança, ou seja, nenhum dado que for inserido na rede pode ser simplesmente apagado sem deixar rastros, pois todas as novas inserções devem ser validadas por todos os elementos do conjunto de equipamentos que estão ligados a rede blockchain. Para isto, deve ser utilizado uma regra (algoritmo) que descreve como deve ser feita a inclusão de novos dados. Para um novo elemento ser incluído (ou minerado, como ficou mais popularmente conhecido) é preciso que toda a rede valide a inserção. Um fator primordial do algoritmo de consenso é que ele não pode depender de uma entidade centralizadora, ou seja, precisa ser independente. É por meio dessas regras de consenso que cada usuário da rede pode verificar a autenticidade de todos os dados inseridos em um bloco. Esse mecanismo permite que ninguém precise confiar em ninguém, pois a verificação através de regras faz com que cada usuário possa ter confiança na rede blockchain como um todo.

É fácil compreender esse conceito de consenso, e os benefícios de se ter uma rede onde cada elemento da rede não precisa confiar nos outros, apenas na estrutura em que está inserido, quando traçamos um paralelo com o bitcoin \cite{nakamoto2008bitcoin}. O bitcoin foi a criptomoeda que criou e popularizou o conceito de blockchain. Criada em 2008, nos últimos dois anos ganhou uma popularidade gigantesca ao ponto de qualquer pessoa conseguir facilmente comprar frações de bitcoins em um banco, mesmo sem entender a tecnologia por trás das criptomoedas. O conceito de mineração fica muito claro quando entendemos que o ato de minerar um bitcoin é oferecer potencial computacional do seu computador, servidor ou rede de mineração para realizar os cálculos necessários para validar as regras de consenso que foram citadas. Assim, ao gastar um bitcoin não é necessário que cada computador do mundo que minera valide a transação, pois já há confiança na rede como um todo, adquirido no processo de mineração, e o processo de verificação de um token de uma carteira de bitcoins é muito mais fácil, pois é o caminho contráio (já foi encontrado um valor que satisfaz as regras, agora basta fazer os cálculos necessários para verificar se ele é válido). Esse conceito simples porém poderoso que fez com que as criptomoedas tivessem um crescimento gigantesco nos últimos anos \cite{vasek2015there}, se tornando um assunto da moda em todo o mundo.

Ainda é possível traçar um paralelo entre Fog Computing \cite{bonomi2014fog} e a solução abordada nesse trabalho, do uso de blockchain como solução de segurança para internet das coisas. O volume de dados utilizado para processar a rede blockchain pode se beneficiar bastante do pré processamento fornecido pela Fog Computing, que evita a transferência da grande massa de dados coletados pelos sensores inteligentes dos diversos dispositivos IoT para a núvem, tendo um custo de processamento e transmissão. Apesar de Fog Computing não estar no escopo deste trabalho, é uma solução que aparenta se encaixar bem dentro das características e aspectos relevantes do problema tratado, e que poderia ser implementado em um trabalho futuro como um complemento a arquitetura genérica apresentada mais a frente, na seção 7.