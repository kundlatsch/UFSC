\section{Conclusões e Trabalhos Futuros}

A ideia desse trabalho foi fazer uma revisão sistemática da literatura até então existente, apresentação de um modelo genérico para implementar a solução proposta e ter uma discussão dos benefícios, desafios e do futuro de segurança em internet das coisas, em específico do uso de técnicas da blockchain para tratar desses problemas. 

O modelo apresentado, que descreve os sistemas apresentados na revisão sistemática, pode servir como base para diversas implementações de sistemas de segurança para redes de internet das coisas utilizando blockchain. Apesar de servir muito bem, e já ter sido provado eficiente através de implementações iniciais, ainda existem problemas a serem resolvidos na área como um todo. A maior dificuldade, sem dúvida, é o alto consumo energético, derivado do requisito computacional que os algoritmos de segurança blockchain requerem, além da memória utilizada. Apesar do resultado da implementação específica ter sido muito satisfatório tanto em custo computacional quanto em custo energético, mais estudo na área é necessário para poder se chegar a conclusões concretas e que abranjam de maneira mais geral modelos segurança para redes do tipo. Como os sensores e microprocessadores utilizados pelas ``coisas'' nas redes de IoT serem muito simples, com baixo potencial computacional, baixa memória e necessidade de baixo consumo energético por questões térmicas, muito dos algoritmos de blockchain precisam ser revistos para que esse tipo de solução se torne simples de ser implantada e comum de ser vista nas redes domésticas, industriais e científicas.

O principal trabalho discutido em nossa revisão, e que foi utilizado como base para montarmos uma proposta concreta de solução genérica de segurança para internet das coisas utilizando blockchain \cite{dorri2017blockchain}, mostra como é possível fazer implementações do tipo, pois no trabalho realmente foi feita uma rede completa de internet das coisas para que os testes fossem realizados, tanto com blockchain quanto sem blockchain. Apesar do exemplo com blockchain ter apresentado resultados espetaculares com um custo de energia bem reduzido e tempo de resposta adicional quase desconsideráveis, mesmo que maior do que a implementação sem blockchain, que em questão de segurança era muito inferior devido a natureza simples das verficações de segurança e confiabilidade da rede, a replicação em larga escala do mesmo modelo ainda assim poderia trazer problemas por conta de um crescimento linear do custo. Mesmo assim, esse trabalho é pioneiro na área de otimização do tipo de solução apresentada, e tende a ser um tipo de trabalho a se repetir em novas esferas, com contextos diferentes e implementações ainda mais otimizadas que vão tornar possível a indústria aplicar esse tipo de solução a qualquer rede de coisas inteligentes a serem vendidas para o usuário final.

Em nosso trabalho apresentamos um modelo genérico, que possui certa validação mas que ainda pode ser aprimorado e testado arduamente para conseguir dados mais objetivos e relevantes para que a proposta como um todo se concretize como uma solução válida e implementável para redes de diversos tamanhos. Caso isso não aconteça, é certo que para determinados tipos e tamanhos de redes de internet das coisas o modelo funciona, então mesmo assim pode ser utilizado, apenas em empreitadas específicas e talvez reduzidas.

Portanto, os trabalhos futuros devem almejar não só a corretude da solução proposta, com modelos que já foram apresentados como funcionais por trabalhos anteriores, mas também focar na otimização das soluções para que tenham necessidade de menor custo computacional e consumo energético, para que em aplicações de larga escala a solução utilizando blockchain continue viável, mesmo com possivelmente milhões de coisas inteligentes interconectadas. Uma alternativa para essa solução é investir em soluções distribuídas, onde a computação pode ser distribuída de maneira eficiente através de algoritmos de distribuição de carga entre os diversos microprocessadores da rede IoT, extraindo o máximo potencial que essa tecnologia tem a oferecer com sua natureza de muitos dispositivos com pouco potencial de computação individual.

Além do avanço no trabalho atual como citado no parágrafo anterior, a arquitetura apresentada, bem como as contribuições que ela traz para a segurança em internet das coisas mesclando o crescente uso de controladores inteligentes com conceitos como blockchain, podem ser ainda mais mesclados com conceitos avançados de computação, IoT ou segurança. Um exemplo claro disso é a possibilidade de aplicar a técnica de Fog computing no modelo apresentado. Essa adição está fora do escopo do trabalho realizado, mas combina perfeitamente com a ideia de reduzir custo computacional de maneira geral na rede de internet das coisas e nesse caso específico para reduzir o custo computacional da implementação do Blockchain em um ambiente com diversos sensores transmitindo um grande volume de dados constantemente.

Por último, vale ressaltar a importância da transparência e da comunicação humana além das estratégias de segurança adotada. É crucial que o consumidor saiba que tipo de dados estão sendo coletados, e que tipo de sensores existem nos dispositivos inteligentes instalados em sua casa para que o relacionamento entre o consumidor e a rede utilizada seja saudável, despertando não só o interesse comercial com um aproveito unilateral por parte das empresas que provém esse tipo de serviço. Aviso da escolha, do mecanismo de acesso e da precisão da informação recolhida, políticas efetivas de minimização dos dados coletados e prestação de contas das medidas e falhas de segurança nos dispositivos são alguns fundamentos que tem se mostrado tendência por empresas que presam por um relacionamento saudável com o consumidor.
